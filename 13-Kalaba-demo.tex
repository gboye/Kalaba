\documentclass[a4paper,12pt]{article}
\usepackage[utf8]{inputenc}
\usepackage{linearb}
%\usepackage{coffee4}
\usepackage{graphicx}
\usepackage{fourier}
\usepackage{tipa,booktabs,multicol,geometry,pifont,gb4e,natbib,relsize}

%\usepackage[T1]{fontenc}
%\usepackage{baskervald}


\geometry{scale={0.91,0.86}} % scale={largeur,hauteur}

\newcommand{\blanc}[1]{\textsc{#1}}
\newcommand{\cacherGloses}[1]{\tiny{}#1\normalsize{}}
\renewcommand{\blanc}[1]{\tiny{}#1\normalsize{}}
%\renewcommand{\blanc}[1]{}
\newcommand{\commentaire}[1]{}
\newcommand{\refp}[1]{(\ref{#1})}
\newcommand{\key}[1]{}
\newcommand{\strutgb}[1]{\rule[-#1]{0pt}{#1}}
\newcommand{\strutgh}[1]{\rule{0pt}{#1}}
\newcommand{\grapho}[1]{\footnotesize\textlinb{#1}}
%\renewcommand{\grapho}[1]{\tiny}

\newcommand{\blot}[2]{\includegraphics[scale=#2]{#1}}
\newcommand{\cachea}[1]{\raisebox{-8pt}[0pt][0pt]{\hspace{-2pt}\makebox[0pt][l]{\blot{blot1n.png}{.1}}}\hspace{2pt}#1}
\newcommand{\cacheb}[1]{\raisebox{-8pt}[0pt][0pt]{\hspace{-2pt}\makebox[0pt][l]{\blot{blot2.png}{.09}}}\hspace{2pt}#1}
\newcommand{\cachec}[1]{\raisebox{-12pt}[0pt][0pt]{\hspace{-10pt}\makebox[0pt][l]{\blot{blot3.png}{.17}}}\hspace{10pt}#1}
\newcommand{\cached}[1]{\raisebox{-10pt}[0pt][0pt]{\hspace{-9pt}\makebox[0pt][l]{\blot{blot4.png}{.3}}}\hspace{9pt}#1}
\newcommand{\cachee}[1]{\raisebox{-8pt}[0pt][0pt]{\hspace{-8pt}\makebox[0pt][l]{\blot{blot5.png}{.15}}}\hspace{8pt}#1}
\newcommand{\cachef}[1]{\raisebox{-14pt}[0pt][0pt]{\hspace{-8pt}\makebox[0pt][l]{\blot{blot6.png}{.15}}}\hspace{8pt}#1}
\newcommand{\cacheg}[1]{\raisebox{-8pt}[0pt][0pt]{\hspace{-0pt}\makebox[0pt][l]{\blot{blot7.png}{.12}}}\hspace{0pt}#1}
\newcommand{\cacheh}[1]{#1}
%\newcommand{\cachei}[1]{\raisebox{-8pt}[0pt][0pt]{\hspace{-2pt}\makebox[0pt][l]{\blot{blot1g.png}{.1}}}\hspace{2pt}#1}
%\newcommand{\cachej}[1]{\raisebox{-8pt}[0pt][0pt]{\hspace{-2pt}\makebox[0pt][l]{\blot{blot1d.png}{.1}}}\hspace{2pt}#1}

\let\gbtextipa=\textipa
\renewcommand{\textipa}[1]{\textsf{\gbtextipa{#1}}}

\newcommand{\textecursif}[1]{\fontfamily{pzc}\large\emph{#1}\fontfamily{ptm}}

\newenvironment{reponse}{\begin{quote}\begin{sffamily}}{\end{sffamily}\end{quote}}
\renewenvironment{reponse}[1]{}{}

\newcommand{\observation}[1]{\linebreak\parbox{.9\textwidth}{\noindent%
\textecursif{#1}%
\medskip}%parbox
}

\title{\vspace{-3\baselineskip}Description grammaticale}
% \author{}
\date{\vspace{-2\baselineskip}\Large Devoir à rendre \textbf{avant} le 11 décembre, 12h.\\
\normalsize une copie par groupe de 4 ou 3 personnes}
\begin{document}
\maketitle
%\pagestyle{empty}
\thispagestyle{empty}

\newcommand{\indice}[1]{\relsize{-2}{\ensuremath{_{\textnormal{#1}}}}}
\newcommand{\fldr}{\ensuremath{\longrightarrow}}


\section{\textbf{Analyse de corpus kalaba} (80\%)}\setcounter{exx}{0}%
%\vspace{-1\baselineskip}%
\noindent
Imaginez que vous faites partie d'une équipe chargée de fabriquer une grammaire du 
kalaba. Un de vos collègues, un linguiste débutant, s'est déjà rendu sur le terrain, 
il a constitué un échantillon et a presque terminé de transcrire les données.
\bigskip
%\key{%
%\pagebreak
%}


%: Lexique
\textbf{Lexique}\label{sect-lex}



\newcommand{\noirASg}{\strutgb{0pt}\grapho{gYq}}\newcommand{\noirASgP}{\textipa{dosiv}}\newcommand{\noirASgG}{noir\cacherGloses{.1}\cacherGloses{-1.Sg}\cacherGloses{-1.A}}\newcommand{\noirADu}{\strutgb{0pt}\grapho{gsq}}\newcommand{\noirADuP}{\textipa{dosav}}\newcommand{\noirADuG}{noir\cacherGloses{.1}\cacherGloses{-Du}\cacherGloses{-1.A}}\newcommand{\noirAPl}{\strutgb{0pt}\grapho{g1q}}\newcommand{\noirAPlP}{\textipa{dosov}}\newcommand{\noirAPlG}{noir\cacherGloses{.1}\cacherGloses{-1.Pl}\cacherGloses{-1.A}}\newcommand{\noirBSg}{\strutgb{0pt}\grapho{gYq}}\newcommand{\noirBSgP}{\textipa{dosif}}\newcommand{\noirBSgG}{noir\cacherGloses{.1}\cacherGloses{-1.Sg}\cacherGloses{-1.B}}\newcommand{\noirBDu}{\strutgb{0pt}\grapho{gsq}}\newcommand{\noirBDuP}{\textipa{dosaf}}\newcommand{\noirBDuG}{noir\cacherGloses{.1}\cacherGloses{-Du}\cacherGloses{-1.B}}\newcommand{\noirBPl}{\strutgb{0pt}\grapho{g1q}}\newcommand{\noirBPlP}{\textipa{dosof}}\newcommand{\noirBPlG}{noir\cacherGloses{.1}\cacherGloses{-1.Pl}\cacherGloses{-1.B}}\newcommand{\noirCSg}{\strutgb{0pt}\grapho{gYd}}\newcommand{\noirCSgP}{\textipa{dosid}}\newcommand{\noirCSgG}{noir\cacherGloses{.1}\cacherGloses{-1.Sg}\cacherGloses{-1.C}}\newcommand{\noirCDu}{\strutgb{0pt}\grapho{gsd}}\newcommand{\noirCDuP}{\textipa{dosad}}\newcommand{\noirCDuG}{noir\cacherGloses{.1}\cacherGloses{-Du}\cacherGloses{-1.C}}\newcommand{\noirCPl}{\strutgb{0pt}\grapho{g1d}}\newcommand{\noirCPlP}{\textipa{dosod}}\newcommand{\noirCPlG}{noir\cacherGloses{.1}\cacherGloses{-1.Pl}\cacherGloses{-1.C}}\newcommand{\noirDSg}{\strutgb{0pt}\grapho{gYk}}\newcommand{\noirDSgP}{\textipa{dosik}}\newcommand{\noirDSgG}{noir\cacherGloses{.1}\cacherGloses{-1.Sg}\cacherGloses{-1.D}}\newcommand{\noirDDu}{\strutgb{0pt}\grapho{gsk}}\newcommand{\noirDDuP}{\textipa{dosak}}\newcommand{\noirDDuG}{noir\cacherGloses{.1}\cacherGloses{-Du}\cacherGloses{-1.D}}\newcommand{\noirDPl}{\strutgb{0pt}\grapho{g1k}}\newcommand{\noirDPlP}{\textipa{dosok}}\newcommand{\noirDPlG}{noir\cacherGloses{.1}\cacherGloses{-1.Pl}\cacherGloses{-1.D}}\newcommand{\grandASg}{\strutgb{0pt}\grapho{MCq}}\newcommand{\grandASgP}{\textipa{meNiv}}\newcommand{\grandASgG}{grand\cacherGloses{.1}\cacherGloses{-1.Sg}\cacherGloses{-1.A}}\newcommand{\grandADu}{\strutgb{0pt}\grapho{Mnq}}\newcommand{\grandADuP}{\textipa{meNav}}\newcommand{\grandADuG}{grand\cacherGloses{.1}\cacherGloses{-Du}\cacherGloses{-1.A}}\newcommand{\grandAPl}{\strutgb{0pt}\grapho{MEq}}\newcommand{\grandAPlP}{\textipa{meNov}}\newcommand{\grandAPlG}{grand\cacherGloses{.1}\cacherGloses{-1.Pl}\cacherGloses{-1.A}}\newcommand{\grandBSg}{\strutgb{0pt}\grapho{MCq}}\newcommand{\grandBSgP}{\textipa{meNif}}\newcommand{\grandBSgG}{grand\cacherGloses{.1}\cacherGloses{-1.Sg}\cacherGloses{-1.B}}\newcommand{\grandBDu}{\strutgb{0pt}\grapho{Mnq}}\newcommand{\grandBDuP}{\textipa{meNaf}}\newcommand{\grandBDuG}{grand\cacherGloses{.1}\cacherGloses{-Du}\cacherGloses{-1.B}}\newcommand{\grandBPl}{\strutgb{0pt}\grapho{MEq}}\newcommand{\grandBPlP}{\textipa{meNof}}\newcommand{\grandBPlG}{grand\cacherGloses{.1}\cacherGloses{-1.Pl}\cacherGloses{-1.B}}\newcommand{\grandCSg}{\strutgb{0pt}\grapho{MCd}}\newcommand{\grandCSgP}{\textipa{meNid}}\newcommand{\grandCSgG}{grand\cacherGloses{.1}\cacherGloses{-1.Sg}\cacherGloses{-1.C}}\newcommand{\grandCDu}{\strutgb{0pt}\grapho{Mnd}}\newcommand{\grandCDuP}{\textipa{meNad}}\newcommand{\grandCDuG}{grand\cacherGloses{.1}\cacherGloses{-Du}\cacherGloses{-1.C}}\newcommand{\grandCPl}{\strutgb{0pt}\grapho{MEd}}\newcommand{\grandCPlP}{\textipa{meNod}}\newcommand{\grandCPlG}{grand\cacherGloses{.1}\cacherGloses{-1.Pl}\cacherGloses{-1.C}}\newcommand{\grandDSg}{\strutgb{0pt}\grapho{MCk}}\newcommand{\grandDSgP}{\textipa{meNik}}\newcommand{\grandDSgG}{grand\cacherGloses{.1}\cacherGloses{-1.Sg}\cacherGloses{-1.D}}\newcommand{\grandDDu}{\strutgb{0pt}\grapho{Mnk}}\newcommand{\grandDDuP}{\textipa{meNak}}\newcommand{\grandDDuG}{grand\cacherGloses{.1}\cacherGloses{-Du}\cacherGloses{-1.D}}\newcommand{\grandDPl}{\strutgb{0pt}\grapho{MEk}}\newcommand{\grandDPlP}{\textipa{meNok}}\newcommand{\grandDPlG}{grand\cacherGloses{.1}\cacherGloses{-1.Pl}\cacherGloses{-1.D}}\newcommand{\petitASg}{\strutgb{0pt}\grapho{Vcq}}\newcommand{\petitASgP}{\textipa{rugiv}}\newcommand{\petitASgG}{petit\cacherGloses{.1}\cacherGloses{-1.Sg}\cacherGloses{-1.A}}\newcommand{\petitADu}{\strutgb{0pt}\grapho{Vkq}}\newcommand{\petitADuP}{\textipa{rugav}}\newcommand{\petitADuG}{petit\cacherGloses{.1}\cacherGloses{-Du}\cacherGloses{-1.A}}\newcommand{\petitAPl}{\strutgb{0pt}\grapho{Vhq}}\newcommand{\petitAPlP}{\textipa{rugov}}\newcommand{\petitAPlG}{petit\cacherGloses{.1}\cacherGloses{-1.Pl}\cacherGloses{-1.A}}\newcommand{\petitBSg}{\strutgb{0pt}\grapho{Vcq}}\newcommand{\petitBSgP}{\textipa{rugif}}\newcommand{\petitBSgG}{petit\cacherGloses{.1}\cacherGloses{-1.Sg}\cacherGloses{-1.B}}\newcommand{\petitBDu}{\strutgb{0pt}\grapho{Vkq}}\newcommand{\petitBDuP}{\textipa{rugaf}}\newcommand{\petitBDuG}{petit\cacherGloses{.1}\cacherGloses{-Du}\cacherGloses{-1.B}}\newcommand{\petitBPl}{\strutgb{0pt}\grapho{Vhq}}\newcommand{\petitBPlP}{\textipa{rugof}}\newcommand{\petitBPlG}{petit\cacherGloses{.1}\cacherGloses{-1.Pl}\cacherGloses{-1.B}}\newcommand{\petitCSg}{\strutgb{0pt}\grapho{Vcd}}\newcommand{\petitCSgP}{\textipa{rugid}}\newcommand{\petitCSgG}{petit\cacherGloses{.1}\cacherGloses{-1.Sg}\cacherGloses{-1.C}}\newcommand{\petitCDu}{\strutgb{0pt}\grapho{Vkd}}\newcommand{\petitCDuP}{\textipa{rugad}}\newcommand{\petitCDuG}{petit\cacherGloses{.1}\cacherGloses{-Du}\cacherGloses{-1.C}}\newcommand{\petitCPl}{\strutgb{0pt}\grapho{Vhd}}\newcommand{\petitCPlP}{\textipa{rugod}}\newcommand{\petitCPlG}{petit\cacherGloses{.1}\cacherGloses{-1.Pl}\cacherGloses{-1.C}}\newcommand{\petitDSg}{\strutgb{0pt}\grapho{Vck}}\newcommand{\petitDSgP}{\textipa{rugik}}\newcommand{\petitDSgG}{petit\cacherGloses{.1}\cacherGloses{-1.Sg}\cacherGloses{-1.D}}\newcommand{\petitDDu}{\strutgb{0pt}\grapho{Vkk}}\newcommand{\petitDDuP}{\textipa{rugak}}\newcommand{\petitDDuG}{petit\cacherGloses{.1}\cacherGloses{-Du}\cacherGloses{-1.D}}\newcommand{\petitDPl}{\strutgb{0pt}\grapho{Vhk}}\newcommand{\petitDPlP}{\textipa{rugok}}\newcommand{\petitDPlG}{petit\cacherGloses{.1}\cacherGloses{-1.Pl}\cacherGloses{-1.D}}\newcommand{\blancASg}{\strutgb{0pt}\grapho{EOq}}\newcommand{\blancASgP}{\textipa{noriv}}\newcommand{\blancASgG}{blanc\cacherGloses{.1}\cacherGloses{-1.Sg}\cacherGloses{-1.A}}\newcommand{\blancADu}{\strutgb{0pt}\grapho{Erq}}\newcommand{\blancADuP}{\textipa{norav}}\newcommand{\blancADuG}{blanc\cacherGloses{.1}\cacherGloses{-Du}\cacherGloses{-1.A}}\newcommand{\blancAPl}{\strutgb{0pt}\grapho{EUq}}\newcommand{\blancAPlP}{\textipa{norov}}\newcommand{\blancAPlG}{blanc\cacherGloses{.1}\cacherGloses{-1.Pl}\cacherGloses{-1.A}}\newcommand{\blancBSg}{\strutgb{0pt}\grapho{EOq}}\newcommand{\blancBSgP}{\textipa{norif}}\newcommand{\blancBSgG}{blanc\cacherGloses{.1}\cacherGloses{-1.Sg}\cacherGloses{-1.B}}\newcommand{\blancBDu}{\strutgb{0pt}\grapho{Erq}}\newcommand{\blancBDuP}{\textipa{noraf}}\newcommand{\blancBDuG}{blanc\cacherGloses{.1}\cacherGloses{-Du}\cacherGloses{-1.B}}\newcommand{\blancBPl}{\strutgb{0pt}\grapho{EUq}}\newcommand{\blancBPlP}{\textipa{norof}}\newcommand{\blancBPlG}{blanc\cacherGloses{.1}\cacherGloses{-1.Pl}\cacherGloses{-1.B}}\newcommand{\blancCSg}{\strutgb{0pt}\grapho{EOd}}\newcommand{\blancCSgP}{\textipa{norid}}\newcommand{\blancCSgG}{blanc\cacherGloses{.1}\cacherGloses{-1.Sg}\cacherGloses{-1.C}}\newcommand{\blancCDu}{\strutgb{0pt}\grapho{Erd}}\newcommand{\blancCDuP}{\textipa{norad}}\newcommand{\blancCDuG}{blanc\cacherGloses{.1}\cacherGloses{-Du}\cacherGloses{-1.C}}\newcommand{\blancCPl}{\strutgb{0pt}\grapho{EUd}}\newcommand{\blancCPlP}{\textipa{norod}}\newcommand{\blancCPlG}{blanc\cacherGloses{.1}\cacherGloses{-1.Pl}\cacherGloses{-1.C}}\newcommand{\blancDSg}{\strutgb{0pt}\grapho{EOk}}\newcommand{\blancDSgP}{\textipa{norik}}\newcommand{\blancDSgG}{blanc\cacherGloses{.1}\cacherGloses{-1.Sg}\cacherGloses{-1.D}}\newcommand{\blancDDu}{\strutgb{0pt}\grapho{Erk}}\newcommand{\blancDDuP}{\textipa{norak}}\newcommand{\blancDDuG}{blanc\cacherGloses{.1}\cacherGloses{-Du}\cacherGloses{-1.D}}\newcommand{\blancDPl}{\strutgb{0pt}\grapho{EUk}}\newcommand{\blancDPlP}{\textipa{norok}}\newcommand{\blancDPlG}{blanc\cacherGloses{.1}\cacherGloses{-1.Pl}\cacherGloses{-1.D}}\newcommand{\basASg}{\strutgb{0pt}\grapho{Ocq}}\newcommand{\basASgP}{\textipa{likiv}}\newcommand{\basASgG}{bas\cacherGloses{.1}\cacherGloses{-1.Sg}\cacherGloses{-1.A}}\newcommand{\basADu}{\strutgb{0pt}\grapho{Okq}}\newcommand{\basADuP}{\textipa{likav}}\newcommand{\basADuG}{bas\cacherGloses{.1}\cacherGloses{-Du}\cacherGloses{-1.A}}\newcommand{\basAPl}{\strutgb{0pt}\grapho{Ohq}}\newcommand{\basAPlP}{\textipa{likov}}\newcommand{\basAPlG}{bas\cacherGloses{.1}\cacherGloses{-1.Pl}\cacherGloses{-1.A}}\newcommand{\basBSg}{\strutgb{0pt}\grapho{Ocq}}\newcommand{\basBSgP}{\textipa{likif}}\newcommand{\basBSgG}{bas\cacherGloses{.1}\cacherGloses{-1.Sg}\cacherGloses{-1.B}}\newcommand{\basBDu}{\strutgb{0pt}\grapho{Okq}}\newcommand{\basBDuP}{\textipa{likaf}}\newcommand{\basBDuG}{bas\cacherGloses{.1}\cacherGloses{-Du}\cacherGloses{-1.B}}\newcommand{\basBPl}{\strutgb{0pt}\grapho{Ohq}}\newcommand{\basBPlP}{\textipa{likof}}\newcommand{\basBPlG}{bas\cacherGloses{.1}\cacherGloses{-1.Pl}\cacherGloses{-1.B}}\newcommand{\basCSg}{\strutgb{0pt}\grapho{Ocd}}\newcommand{\basCSgP}{\textipa{likid}}\newcommand{\basCSgG}{bas\cacherGloses{.1}\cacherGloses{-1.Sg}\cacherGloses{-1.C}}\newcommand{\basCDu}{\strutgb{0pt}\grapho{Okd}}\newcommand{\basCDuP}{\textipa{likad}}\newcommand{\basCDuG}{bas\cacherGloses{.1}\cacherGloses{-Du}\cacherGloses{-1.C}}\newcommand{\basCPl}{\strutgb{0pt}\grapho{Ohd}}\newcommand{\basCPlP}{\textipa{likod}}\newcommand{\basCPlG}{bas\cacherGloses{.1}\cacherGloses{-1.Pl}\cacherGloses{-1.C}}\newcommand{\basDSg}{\strutgb{0pt}\grapho{Ock}}\newcommand{\basDSgP}{\textipa{likik}}\newcommand{\basDSgG}{bas\cacherGloses{.1}\cacherGloses{-1.Sg}\cacherGloses{-1.D}}\newcommand{\basDDu}{\strutgb{0pt}\grapho{Okk}}\newcommand{\basDDuP}{\textipa{likak}}\newcommand{\basDDuG}{bas\cacherGloses{.1}\cacherGloses{-Du}\cacherGloses{-1.D}}\newcommand{\basDPl}{\strutgb{0pt}\grapho{Ohk}}\newcommand{\basDPlP}{\textipa{likok}}\newcommand{\basDPlG}{bas\cacherGloses{.1}\cacherGloses{-1.Pl}\cacherGloses{-1.D}}\newcommand{\quatreASg}{\strutgb{0pt}\grapho{jihm}}\newcommand{\quatreASgP}{\textipa{jigom}}\newcommand{\quatreASgG}{quatre\cacherGloses{.2}\cacherGloses{-2.Sg}\cacherGloses{-2.A}}\newcommand{\quatreADu}{\strutgb{0pt}\grapho{jikm}}\newcommand{\quatreADuP}{\textipa{jigam}}\newcommand{\quatreADuG}{quatre\cacherGloses{.2}\cacherGloses{-Du}\cacherGloses{-2.A}}\newcommand{\quatreAPl}{\strutgb{0pt}\grapho{jicm}}\newcommand{\quatreAPlP}{\textipa{jigim}}\newcommand{\quatreAPlG}{quatre\cacherGloses{.2}\cacherGloses{-2.Pl}\cacherGloses{-2.A}}\newcommand{\quatreBSg}{\strutgb{0pt}\grapho{jihq}}\newcommand{\quatreBSgP}{\textipa{jigov}}\newcommand{\quatreBSgG}{quatre\cacherGloses{.2}\cacherGloses{-2.Sg}\cacherGloses{-2.B}}\newcommand{\quatreBDu}{\strutgb{0pt}\grapho{jikq}}\newcommand{\quatreBDuP}{\textipa{jigav}}\newcommand{\quatreBDuG}{quatre\cacherGloses{.2}\cacherGloses{-Du}\cacherGloses{-2.B}}\newcommand{\quatreBPl}{\strutgb{0pt}\grapho{jicq}}\newcommand{\quatreBPlP}{\textipa{jigiv}}\newcommand{\quatreBPlG}{quatre\cacherGloses{.2}\cacherGloses{-2.Pl}\cacherGloses{-2.B}}\newcommand{\quatreCSg}{\strutgb{0pt}\grapho{jiht}}\newcommand{\quatreCSgP}{\textipa{jigot}}\newcommand{\quatreCSgG}{quatre\cacherGloses{.2}\cacherGloses{-2.Sg}\cacherGloses{-2.C}}\newcommand{\quatreCDu}{\strutgb{0pt}\grapho{jikt}}\newcommand{\quatreCDuP}{\textipa{jigat}}\newcommand{\quatreCDuG}{quatre\cacherGloses{.2}\cacherGloses{-Du}\cacherGloses{-2.C}}\newcommand{\quatreCPl}{\strutgb{0pt}\grapho{jict}}\newcommand{\quatreCPlP}{\textipa{jigit}}\newcommand{\quatreCPlG}{quatre\cacherGloses{.2}\cacherGloses{-2.Pl}\cacherGloses{-2.C}}\newcommand{\quatreDSg}{\strutgb{0pt}\grapho{jihk}}\newcommand{\quatreDSgP}{\textipa{jigog}}\newcommand{\quatreDSgG}{quatre\cacherGloses{.2}\cacherGloses{-2.Sg}\cacherGloses{-2.D}}\newcommand{\quatreDDu}{\strutgb{0pt}\grapho{jikk}}\newcommand{\quatreDDuP}{\textipa{jigag}}\newcommand{\quatreDDuG}{quatre\cacherGloses{.2}\cacherGloses{-Du}\cacherGloses{-2.D}}\newcommand{\quatreDPl}{\strutgb{0pt}\grapho{jick}}\newcommand{\quatreDPlP}{\textipa{jigig}}\newcommand{\quatreDPlG}{quatre\cacherGloses{.2}\cacherGloses{-2.Pl}\cacherGloses{-2.D}}\newcommand{\troisASg}{\strutgb{0pt}\grapho{p4m}}\newcommand{\troisASgP}{\textipa{batom}}\newcommand{\troisASgG}{trois\cacherGloses{.2}\cacherGloses{-2.Sg}\cacherGloses{-2.A}}\newcommand{\troisADu}{\strutgb{0pt}\grapho{ptm}}\newcommand{\troisADuP}{\textipa{batam}}\newcommand{\troisADuG}{trois\cacherGloses{.2}\cacherGloses{-Du}\cacherGloses{-2.A}}\newcommand{\troisAPl}{\strutgb{0pt}\grapho{p3m}}\newcommand{\troisAPlP}{\textipa{batim}}\newcommand{\troisAPlG}{trois\cacherGloses{.2}\cacherGloses{-2.Pl}\cacherGloses{-2.A}}\newcommand{\troisBSg}{\strutgb{0pt}\grapho{p4q}}\newcommand{\troisBSgP}{\textipa{batov}}\newcommand{\troisBSgG}{trois\cacherGloses{.2}\cacherGloses{-2.Sg}\cacherGloses{-2.B}}\newcommand{\troisBDu}{\strutgb{0pt}\grapho{ptq}}\newcommand{\troisBDuP}{\textipa{batav}}\newcommand{\troisBDuG}{trois\cacherGloses{.2}\cacherGloses{-Du}\cacherGloses{-2.B}}\newcommand{\troisBPl}{\strutgb{0pt}\grapho{p3q}}\newcommand{\troisBPlP}{\textipa{bativ}}\newcommand{\troisBPlG}{trois\cacherGloses{.2}\cacherGloses{-2.Pl}\cacherGloses{-2.B}}\newcommand{\troisCSg}{\strutgb{0pt}\grapho{p4t}}\newcommand{\troisCSgP}{\textipa{batot}}\newcommand{\troisCSgG}{trois\cacherGloses{.2}\cacherGloses{-2.Sg}\cacherGloses{-2.C}}\newcommand{\troisCDu}{\strutgb{0pt}\grapho{ptt}}\newcommand{\troisCDuP}{\textipa{batat}}\newcommand{\troisCDuG}{trois\cacherGloses{.2}\cacherGloses{-Du}\cacherGloses{-2.C}}\newcommand{\troisCPl}{\strutgb{0pt}\grapho{p3t}}\newcommand{\troisCPlP}{\textipa{batit}}\newcommand{\troisCPlG}{trois\cacherGloses{.2}\cacherGloses{-2.Pl}\cacherGloses{-2.C}}\newcommand{\troisDSg}{\strutgb{0pt}\grapho{p4k}}\newcommand{\troisDSgP}{\textipa{batog}}\newcommand{\troisDSgG}{trois\cacherGloses{.2}\cacherGloses{-2.Sg}\cacherGloses{-2.D}}\newcommand{\troisDDu}{\strutgb{0pt}\grapho{ptk}}\newcommand{\troisDDuP}{\textipa{batag}}\newcommand{\troisDDuG}{trois\cacherGloses{.2}\cacherGloses{-Du}\cacherGloses{-2.D}}\newcommand{\troisDPl}{\strutgb{0pt}\grapho{p3k}}\newcommand{\troisDPlP}{\textipa{batig}}\newcommand{\troisDPlG}{trois\cacherGloses{.2}\cacherGloses{-2.Pl}\cacherGloses{-2.D}}\newcommand{\jauneASg}{\strutgb{0pt}\grapho{vEm}}\newcommand{\jauneASgP}{\textipa{kunom}}\newcommand{\jauneASgG}{jaune\cacherGloses{.2}\cacherGloses{-2.Sg}\cacherGloses{-2.A}}\newcommand{\jauneADu}{\strutgb{0pt}\grapho{vnm}}\newcommand{\jauneADuP}{\textipa{kunam}}\newcommand{\jauneADuG}{jaune\cacherGloses{.2}\cacherGloses{-Du}\cacherGloses{-2.A}}\newcommand{\jauneAPl}{\strutgb{0pt}\grapho{vCm}}\newcommand{\jauneAPlP}{\textipa{kunim}}\newcommand{\jauneAPlG}{jaune\cacherGloses{.2}\cacherGloses{-2.Pl}\cacherGloses{-2.A}}\newcommand{\jauneBSg}{\strutgb{0pt}\grapho{vEq}}\newcommand{\jauneBSgP}{\textipa{kunov}}\newcommand{\jauneBSgG}{jaune\cacherGloses{.2}\cacherGloses{-2.Sg}\cacherGloses{-2.B}}\newcommand{\jauneBDu}{\strutgb{0pt}\grapho{vnq}}\newcommand{\jauneBDuP}{\textipa{kunav}}\newcommand{\jauneBDuG}{jaune\cacherGloses{.2}\cacherGloses{-Du}\cacherGloses{-2.B}}\newcommand{\jauneBPl}{\strutgb{0pt}\grapho{vCq}}\newcommand{\jauneBPlP}{\textipa{kuniv}}\newcommand{\jauneBPlG}{jaune\cacherGloses{.2}\cacherGloses{-2.Pl}\cacherGloses{-2.B}}\newcommand{\jauneCSg}{\strutgb{0pt}\grapho{vEt}}\newcommand{\jauneCSgP}{\textipa{kunot}}\newcommand{\jauneCSgG}{jaune\cacherGloses{.2}\cacherGloses{-2.Sg}\cacherGloses{-2.C}}\newcommand{\jauneCDu}{\strutgb{0pt}\grapho{vnt}}\newcommand{\jauneCDuP}{\textipa{kunat}}\newcommand{\jauneCDuG}{jaune\cacherGloses{.2}\cacherGloses{-Du}\cacherGloses{-2.C}}\newcommand{\jauneCPl}{\strutgb{0pt}\grapho{vCt}}\newcommand{\jauneCPlP}{\textipa{kunit}}\newcommand{\jauneCPlG}{jaune\cacherGloses{.2}\cacherGloses{-2.Pl}\cacherGloses{-2.C}}\newcommand{\jauneDSg}{\strutgb{0pt}\grapho{vEk}}\newcommand{\jauneDSgP}{\textipa{kunog}}\newcommand{\jauneDSgG}{jaune\cacherGloses{.2}\cacherGloses{-2.Sg}\cacherGloses{-2.D}}\newcommand{\jauneDDu}{\strutgb{0pt}\grapho{vnk}}\newcommand{\jauneDDuP}{\textipa{kunag}}\newcommand{\jauneDDuG}{jaune\cacherGloses{.2}\cacherGloses{-Du}\cacherGloses{-2.D}}\newcommand{\jauneDPl}{\strutgb{0pt}\grapho{vCk}}\newcommand{\jauneDPlP}{\textipa{kunig}}\newcommand{\jauneDPlG}{jaune\cacherGloses{.2}\cacherGloses{-2.Pl}\cacherGloses{-2.D}}\newcommand{\rougeASg}{\strutgb{0pt}\grapho{cUm}}\newcommand{\rougeASgP}{\textipa{gilom}}\newcommand{\rougeASgG}{rouge\cacherGloses{.2}\cacherGloses{-2.Sg}\cacherGloses{-2.A}}\newcommand{\rougeADu}{\strutgb{0pt}\grapho{crm}}\newcommand{\rougeADuP}{\textipa{gilam}}\newcommand{\rougeADuG}{rouge\cacherGloses{.2}\cacherGloses{-Du}\cacherGloses{-2.A}}\newcommand{\rougeAPl}{\strutgb{0pt}\grapho{cOm}}\newcommand{\rougeAPlP}{\textipa{gilim}}\newcommand{\rougeAPlG}{rouge\cacherGloses{.2}\cacherGloses{-2.Pl}\cacherGloses{-2.A}}\newcommand{\rougeBSg}{\strutgb{0pt}\grapho{cUq}}\newcommand{\rougeBSgP}{\textipa{gilov}}\newcommand{\rougeBSgG}{rouge\cacherGloses{.2}\cacherGloses{-2.Sg}\cacherGloses{-2.B}}\newcommand{\rougeBDu}{\strutgb{0pt}\grapho{crq}}\newcommand{\rougeBDuP}{\textipa{gilav}}\newcommand{\rougeBDuG}{rouge\cacherGloses{.2}\cacherGloses{-Du}\cacherGloses{-2.B}}\newcommand{\rougeBPl}{\strutgb{0pt}\grapho{cOq}}\newcommand{\rougeBPlP}{\textipa{giliv}}\newcommand{\rougeBPlG}{rouge\cacherGloses{.2}\cacherGloses{-2.Pl}\cacherGloses{-2.B}}\newcommand{\rougeCSg}{\strutgb{0pt}\grapho{cUt}}\newcommand{\rougeCSgP}{\textipa{gilot}}\newcommand{\rougeCSgG}{rouge\cacherGloses{.2}\cacherGloses{-2.Sg}\cacherGloses{-2.C}}\newcommand{\rougeCDu}{\strutgb{0pt}\grapho{crt}}\newcommand{\rougeCDuP}{\textipa{gilat}}\newcommand{\rougeCDuG}{rouge\cacherGloses{.2}\cacherGloses{-Du}\cacherGloses{-2.C}}\newcommand{\rougeCPl}{\strutgb{0pt}\grapho{cOt}}\newcommand{\rougeCPlP}{\textipa{gilit}}\newcommand{\rougeCPlG}{rouge\cacherGloses{.2}\cacherGloses{-2.Pl}\cacherGloses{-2.C}}\newcommand{\rougeDSg}{\strutgb{0pt}\grapho{cUk}}\newcommand{\rougeDSgP}{\textipa{gilog}}\newcommand{\rougeDSgG}{rouge\cacherGloses{.2}\cacherGloses{-2.Sg}\cacherGloses{-2.D}}\newcommand{\rougeDDu}{\strutgb{0pt}\grapho{crk}}\newcommand{\rougeDDuP}{\textipa{gilag}}\newcommand{\rougeDDuG}{rouge\cacherGloses{.2}\cacherGloses{-Du}\cacherGloses{-2.D}}\newcommand{\rougeDPl}{\strutgb{0pt}\grapho{cOk}}\newcommand{\rougeDPlP}{\textipa{gilig}}\newcommand{\rougeDPlG}{rouge\cacherGloses{.2}\cacherGloses{-2.Pl}\cacherGloses{-2.D}}\newcommand{\grosASg}{\strutgb{0pt}\grapho{rHm}}\newcommand{\grosASgP}{\textipa{lapom}}\newcommand{\grosASgG}{gros\cacherGloses{.2}\cacherGloses{-2.Sg}\cacherGloses{-2.A}}\newcommand{\grosADu}{\strutgb{0pt}\grapho{rpm}}\newcommand{\grosADuP}{\textipa{lapam}}\newcommand{\grosADuG}{gros\cacherGloses{.2}\cacherGloses{-Du}\cacherGloses{-2.A}}\newcommand{\grosAPl}{\strutgb{0pt}\grapho{rGm}}\newcommand{\grosAPlP}{\textipa{lapim}}\newcommand{\grosAPlG}{gros\cacherGloses{.2}\cacherGloses{-2.Pl}\cacherGloses{-2.A}}\newcommand{\grosBSg}{\strutgb{0pt}\grapho{rHq}}\newcommand{\grosBSgP}{\textipa{lapov}}\newcommand{\grosBSgG}{gros\cacherGloses{.2}\cacherGloses{-2.Sg}\cacherGloses{-2.B}}\newcommand{\grosBDu}{\strutgb{0pt}\grapho{rpq}}\newcommand{\grosBDuP}{\textipa{lapav}}\newcommand{\grosBDuG}{gros\cacherGloses{.2}\cacherGloses{-Du}\cacherGloses{-2.B}}\newcommand{\grosBPl}{\strutgb{0pt}\grapho{rGq}}\newcommand{\grosBPlP}{\textipa{lapiv}}\newcommand{\grosBPlG}{gros\cacherGloses{.2}\cacherGloses{-2.Pl}\cacherGloses{-2.B}}\newcommand{\grosCSg}{\strutgb{0pt}\grapho{rHt}}\newcommand{\grosCSgP}{\textipa{lapot}}\newcommand{\grosCSgG}{gros\cacherGloses{.2}\cacherGloses{-2.Sg}\cacherGloses{-2.C}}\newcommand{\grosCDu}{\strutgb{0pt}\grapho{rpt}}\newcommand{\grosCDuP}{\textipa{lapat}}\newcommand{\grosCDuG}{gros\cacherGloses{.2}\cacherGloses{-Du}\cacherGloses{-2.C}}\newcommand{\grosCPl}{\strutgb{0pt}\grapho{rGt}}\newcommand{\grosCPlP}{\textipa{lapit}}\newcommand{\grosCPlG}{gros\cacherGloses{.2}\cacherGloses{-2.Pl}\cacherGloses{-2.C}}\newcommand{\grosDSg}{\strutgb{0pt}\grapho{rHk}}\newcommand{\grosDSgP}{\textipa{lapog}}\newcommand{\grosDSgG}{gros\cacherGloses{.2}\cacherGloses{-2.Sg}\cacherGloses{-2.D}}\newcommand{\grosDDu}{\strutgb{0pt}\grapho{rpk}}\newcommand{\grosDDuP}{\textipa{lapag}}\newcommand{\grosDDuG}{gros\cacherGloses{.2}\cacherGloses{-Du}\cacherGloses{-2.D}}\newcommand{\grosDPl}{\strutgb{0pt}\grapho{rGk}}\newcommand{\grosDPlP}{\textipa{lapig}}\newcommand{\grosDPlG}{gros\cacherGloses{.2}\cacherGloses{-2.Pl}\cacherGloses{-2.D}}\newcommand{\maigreASg}{\strutgb{0pt}\grapho{FUm}}\newcommand{\maigreASgP}{\textipa{Nulom}}\newcommand{\maigreASgG}{maigre\cacherGloses{.2}\cacherGloses{-2.Sg}\cacherGloses{-2.A}}\newcommand{\maigreADu}{\strutgb{0pt}\grapho{Frm}}\newcommand{\maigreADuP}{\textipa{Nulam}}\newcommand{\maigreADuG}{maigre\cacherGloses{.2}\cacherGloses{-Du}\cacherGloses{-2.A}}\newcommand{\maigreAPl}{\strutgb{0pt}\grapho{FOm}}\newcommand{\maigreAPlP}{\textipa{Nulim}}\newcommand{\maigreAPlG}{maigre\cacherGloses{.2}\cacherGloses{-2.Pl}\cacherGloses{-2.A}}\newcommand{\maigreBSg}{\strutgb{0pt}\grapho{FUq}}\newcommand{\maigreBSgP}{\textipa{Nulov}}\newcommand{\maigreBSgG}{maigre\cacherGloses{.2}\cacherGloses{-2.Sg}\cacherGloses{-2.B}}\newcommand{\maigreBDu}{\strutgb{0pt}\grapho{Frq}}\newcommand{\maigreBDuP}{\textipa{Nulav}}\newcommand{\maigreBDuG}{maigre\cacherGloses{.2}\cacherGloses{-Du}\cacherGloses{-2.B}}\newcommand{\maigreBPl}{\strutgb{0pt}\grapho{FOq}}\newcommand{\maigreBPlP}{\textipa{Nuliv}}\newcommand{\maigreBPlG}{maigre\cacherGloses{.2}\cacherGloses{-2.Pl}\cacherGloses{-2.B}}\newcommand{\maigreCSg}{\strutgb{0pt}\grapho{FUt}}\newcommand{\maigreCSgP}{\textipa{Nulot}}\newcommand{\maigreCSgG}{maigre\cacherGloses{.2}\cacherGloses{-2.Sg}\cacherGloses{-2.C}}\newcommand{\maigreCDu}{\strutgb{0pt}\grapho{Frt}}\newcommand{\maigreCDuP}{\textipa{Nulat}}\newcommand{\maigreCDuG}{maigre\cacherGloses{.2}\cacherGloses{-Du}\cacherGloses{-2.C}}\newcommand{\maigreCPl}{\strutgb{0pt}\grapho{FOt}}\newcommand{\maigreCPlP}{\textipa{Nulit}}\newcommand{\maigreCPlG}{maigre\cacherGloses{.2}\cacherGloses{-2.Pl}\cacherGloses{-2.C}}\newcommand{\maigreDSg}{\strutgb{0pt}\grapho{FUk}}\newcommand{\maigreDSgP}{\textipa{Nulog}}\newcommand{\maigreDSgG}{maigre\cacherGloses{.2}\cacherGloses{-2.Sg}\cacherGloses{-2.D}}\newcommand{\maigreDDu}{\strutgb{0pt}\grapho{Frk}}\newcommand{\maigreDDuP}{\textipa{Nulag}}\newcommand{\maigreDDuG}{maigre\cacherGloses{.2}\cacherGloses{-Du}\cacherGloses{-2.D}}\newcommand{\maigreDPl}{\strutgb{0pt}\grapho{FOk}}\newcommand{\maigreDPlP}{\textipa{Nulig}}\newcommand{\maigreDPlG}{maigre\cacherGloses{.2}\cacherGloses{-2.Pl}\cacherGloses{-2.D}}\newcommand{\infirmiereASgErg}{\strutgb{0pt}\grapho{RmAgK}}\newcommand{\infirmiereASgErgP}{\textipa{remamodoke}}\newcommand{\infirmiereASgErgG}{\cacherGloses{Erg-}infirmière\cacherGloses{.A}\cacherGloses{xSg}}\newcommand{\infirmiereASgAbs}{\strutgb{0pt}\grapho{HmAgK}}\newcommand{\infirmiereASgAbsP}{\textipa{bomamodoke}}\newcommand{\infirmiereASgAbsG}{\cacherGloses{Abs-}infirmière\cacherGloses{.A}\cacherGloses{xSg}}\newcommand{\infirmiereASgObl}{\strutgb{0pt}\grapho{hmAgK}}\newcommand{\infirmiereASgOblP}{\textipa{komamodoke}}\newcommand{\infirmiereASgOblG}{\cacherGloses{Obl-}infirmière\cacherGloses{.A}\cacherGloses{xSg}}\newcommand{\infirmiereASgDat}{\strutgb{0pt}\grapho{OmAgK}}\newcommand{\infirmiereASgDatP}{\textipa{limamodoke}}\newcommand{\infirmiereASgDatG}{\cacherGloses{Dat-}infirmière\cacherGloses{.A}\cacherGloses{xSg}}\newcommand{\infirmiereADuErg}{\strutgb{0pt}\grapho{RAHft}}\newcommand{\infirmiereADuErgP}{\textipa{remopodit}}\newcommand{\infirmiereADuErgG}{\cacherGloses{Erg-}infirmière\cacherGloses{.A}\cacherGloses{xA.Du}}\newcommand{\infirmiereADuAbs}{\strutgb{0pt}\grapho{HAHft}}\newcommand{\infirmiereADuAbsP}{\textipa{bomopodit}}\newcommand{\infirmiereADuAbsG}{\cacherGloses{Abs-}infirmière\cacherGloses{.A}\cacherGloses{xA.Du}}\newcommand{\infirmiereADuObl}{\strutgb{0pt}\grapho{hAHft}}\newcommand{\infirmiereADuOblP}{\textipa{komopodit}}\newcommand{\infirmiereADuOblG}{\cacherGloses{Obl-}infirmière\cacherGloses{.A}\cacherGloses{xA.Du}}\newcommand{\infirmiereADuDat}{\strutgb{0pt}\grapho{OAHft}}\newcommand{\infirmiereADuDatP}{\textipa{limopodit}}\newcommand{\infirmiereADuDatG}{\cacherGloses{Dat-}infirmière\cacherGloses{.A}\cacherGloses{xA.Du}}\newcommand{\infirmiereAPlErg}{\strutgb{0pt}\grapho{RAgIc}}\newcommand{\infirmiereAPlErgP}{\textipa{remodopuki}}\newcommand{\infirmiereAPlErgG}{\cacherGloses{Erg-}infirmière\cacherGloses{.A}\cacherGloses{xA.Pl}}\newcommand{\infirmiereAPlAbs}{\strutgb{0pt}\grapho{HAgIc}}\newcommand{\infirmiereAPlAbsP}{\textipa{bomodopuki}}\newcommand{\infirmiereAPlAbsG}{\cacherGloses{Abs-}infirmière\cacherGloses{.A}\cacherGloses{xA.Pl}}\newcommand{\infirmiereAPlObl}{\strutgb{0pt}\grapho{hAgIc}}\newcommand{\infirmiereAPlOblP}{\textipa{komodopuki}}\newcommand{\infirmiereAPlOblG}{\cacherGloses{Obl-}infirmière\cacherGloses{.A}\cacherGloses{xA.Pl}}\newcommand{\infirmiereAPlDat}{\strutgb{0pt}\grapho{OAgIc}}\newcommand{\infirmiereAPlDatP}{\textipa{limodopuki}}\newcommand{\infirmiereAPlDatG}{\cacherGloses{Dat-}infirmière\cacherGloses{.A}\cacherGloses{xA.Pl}}\newcommand{\KatishaASgErg}{\strutgb{0pt}\grapho{Rk5tS}}\newcommand{\KatishaASgErgP}{\textipa{rekatutaSe}}\newcommand{\KatishaASgErgG}{\cacherGloses{Erg-}Katisha\cacherGloses{.A}\cacherGloses{xSg}}\newcommand{\KatishaASgAbs}{\strutgb{0pt}\grapho{Hk5tS}}\newcommand{\KatishaASgAbsP}{\textipa{bokatutaSe}}\newcommand{\KatishaASgAbsG}{\cacherGloses{Abs-}Katisha\cacherGloses{.A}\cacherGloses{xSg}}\newcommand{\KatishaASgObl}{\strutgb{0pt}\grapho{hk5tS}}\newcommand{\KatishaASgOblP}{\textipa{kokatutaSe}}\newcommand{\KatishaASgOblG}{\cacherGloses{Obl-}Katisha\cacherGloses{.A}\cacherGloses{xSg}}\newcommand{\KatishaASgDat}{\strutgb{0pt}\grapho{Ok5tS}}\newcommand{\KatishaASgDatP}{\textipa{likatutaSe}}\newcommand{\KatishaASgDatG}{\cacherGloses{Dat-}Katisha\cacherGloses{.A}\cacherGloses{xSg}}\newcommand{\KatishaADuErg}{\strutgb{0pt}\grapho{Rvp3s}}\newcommand{\KatishaADuErgP}{\textipa{rekupatis}}\newcommand{\KatishaADuErgG}{\cacherGloses{Erg-}Katisha\cacherGloses{.A}\cacherGloses{xA.Du}}\newcommand{\KatishaADuAbs}{\strutgb{0pt}\grapho{Hvp3s}}\newcommand{\KatishaADuAbsP}{\textipa{bokupatis}}\newcommand{\KatishaADuAbsG}{\cacherGloses{Abs-}Katisha\cacherGloses{.A}\cacherGloses{xA.Du}}\newcommand{\KatishaADuObl}{\strutgb{0pt}\grapho{hvp3s}}\newcommand{\KatishaADuOblP}{\textipa{kokupatis}}\newcommand{\KatishaADuOblG}{\cacherGloses{Obl-}Katisha\cacherGloses{.A}\cacherGloses{xA.Du}}\newcommand{\KatishaADuDat}{\strutgb{0pt}\grapho{Ovp3s}}\newcommand{\KatishaADuDatP}{\textipa{likupatis}}\newcommand{\KatishaADuDatG}{\cacherGloses{Dat-}Katisha\cacherGloses{.A}\cacherGloses{xA.Du}}\newcommand{\KatishaAPlErg}{\strutgb{0pt}\grapho{R55quY}}\newcommand{\KatishaAPlErgP}{\textipa{retutufuSi}}\newcommand{\KatishaAPlErgG}{\cacherGloses{Erg-}Katisha\cacherGloses{.A}\cacherGloses{xA.Pl}}\newcommand{\KatishaAPlAbs}{\strutgb{0pt}\grapho{H55quY}}\newcommand{\KatishaAPlAbsP}{\textipa{botutufuSi}}\newcommand{\KatishaAPlAbsG}{\cacherGloses{Abs-}Katisha\cacherGloses{.A}\cacherGloses{xA.Pl}}\newcommand{\KatishaAPlObl}{\strutgb{0pt}\grapho{h55quY}}\newcommand{\KatishaAPlOblP}{\textipa{kotutufuSi}}\newcommand{\KatishaAPlOblG}{\cacherGloses{Obl-}Katisha\cacherGloses{.A}\cacherGloses{xA.Pl}}\newcommand{\KatishaAPlDat}{\strutgb{0pt}\grapho{O55quY}}\newcommand{\KatishaAPlDatP}{\textipa{litutufuSi}}\newcommand{\KatishaAPlDatG}{\cacherGloses{Dat-}Katisha\cacherGloses{.A}\cacherGloses{xA.Pl}}\newcommand{\plaineASgErg}{\strutgb{0pt}\grapho{RpHPZ}}\newcommand{\plaineASgErgP}{\textipa{repapobeze}}\newcommand{\plaineASgErgG}{\cacherGloses{Erg-}plaine\cacherGloses{.A}\cacherGloses{xSg}}\newcommand{\plaineASgAbs}{\strutgb{0pt}\grapho{HpHPZ}}\newcommand{\plaineASgAbsP}{\textipa{bopapobeze}}\newcommand{\plaineASgAbsG}{\cacherGloses{Abs-}plaine\cacherGloses{.A}\cacherGloses{xSg}}\newcommand{\plaineASgObl}{\strutgb{0pt}\grapho{hpHPZ}}\newcommand{\plaineASgOblP}{\textipa{kopapobeze}}\newcommand{\plaineASgOblG}{\cacherGloses{Obl-}plaine\cacherGloses{.A}\cacherGloses{xSg}}\newcommand{\plaineASgDat}{\strutgb{0pt}\grapho{OpHPZ}}\newcommand{\plaineASgDatP}{\textipa{lipapobeze}}\newcommand{\plaineASgDatG}{\cacherGloses{Dat-}plaine\cacherGloses{.A}\cacherGloses{xSg}}\newcommand{\plaineADuErg}{\strutgb{0pt}\grapho{RHPGq}}\newcommand{\plaineADuErgP}{\textipa{repopebiv}}\newcommand{\plaineADuErgG}{\cacherGloses{Erg-}plaine\cacherGloses{.A}\cacherGloses{xA.Du}}\newcommand{\plaineADuAbs}{\strutgb{0pt}\grapho{HHPGq}}\newcommand{\plaineADuAbsP}{\textipa{bopopebiv}}\newcommand{\plaineADuAbsG}{\cacherGloses{Abs-}plaine\cacherGloses{.A}\cacherGloses{xA.Du}}\newcommand{\plaineADuObl}{\strutgb{0pt}\grapho{hHPGq}}\newcommand{\plaineADuOblP}{\textipa{kopopebiv}}\newcommand{\plaineADuOblG}{\cacherGloses{Obl-}plaine\cacherGloses{.A}\cacherGloses{xA.Du}}\newcommand{\plaineADuDat}{\strutgb{0pt}\grapho{OHPGq}}\newcommand{\plaineADuDatP}{\textipa{lipopebiv}}\newcommand{\plaineADuDatG}{\cacherGloses{Dat-}plaine\cacherGloses{.A}\cacherGloses{xA.Du}}\newcommand{\plaineAPlErg}{\strutgb{0pt}\grapho{RHHquzi}}\newcommand{\plaineAPlErgP}{\textipa{repobofuzi}}\newcommand{\plaineAPlErgG}{\cacherGloses{Erg-}plaine\cacherGloses{.A}\cacherGloses{xA.Pl}}\newcommand{\plaineAPlAbs}{\strutgb{0pt}\grapho{HHHquzi}}\newcommand{\plaineAPlAbsP}{\textipa{bopobofuzi}}\newcommand{\plaineAPlAbsG}{\cacherGloses{Abs-}plaine\cacherGloses{.A}\cacherGloses{xA.Pl}}\newcommand{\plaineAPlObl}{\strutgb{0pt}\grapho{hHHquzi}}\newcommand{\plaineAPlOblP}{\textipa{kopobofuzi}}\newcommand{\plaineAPlOblG}{\cacherGloses{Obl-}plaine\cacherGloses{.A}\cacherGloses{xA.Pl}}\newcommand{\plaineAPlDat}{\strutgb{0pt}\grapho{OHHquzi}}\newcommand{\plaineAPlDatP}{\textipa{lipobofuzi}}\newcommand{\plaineAPlDatG}{\cacherGloses{Dat-}plaine\cacherGloses{.A}\cacherGloses{xA.Pl}}\newcommand{\viandeASgErg}{\strutgb{0pt}\grapho{RpIdN}}\newcommand{\viandeASgErgP}{\textipa{repapudaNe}}\newcommand{\viandeASgErgG}{\cacherGloses{Erg-}viande\cacherGloses{.A}\cacherGloses{xSg}}\newcommand{\viandeASgAbs}{\strutgb{0pt}\grapho{HpIdN}}\newcommand{\viandeASgAbsP}{\textipa{bopapudaNe}}\newcommand{\viandeASgAbsG}{\cacherGloses{Abs-}viande\cacherGloses{.A}\cacherGloses{xSg}}\newcommand{\viandeASgObl}{\strutgb{0pt}\grapho{hpIdN}}\newcommand{\viandeASgOblP}{\textipa{kopapudaNe}}\newcommand{\viandeASgOblG}{\cacherGloses{Obl-}viande\cacherGloses{.A}\cacherGloses{xSg}}\newcommand{\viandeASgDat}{\strutgb{0pt}\grapho{OpIdN}}\newcommand{\viandeASgDatP}{\textipa{lipapudaNe}}\newcommand{\viandeASgDatG}{\cacherGloses{Dat-}viande\cacherGloses{.A}\cacherGloses{xSg}}\newcommand{\viandeADuErg}{\strutgb{0pt}\grapho{RIpfn}}\newcommand{\viandeADuErgP}{\textipa{repupadin}}\newcommand{\viandeADuErgG}{\cacherGloses{Erg-}viande\cacherGloses{.A}\cacherGloses{xA.Du}}\newcommand{\viandeADuAbs}{\strutgb{0pt}\grapho{HIpfn}}\newcommand{\viandeADuAbsP}{\textipa{bopupadin}}\newcommand{\viandeADuAbsG}{\cacherGloses{Abs-}viande\cacherGloses{.A}\cacherGloses{xA.Du}}\newcommand{\viandeADuObl}{\strutgb{0pt}\grapho{hIpfn}}\newcommand{\viandeADuOblP}{\textipa{kopupadin}}\newcommand{\viandeADuOblG}{\cacherGloses{Obl-}viande\cacherGloses{.A}\cacherGloses{xA.Du}}\newcommand{\viandeADuDat}{\strutgb{0pt}\grapho{OIpfn}}\newcommand{\viandeADuDatP}{\textipa{lipupadin}}\newcommand{\viandeADuDatG}{\cacherGloses{Dat-}viande\cacherGloses{.A}\cacherGloses{xA.Du}}\newcommand{\viandeAPlErg}{\strutgb{0pt}\grapho{RIxBC}}\newcommand{\viandeAPlErgP}{\textipa{repudumuNi}}\newcommand{\viandeAPlErgG}{\cacherGloses{Erg-}viande\cacherGloses{.A}\cacherGloses{xA.Pl}}\newcommand{\viandeAPlAbs}{\strutgb{0pt}\grapho{HIxBC}}\newcommand{\viandeAPlAbsP}{\textipa{bopudumuNi}}\newcommand{\viandeAPlAbsG}{\cacherGloses{Abs-}viande\cacherGloses{.A}\cacherGloses{xA.Pl}}\newcommand{\viandeAPlObl}{\strutgb{0pt}\grapho{hIxBC}}\newcommand{\viandeAPlOblP}{\textipa{kopudumuNi}}\newcommand{\viandeAPlOblG}{\cacherGloses{Obl-}viande\cacherGloses{.A}\cacherGloses{xA.Pl}}\newcommand{\viandeAPlDat}{\strutgb{0pt}\grapho{OIxBC}}\newcommand{\viandeAPlDatP}{\textipa{lipudumuNi}}\newcommand{\viandeAPlDatG}{\cacherGloses{Dat-}viande\cacherGloses{.A}\cacherGloses{xA.Pl}}\newcommand{\balaiASgErg}{\strutgb{0pt}\grapho{RkxtK}}\newcommand{\balaiASgErgP}{\textipa{regadutake}}\newcommand{\balaiASgErgG}{\cacherGloses{Erg-}balai\cacherGloses{.A}\cacherGloses{xSg}}\newcommand{\balaiASgAbs}{\strutgb{0pt}\grapho{HkxtK}}\newcommand{\balaiASgAbsP}{\textipa{bogadutake}}\newcommand{\balaiASgAbsG}{\cacherGloses{Abs-}balai\cacherGloses{.A}\cacherGloses{xSg}}\newcommand{\balaiASgObl}{\strutgb{0pt}\grapho{hkxtK}}\newcommand{\balaiASgOblP}{\textipa{kogadutake}}\newcommand{\balaiASgOblG}{\cacherGloses{Obl-}balai\cacherGloses{.A}\cacherGloses{xSg}}\newcommand{\balaiASgDat}{\strutgb{0pt}\grapho{OkxtK}}\newcommand{\balaiASgDatP}{\textipa{ligadutake}}\newcommand{\balaiASgDatG}{\cacherGloses{Dat-}balai\cacherGloses{.A}\cacherGloses{xSg}}\newcommand{\balaiADuErg}{\strutgb{0pt}\grapho{Rvp3t}}\newcommand{\balaiADuErgP}{\textipa{regupatit}}\newcommand{\balaiADuErgG}{\cacherGloses{Erg-}balai\cacherGloses{.A}\cacherGloses{xA.Du}}\newcommand{\balaiADuAbs}{\strutgb{0pt}\grapho{Hvp3t}}\newcommand{\balaiADuAbsP}{\textipa{bogupatit}}\newcommand{\balaiADuAbsG}{\cacherGloses{Abs-}balai\cacherGloses{.A}\cacherGloses{xA.Du}}\newcommand{\balaiADuObl}{\strutgb{0pt}\grapho{hvp3t}}\newcommand{\balaiADuOblP}{\textipa{kogupatit}}\newcommand{\balaiADuOblG}{\cacherGloses{Obl-}balai\cacherGloses{.A}\cacherGloses{xA.Du}}\newcommand{\balaiADuDat}{\strutgb{0pt}\grapho{Ovp3t}}\newcommand{\balaiADuDatP}{\textipa{ligupatit}}\newcommand{\balaiADuDatG}{\cacherGloses{Dat-}balai\cacherGloses{.A}\cacherGloses{xA.Du}}\newcommand{\balaiAPlErg}{\strutgb{0pt}\grapho{Rx5Ic}}\newcommand{\balaiAPlErgP}{\textipa{redutupuki}}\newcommand{\balaiAPlErgG}{\cacherGloses{Erg-}balai\cacherGloses{.A}\cacherGloses{xA.Pl}}\newcommand{\balaiAPlAbs}{\strutgb{0pt}\grapho{Hx5Ic}}\newcommand{\balaiAPlAbsP}{\textipa{bodutupuki}}\newcommand{\balaiAPlAbsG}{\cacherGloses{Abs-}balai\cacherGloses{.A}\cacherGloses{xA.Pl}}\newcommand{\balaiAPlObl}{\strutgb{0pt}\grapho{hx5Ic}}\newcommand{\balaiAPlOblP}{\textipa{kodutupuki}}\newcommand{\balaiAPlOblG}{\cacherGloses{Obl-}balai\cacherGloses{.A}\cacherGloses{xA.Pl}}\newcommand{\balaiAPlDat}{\strutgb{0pt}\grapho{Ox5Ic}}\newcommand{\balaiAPlDatP}{\textipa{lidutupuki}}\newcommand{\balaiAPlDatG}{\cacherGloses{Dat-}balai\cacherGloses{.A}\cacherGloses{xA.Pl}}\newcommand{\fruitASgErg}{\strutgb{0pt}\grapho{RtH1K}}\newcommand{\fruitASgErgP}{\textipa{retapoSoke}}\newcommand{\fruitASgErgG}{\cacherGloses{Erg-}fruit\cacherGloses{.A}\cacherGloses{xSg}}\newcommand{\fruitASgAbs}{\strutgb{0pt}\grapho{HtH1K}}\newcommand{\fruitASgAbsP}{\textipa{botapoSoke}}\newcommand{\fruitASgAbsG}{\cacherGloses{Abs-}fruit\cacherGloses{.A}\cacherGloses{xSg}}\newcommand{\fruitASgObl}{\strutgb{0pt}\grapho{htH1K}}\newcommand{\fruitASgOblP}{\textipa{kotapoSoke}}\newcommand{\fruitASgOblG}{\cacherGloses{Obl-}fruit\cacherGloses{.A}\cacherGloses{xSg}}\newcommand{\fruitASgDat}{\strutgb{0pt}\grapho{OtH1K}}\newcommand{\fruitASgDatP}{\textipa{litapoSoke}}\newcommand{\fruitASgDatG}{\cacherGloses{Dat-}fruit\cacherGloses{.A}\cacherGloses{xSg}}\newcommand{\fruitADuErg}{\strutgb{0pt}\grapho{R48Yt}}\newcommand{\fruitADuErgP}{\textipa{retofoSit}}\newcommand{\fruitADuErgG}{\cacherGloses{Erg-}fruit\cacherGloses{.A}\cacherGloses{xA.Du}}\newcommand{\fruitADuAbs}{\strutgb{0pt}\grapho{H48Yt}}\newcommand{\fruitADuAbsP}{\textipa{botofoSit}}\newcommand{\fruitADuAbsG}{\cacherGloses{Abs-}fruit\cacherGloses{.A}\cacherGloses{xA.Du}}\newcommand{\fruitADuObl}{\strutgb{0pt}\grapho{h48Yt}}\newcommand{\fruitADuOblP}{\textipa{kotofoSit}}\newcommand{\fruitADuOblG}{\cacherGloses{Obl-}fruit\cacherGloses{.A}\cacherGloses{xA.Du}}\newcommand{\fruitADuDat}{\strutgb{0pt}\grapho{O48Yt}}\newcommand{\fruitADuDatP}{\textipa{litofoSit}}\newcommand{\fruitADuDatG}{\cacherGloses{Dat-}fruit\cacherGloses{.A}\cacherGloses{xA.Du}}\newcommand{\fruitAPlErg}{\strutgb{0pt}\grapho{RH1Ic}}\newcommand{\fruitAPlErgP}{\textipa{repoSopuki}}\newcommand{\fruitAPlErgG}{\cacherGloses{Erg-}fruit\cacherGloses{.A}\cacherGloses{xA.Pl}}\newcommand{\fruitAPlAbs}{\strutgb{0pt}\grapho{HH1Ic}}\newcommand{\fruitAPlAbsP}{\textipa{bopoSopuki}}\newcommand{\fruitAPlAbsG}{\cacherGloses{Abs-}fruit\cacherGloses{.A}\cacherGloses{xA.Pl}}\newcommand{\fruitAPlObl}{\strutgb{0pt}\grapho{hH1Ic}}\newcommand{\fruitAPlOblP}{\textipa{kopoSopuki}}\newcommand{\fruitAPlOblG}{\cacherGloses{Obl-}fruit\cacherGloses{.A}\cacherGloses{xA.Pl}}\newcommand{\fruitAPlDat}{\strutgb{0pt}\grapho{OH1Ic}}\newcommand{\fruitAPlDatP}{\textipa{lipoSopuki}}\newcommand{\fruitAPlDatG}{\cacherGloses{Dat-}fruit\cacherGloses{.A}\cacherGloses{xA.Pl}}\newcommand{\filleCSgErg}{\strutgb{0pt}\grapho{RmmCT}}\newcommand{\filleCSgErgP}{\textipa{remamanite}}\newcommand{\filleCSgErgG}{\cacherGloses{Erg-}fille\cacherGloses{.C}\cacherGloses{xSg}}\newcommand{\filleCSgAbs}{\strutgb{0pt}\grapho{HmmCT}}\newcommand{\filleCSgAbsP}{\textipa{bomamanite}}\newcommand{\filleCSgAbsG}{\cacherGloses{Abs-}fille\cacherGloses{.C}\cacherGloses{xSg}}\newcommand{\filleCSgObl}{\strutgb{0pt}\grapho{hmmCT}}\newcommand{\filleCSgOblP}{\textipa{komamanite}}\newcommand{\filleCSgOblG}{\cacherGloses{Obl-}fille\cacherGloses{.C}\cacherGloses{xSg}}\newcommand{\filleCSgDat}{\strutgb{0pt}\grapho{OmmCT}}\newcommand{\filleCSgDatP}{\textipa{limamanite}}\newcommand{\filleCSgDatG}{\cacherGloses{Dat-}fille\cacherGloses{.C}\cacherGloses{xSg}}\newcommand{\filleCDuErg}{\strutgb{0pt}\grapho{RmyCp}}\newcommand{\filleCDuErgP}{\textipa{remaminipa}}\newcommand{\filleCDuErgG}{\cacherGloses{Erg-}fille\cacherGloses{.C}\cacherGloses{xC.Du}}\newcommand{\filleCDuAbs}{\strutgb{0pt}\grapho{HmyCp}}\newcommand{\filleCDuAbsP}{\textipa{bomaminipa}}\newcommand{\filleCDuAbsG}{\cacherGloses{Abs-}fille\cacherGloses{.C}\cacherGloses{xC.Du}}\newcommand{\filleCDuObl}{\strutgb{0pt}\grapho{hmyCp}}\newcommand{\filleCDuOblP}{\textipa{komaminipa}}\newcommand{\filleCDuOblG}{\cacherGloses{Obl-}fille\cacherGloses{.C}\cacherGloses{xC.Du}}\newcommand{\filleCDuDat}{\strutgb{0pt}\grapho{OmyCp}}\newcommand{\filleCDuDatP}{\textipa{limaminipa}}\newcommand{\filleCDuDatG}{\cacherGloses{Dat-}fille\cacherGloses{.C}\cacherGloses{xC.Du}}\newcommand{\filleCPlErg}{\strutgb{0pt}\grapho{RBnI5}}\newcommand{\filleCPlErgP}{\textipa{remunaputu}}\newcommand{\filleCPlErgG}{\cacherGloses{Erg-}fille\cacherGloses{.C}\cacherGloses{xC.Pl}}\newcommand{\filleCPlAbs}{\strutgb{0pt}\grapho{HBnI5}}\newcommand{\filleCPlAbsP}{\textipa{bomunaputu}}\newcommand{\filleCPlAbsG}{\cacherGloses{Abs-}fille\cacherGloses{.C}\cacherGloses{xC.Pl}}\newcommand{\filleCPlObl}{\strutgb{0pt}\grapho{hBnI5}}\newcommand{\filleCPlOblP}{\textipa{komunaputu}}\newcommand{\filleCPlOblG}{\cacherGloses{Obl-}fille\cacherGloses{.C}\cacherGloses{xC.Pl}}\newcommand{\filleCPlDat}{\strutgb{0pt}\grapho{OBnI5}}\newcommand{\filleCPlDatP}{\textipa{limunaputu}}\newcommand{\filleCPlDatG}{\cacherGloses{Dat-}fille\cacherGloses{.C}\cacherGloses{xC.Pl}}\newcommand{\coyoteCSgErg}{\strutgb{0pt}\grapho{RpH4S}}\newcommand{\coyoteCSgErgP}{\textipa{repapotoSe}}\newcommand{\coyoteCSgErgG}{\cacherGloses{Erg-}coyote\cacherGloses{.C}\cacherGloses{xSg}}\newcommand{\coyoteCSgAbs}{\strutgb{0pt}\grapho{HpH4S}}\newcommand{\coyoteCSgAbsP}{\textipa{bopapotoSe}}\newcommand{\coyoteCSgAbsG}{\cacherGloses{Abs-}coyote\cacherGloses{.C}\cacherGloses{xSg}}\newcommand{\coyoteCSgObl}{\strutgb{0pt}\grapho{hpH4S}}\newcommand{\coyoteCSgOblP}{\textipa{kopapotoSe}}\newcommand{\coyoteCSgOblG}{\cacherGloses{Obl-}coyote\cacherGloses{.C}\cacherGloses{xSg}}\newcommand{\coyoteCSgDat}{\strutgb{0pt}\grapho{OpH4S}}\newcommand{\coyoteCSgDatP}{\textipa{lipapotoSe}}\newcommand{\coyoteCSgDatG}{\cacherGloses{Dat-}coyote\cacherGloses{.C}\cacherGloses{xSg}}\newcommand{\coyoteCDuErg}{\strutgb{0pt}\grapho{RHH3s}}\newcommand{\coyoteCDuErgP}{\textipa{repopotisa}}\newcommand{\coyoteCDuErgG}{\cacherGloses{Erg-}coyote\cacherGloses{.C}\cacherGloses{xC.Du}}\newcommand{\coyoteCDuAbs}{\strutgb{0pt}\grapho{HHH3s}}\newcommand{\coyoteCDuAbsP}{\textipa{bopopotisa}}\newcommand{\coyoteCDuAbsG}{\cacherGloses{Abs-}coyote\cacherGloses{.C}\cacherGloses{xC.Du}}\newcommand{\coyoteCDuObl}{\strutgb{0pt}\grapho{hHH3s}}\newcommand{\coyoteCDuOblP}{\textipa{kopopotisa}}\newcommand{\coyoteCDuOblG}{\cacherGloses{Obl-}coyote\cacherGloses{.C}\cacherGloses{xC.Du}}\newcommand{\coyoteCDuDat}{\strutgb{0pt}\grapho{OHH3s}}\newcommand{\coyoteCDuDatP}{\textipa{lipopotisa}}\newcommand{\coyoteCDuDatG}{\cacherGloses{Dat-}coyote\cacherGloses{.C}\cacherGloses{xC.Du}}\newcommand{\coyoteCPlErg}{\strutgb{0pt}\grapho{RH4qu2}}\newcommand{\coyoteCPlErgP}{\textipa{repotofuSu}}\newcommand{\coyoteCPlErgG}{\cacherGloses{Erg-}coyote\cacherGloses{.C}\cacherGloses{xC.Pl}}\newcommand{\coyoteCPlAbs}{\strutgb{0pt}\grapho{HH4qu2}}\newcommand{\coyoteCPlAbsP}{\textipa{bopotofuSu}}\newcommand{\coyoteCPlAbsG}{\cacherGloses{Abs-}coyote\cacherGloses{.C}\cacherGloses{xC.Pl}}\newcommand{\coyoteCPlObl}{\strutgb{0pt}\grapho{hH4qu2}}\newcommand{\coyoteCPlOblP}{\textipa{kopotofuSu}}\newcommand{\coyoteCPlOblG}{\cacherGloses{Obl-}coyote\cacherGloses{.C}\cacherGloses{xC.Pl}}\newcommand{\coyoteCPlDat}{\strutgb{0pt}\grapho{OH4qu2}}\newcommand{\coyoteCPlDatP}{\textipa{lipotofuSu}}\newcommand{\coyoteCPlDatG}{\cacherGloses{Dat-}coyote\cacherGloses{.C}\cacherGloses{xC.Pl}}\newcommand{\oeufCSgErg}{\strutgb{0pt}\grapho{Rz9RM}}\newcommand{\oeufCSgErgP}{\textipa{reZazoreme}}\newcommand{\oeufCSgErgG}{\cacherGloses{Erg-}oeuf\cacherGloses{.C}\cacherGloses{xSg}}\newcommand{\oeufCSgAbs}{\strutgb{0pt}\grapho{Hz9RM}}\newcommand{\oeufCSgAbsP}{\textipa{boZazoreme}}\newcommand{\oeufCSgAbsG}{\cacherGloses{Abs-}oeuf\cacherGloses{.C}\cacherGloses{xSg}}\newcommand{\oeufCSgObl}{\strutgb{0pt}\grapho{hz9RM}}\newcommand{\oeufCSgOblP}{\textipa{koZazoreme}}\newcommand{\oeufCSgOblG}{\cacherGloses{Obl-}oeuf\cacherGloses{.C}\cacherGloses{xSg}}\newcommand{\oeufCSgDat}{\strutgb{0pt}\grapho{Oz9RM}}\newcommand{\oeufCSgDatP}{\textipa{liZazoreme}}\newcommand{\oeufCSgDatG}{\cacherGloses{Dat-}oeuf\cacherGloses{.C}\cacherGloses{xSg}}\newcommand{\oeufCDuErg}{\strutgb{0pt}\grapho{R9WOm}}\newcommand{\oeufCDuErgP}{\textipa{reZowerima}}\newcommand{\oeufCDuErgG}{\cacherGloses{Erg-}oeuf\cacherGloses{.C}\cacherGloses{xC.Du}}\newcommand{\oeufCDuAbs}{\strutgb{0pt}\grapho{H9WOm}}\newcommand{\oeufCDuAbsP}{\textipa{boZowerima}}\newcommand{\oeufCDuAbsG}{\cacherGloses{Abs-}oeuf\cacherGloses{.C}\cacherGloses{xC.Du}}\newcommand{\oeufCDuObl}{\strutgb{0pt}\grapho{h9WOm}}\newcommand{\oeufCDuOblP}{\textipa{koZowerima}}\newcommand{\oeufCDuOblG}{\cacherGloses{Obl-}oeuf\cacherGloses{.C}\cacherGloses{xC.Du}}\newcommand{\oeufCDuDat}{\strutgb{0pt}\grapho{O9WOm}}\newcommand{\oeufCDuDatP}{\textipa{liZowerima}}\newcommand{\oeufCDuDatG}{\cacherGloses{Dat-}oeuf\cacherGloses{.C}\cacherGloses{xC.Du}}\newcommand{\oeufCPlErg}{\strutgb{0pt}\grapho{R9UBB}}\newcommand{\oeufCPlErgP}{\textipa{rezoromumu}}\newcommand{\oeufCPlErgG}{\cacherGloses{Erg-}oeuf\cacherGloses{.C}\cacherGloses{xC.Pl}}\newcommand{\oeufCPlAbs}{\strutgb{0pt}\grapho{H9UBB}}\newcommand{\oeufCPlAbsP}{\textipa{bozoromumu}}\newcommand{\oeufCPlAbsG}{\cacherGloses{Abs-}oeuf\cacherGloses{.C}\cacherGloses{xC.Pl}}\newcommand{\oeufCPlObl}{\strutgb{0pt}\grapho{h9UBB}}\newcommand{\oeufCPlOblP}{\textipa{kozoromumu}}\newcommand{\oeufCPlOblG}{\cacherGloses{Obl-}oeuf\cacherGloses{.C}\cacherGloses{xC.Pl}}\newcommand{\oeufCPlDat}{\strutgb{0pt}\grapho{O9UBB}}\newcommand{\oeufCPlDatP}{\textipa{lizoromumu}}\newcommand{\oeufCPlDatG}{\cacherGloses{Dat-}oeuf\cacherGloses{.C}\cacherGloses{xC.Pl}}\newcommand{\villageCSgErg}{\strutgb{0pt}\grapho{Rk5sP}}\newcommand{\villageCSgErgP}{\textipa{rekatusabe}}\newcommand{\villageCSgErgG}{\cacherGloses{Erg-}village\cacherGloses{.C}\cacherGloses{xSg}}\newcommand{\villageCSgAbs}{\strutgb{0pt}\grapho{Hk5sP}}\newcommand{\villageCSgAbsP}{\textipa{bokatusabe}}\newcommand{\villageCSgAbsG}{\cacherGloses{Abs-}village\cacherGloses{.C}\cacherGloses{xSg}}\newcommand{\villageCSgObl}{\strutgb{0pt}\grapho{hk5sP}}\newcommand{\villageCSgOblP}{\textipa{kokatusabe}}\newcommand{\villageCSgOblG}{\cacherGloses{Obl-}village\cacherGloses{.C}\cacherGloses{xSg}}\newcommand{\villageCSgDat}{\strutgb{0pt}\grapho{Ok5sP}}\newcommand{\villageCSgDatP}{\textipa{likatusabe}}\newcommand{\villageCSgDatG}{\cacherGloses{Dat-}village\cacherGloses{.C}\cacherGloses{xSg}}\newcommand{\villageCDuErg}{\strutgb{0pt}\grapho{RvqYp}}\newcommand{\villageCDuErgP}{\textipa{rekufasipa}}\newcommand{\villageCDuErgG}{\cacherGloses{Erg-}village\cacherGloses{.C}\cacherGloses{xC.Du}}\newcommand{\villageCDuAbs}{\strutgb{0pt}\grapho{HvqYp}}\newcommand{\villageCDuAbsP}{\textipa{bokufasipa}}\newcommand{\villageCDuAbsG}{\cacherGloses{Abs-}village\cacherGloses{.C}\cacherGloses{xC.Du}}\newcommand{\villageCDuObl}{\strutgb{0pt}\grapho{hvqYp}}\newcommand{\villageCDuOblP}{\textipa{kokufasipa}}\newcommand{\villageCDuOblG}{\cacherGloses{Obl-}village\cacherGloses{.C}\cacherGloses{xC.Du}}\newcommand{\villageCDuDat}{\strutgb{0pt}\grapho{OvqYp}}\newcommand{\villageCDuDatP}{\textipa{likufasipa}}\newcommand{\villageCDuDatG}{\cacherGloses{Dat-}village\cacherGloses{.C}\cacherGloses{xC.Du}}\newcommand{\villageCPlErg}{\strutgb{0pt}\grapho{R52II}}\newcommand{\villageCPlErgP}{\textipa{retusupubu}}\newcommand{\villageCPlErgG}{\cacherGloses{Erg-}village\cacherGloses{.C}\cacherGloses{xC.Pl}}\newcommand{\villageCPlAbs}{\strutgb{0pt}\grapho{H52II}}\newcommand{\villageCPlAbsP}{\textipa{botusupubu}}\newcommand{\villageCPlAbsG}{\cacherGloses{Abs-}village\cacherGloses{.C}\cacherGloses{xC.Pl}}\newcommand{\villageCPlObl}{\strutgb{0pt}\grapho{h52II}}\newcommand{\villageCPlOblP}{\textipa{kotusupubu}}\newcommand{\villageCPlOblG}{\cacherGloses{Obl-}village\cacherGloses{.C}\cacherGloses{xC.Pl}}\newcommand{\villageCPlDat}{\strutgb{0pt}\grapho{O52II}}\newcommand{\villageCPlDatP}{\textipa{litusupubu}}\newcommand{\villageCPlDatG}{\cacherGloses{Dat-}village\cacherGloses{.C}\cacherGloses{xC.Pl}}\newcommand{\cafeCSgErg}{\strutgb{0pt}\grapho{Rq8RT}}\newcommand{\cafeCSgErgP}{\textipa{refafolete}}\newcommand{\cafeCSgErgG}{\cacherGloses{Erg-}café\cacherGloses{.C}\cacherGloses{xSg}}\newcommand{\cafeCSgAbs}{\strutgb{0pt}\grapho{Hq8RT}}\newcommand{\cafeCSgAbsP}{\textipa{bofafolete}}\newcommand{\cafeCSgAbsG}{\cacherGloses{Abs-}café\cacherGloses{.C}\cacherGloses{xSg}}\newcommand{\cafeCSgObl}{\strutgb{0pt}\grapho{hq8RT}}\newcommand{\cafeCSgOblP}{\textipa{kofafolete}}\newcommand{\cafeCSgOblG}{\cacherGloses{Obl-}café\cacherGloses{.C}\cacherGloses{xSg}}\newcommand{\cafeCSgDat}{\strutgb{0pt}\grapho{Oq8RT}}\newcommand{\cafeCSgDatP}{\textipa{lifafolete}}\newcommand{\cafeCSgDatG}{\cacherGloses{Dat-}café\cacherGloses{.C}\cacherGloses{xSg}}\newcommand{\cafeCDuErg}{\strutgb{0pt}\grapho{R8WOp}}\newcommand{\cafeCDuErgP}{\textipa{refowelipa}}\newcommand{\cafeCDuErgG}{\cacherGloses{Erg-}café\cacherGloses{.C}\cacherGloses{xC.Du}}\newcommand{\cafeCDuAbs}{\strutgb{0pt}\grapho{H8WOp}}\newcommand{\cafeCDuAbsP}{\textipa{bofowelipa}}\newcommand{\cafeCDuAbsG}{\cacherGloses{Abs-}café\cacherGloses{.C}\cacherGloses{xC.Du}}\newcommand{\cafeCDuObl}{\strutgb{0pt}\grapho{h8WOp}}\newcommand{\cafeCDuOblP}{\textipa{kofowelipa}}\newcommand{\cafeCDuOblG}{\cacherGloses{Obl-}café\cacherGloses{.C}\cacherGloses{xC.Du}}\newcommand{\cafeCDuDat}{\strutgb{0pt}\grapho{O8WOp}}\newcommand{\cafeCDuDatP}{\textipa{lifowelipa}}\newcommand{\cafeCDuDatG}{\cacherGloses{Dat-}café\cacherGloses{.C}\cacherGloses{xC.Du}}\newcommand{\cafeCPlErg}{\strutgb{0pt}\grapho{R8UI5}}\newcommand{\cafeCPlErgP}{\textipa{refoloputu}}\newcommand{\cafeCPlErgG}{\cacherGloses{Erg-}café\cacherGloses{.C}\cacherGloses{xC.Pl}}\newcommand{\cafeCPlAbs}{\strutgb{0pt}\grapho{H8UI5}}\newcommand{\cafeCPlAbsP}{\textipa{bofoloputu}}\newcommand{\cafeCPlAbsG}{\cacherGloses{Abs-}café\cacherGloses{.C}\cacherGloses{xC.Pl}}\newcommand{\cafeCPlObl}{\strutgb{0pt}\grapho{h8UI5}}\newcommand{\cafeCPlOblP}{\textipa{kofoloputu}}\newcommand{\cafeCPlOblG}{\cacherGloses{Obl-}café\cacherGloses{.C}\cacherGloses{xC.Pl}}\newcommand{\cafeCPlDat}{\strutgb{0pt}\grapho{O8UI5}}\newcommand{\cafeCPlDatP}{\textipa{lifoloputu}}\newcommand{\cafeCPlDatG}{\cacherGloses{Dat-}café\cacherGloses{.C}\cacherGloses{xC.Pl}}\newcommand{\chasseurCSgErg}{\strutgb{0pt}\grapho{RdpCR}}\newcommand{\chasseurCSgErgP}{\textipa{redabaNile}}\newcommand{\chasseurCSgErgG}{\cacherGloses{Erg-}chasseur\cacherGloses{.C}\cacherGloses{xSg}}\newcommand{\chasseurCSgAbs}{\strutgb{0pt}\grapho{HdpCR}}\newcommand{\chasseurCSgAbsP}{\textipa{bodabaNile}}\newcommand{\chasseurCSgAbsG}{\cacherGloses{Abs-}chasseur\cacherGloses{.C}\cacherGloses{xSg}}\newcommand{\chasseurCSgObl}{\strutgb{0pt}\grapho{hdpCR}}\newcommand{\chasseurCSgOblP}{\textipa{kodabaNile}}\newcommand{\chasseurCSgOblG}{\cacherGloses{Obl-}chasseur\cacherGloses{.C}\cacherGloses{xSg}}\newcommand{\chasseurCSgDat}{\strutgb{0pt}\grapho{OdpCR}}\newcommand{\chasseurCSgDatP}{\textipa{lidabaNile}}\newcommand{\chasseurCSgDatG}{\cacherGloses{Dat-}chasseur\cacherGloses{.C}\cacherGloses{xSg}}\newcommand{\chasseurCDuErg}{\strutgb{0pt}\grapho{RdyCr}}\newcommand{\chasseurCDuErgP}{\textipa{redamiNira}}\newcommand{\chasseurCDuErgG}{\cacherGloses{Erg-}chasseur\cacherGloses{.C}\cacherGloses{xC.Du}}\newcommand{\chasseurCDuAbs}{\strutgb{0pt}\grapho{HdyCr}}\newcommand{\chasseurCDuAbsP}{\textipa{bodamiNira}}\newcommand{\chasseurCDuAbsG}{\cacherGloses{Abs-}chasseur\cacherGloses{.C}\cacherGloses{xC.Du}}\newcommand{\chasseurCDuObl}{\strutgb{0pt}\grapho{hdyCr}}\newcommand{\chasseurCDuOblP}{\textipa{kodamiNira}}\newcommand{\chasseurCDuOblG}{\cacherGloses{Obl-}chasseur\cacherGloses{.C}\cacherGloses{xC.Du}}\newcommand{\chasseurCDuDat}{\strutgb{0pt}\grapho{OdyCr}}\newcommand{\chasseurCDuDatP}{\textipa{lidamiNira}}\newcommand{\chasseurCDuDatG}{\cacherGloses{Dat-}chasseur\cacherGloses{.C}\cacherGloses{xC.Du}}\newcommand{\chasseurCPlErg}{\strutgb{0pt}\grapho{RInwuV}}\newcommand{\chasseurCPlErgP}{\textipa{rebuNawulu}}\newcommand{\chasseurCPlErgG}{\cacherGloses{Erg-}chasseur\cacherGloses{.C}\cacherGloses{xC.Pl}}\newcommand{\chasseurCPlAbs}{\strutgb{0pt}\grapho{HInwuV}}\newcommand{\chasseurCPlAbsP}{\textipa{bobuNawulu}}\newcommand{\chasseurCPlAbsG}{\cacherGloses{Abs-}chasseur\cacherGloses{.C}\cacherGloses{xC.Pl}}\newcommand{\chasseurCPlObl}{\strutgb{0pt}\grapho{hInwuV}}\newcommand{\chasseurCPlOblP}{\textipa{kobuNawulu}}\newcommand{\chasseurCPlOblG}{\cacherGloses{Obl-}chasseur\cacherGloses{.C}\cacherGloses{xC.Pl}}\newcommand{\chasseurCPlDat}{\strutgb{0pt}\grapho{OInwuV}}\newcommand{\chasseurCPlDatP}{\textipa{libuNawulu}}\newcommand{\chasseurCPlDatG}{\cacherGloses{Dat-}chasseur\cacherGloses{.C}\cacherGloses{xC.Pl}}\newcommand{\NicoleBSgErg}{\strutgb{0pt}\grapho{RnmcR}}\newcommand{\NicoleBSgErgP}{\textipa{renamakile}}\newcommand{\NicoleBSgErgG}{\cacherGloses{Erg-}Nicole\cacherGloses{.B}\cacherGloses{xSg}}\newcommand{\NicoleBSgAbs}{\strutgb{0pt}\grapho{HnmcR}}\newcommand{\NicoleBSgAbsP}{\textipa{bonamakile}}\newcommand{\NicoleBSgAbsG}{\cacherGloses{Abs-}Nicole\cacherGloses{.B}\cacherGloses{xSg}}\newcommand{\NicoleBSgObl}{\strutgb{0pt}\grapho{hnmcR}}\newcommand{\NicoleBSgOblP}{\textipa{konamakile}}\newcommand{\NicoleBSgOblG}{\cacherGloses{Obl-}Nicole\cacherGloses{.B}\cacherGloses{xSg}}\newcommand{\NicoleBSgDat}{\strutgb{0pt}\grapho{OnmcR}}\newcommand{\NicoleBSgDatP}{\textipa{linamakile}}\newcommand{\NicoleBSgDatG}{\cacherGloses{Dat-}Nicole\cacherGloses{.B}\cacherGloses{xSg}}\newcommand{\NicoleBDuErg}{\strutgb{0pt}\grapho{RnGcO}}\newcommand{\NicoleBDuErgP}{\textipa{renapikiri}}\newcommand{\NicoleBDuErgG}{\cacherGloses{Erg-}Nicole\cacherGloses{.B}\cacherGloses{xB.Du}}\newcommand{\NicoleBDuAbs}{\strutgb{0pt}\grapho{HnGcO}}\newcommand{\NicoleBDuAbsP}{\textipa{bonapikiri}}\newcommand{\NicoleBDuAbsG}{\cacherGloses{Abs-}Nicole\cacherGloses{.B}\cacherGloses{xB.Du}}\newcommand{\NicoleBDuObl}{\strutgb{0pt}\grapho{hnGcO}}\newcommand{\NicoleBDuOblP}{\textipa{konapikiri}}\newcommand{\NicoleBDuOblG}{\cacherGloses{Obl-}Nicole\cacherGloses{.B}\cacherGloses{xB.Du}}\newcommand{\NicoleBDuDat}{\strutgb{0pt}\grapho{OnGcO}}\newcommand{\NicoleBDuDatP}{\textipa{linapikiri}}\newcommand{\NicoleBDuDatG}{\cacherGloses{Dat-}Nicole\cacherGloses{.B}\cacherGloses{xB.Du}}\newcommand{\NicoleBPlErg}{\strutgb{0pt}\grapho{RBkwur}}\newcommand{\NicoleBPlErgP}{\textipa{remukawula}}\newcommand{\NicoleBPlErgG}{\cacherGloses{Erg-}Nicole\cacherGloses{.B}\cacherGloses{xB.Pl}}\newcommand{\NicoleBPlAbs}{\strutgb{0pt}\grapho{HBkwur}}\newcommand{\NicoleBPlAbsP}{\textipa{bomukawula}}\newcommand{\NicoleBPlAbsG}{\cacherGloses{Abs-}Nicole\cacherGloses{.B}\cacherGloses{xB.Pl}}\newcommand{\NicoleBPlObl}{\strutgb{0pt}\grapho{hBkwur}}\newcommand{\NicoleBPlOblP}{\textipa{komukawula}}\newcommand{\NicoleBPlOblG}{\cacherGloses{Obl-}Nicole\cacherGloses{.B}\cacherGloses{xB.Pl}}\newcommand{\NicoleBPlDat}{\strutgb{0pt}\grapho{OBkwur}}\newcommand{\NicoleBPlDatP}{\textipa{limukawula}}\newcommand{\NicoleBPlDatG}{\cacherGloses{Dat-}Nicole\cacherGloses{.B}\cacherGloses{xB.Pl}}\newcommand{\theBSgErg}{\strutgb{0pt}\grapho{RmBrN}}\newcommand{\theBSgErgP}{\textipa{remamurane}}\newcommand{\theBSgErgG}{\cacherGloses{Erg-}thé\cacherGloses{.B}\cacherGloses{xSg}}\newcommand{\theBSgAbs}{\strutgb{0pt}\grapho{HmBrN}}\newcommand{\theBSgAbsP}{\textipa{bomamurane}}\newcommand{\theBSgAbsG}{\cacherGloses{Abs-}thé\cacherGloses{.B}\cacherGloses{xSg}}\newcommand{\theBSgObl}{\strutgb{0pt}\grapho{hmBrN}}\newcommand{\theBSgOblP}{\textipa{komamurane}}\newcommand{\theBSgOblG}{\cacherGloses{Obl-}thé\cacherGloses{.B}\cacherGloses{xSg}}\newcommand{\theBSgDat}{\strutgb{0pt}\grapho{OmBrN}}\newcommand{\theBSgDatP}{\textipa{limamurane}}\newcommand{\theBSgDatG}{\cacherGloses{Dat-}thé\cacherGloses{.B}\cacherGloses{xSg}}\newcommand{\theBDuErg}{\strutgb{0pt}\grapho{RBwOy}}\newcommand{\theBDuErgP}{\textipa{remuwarimi}}\newcommand{\theBDuErgG}{\cacherGloses{Erg-}thé\cacherGloses{.B}\cacherGloses{xB.Du}}\newcommand{\theBDuAbs}{\strutgb{0pt}\grapho{HBwOy}}\newcommand{\theBDuAbsP}{\textipa{bomuwarimi}}\newcommand{\theBDuAbsG}{\cacherGloses{Abs-}thé\cacherGloses{.B}\cacherGloses{xB.Du}}\newcommand{\theBDuObl}{\strutgb{0pt}\grapho{hBwOy}}\newcommand{\theBDuOblP}{\textipa{komuwarimi}}\newcommand{\theBDuOblG}{\cacherGloses{Obl-}thé\cacherGloses{.B}\cacherGloses{xB.Du}}\newcommand{\theBDuDat}{\strutgb{0pt}\grapho{OBwOy}}\newcommand{\theBDuDatP}{\textipa{limuwarimi}}\newcommand{\theBDuDatG}{\cacherGloses{Dat-}thé\cacherGloses{.B}\cacherGloses{xB.Du}}\newcommand{\theBPlErg}{\strutgb{0pt}\grapho{RBVBn}}\newcommand{\theBPlErgP}{\textipa{remurumuna}}\newcommand{\theBPlErgG}{\cacherGloses{Erg-}thé\cacherGloses{.B}\cacherGloses{xB.Pl}}\newcommand{\theBPlAbs}{\strutgb{0pt}\grapho{HBVBn}}\newcommand{\theBPlAbsP}{\textipa{bomurumuna}}\newcommand{\theBPlAbsG}{\cacherGloses{Abs-}thé\cacherGloses{.B}\cacherGloses{xB.Pl}}\newcommand{\theBPlObl}{\strutgb{0pt}\grapho{hBVBn}}\newcommand{\theBPlOblP}{\textipa{komurumuna}}\newcommand{\theBPlOblG}{\cacherGloses{Obl-}thé\cacherGloses{.B}\cacherGloses{xB.Pl}}\newcommand{\theBPlDat}{\strutgb{0pt}\grapho{OBVBn}}\newcommand{\theBPlDatP}{\textipa{limurumuna}}\newcommand{\theBPlDatG}{\cacherGloses{Dat-}thé\cacherGloses{.B}\cacherGloses{xB.Pl}}\newcommand{\sourisBSgErg}{\strutgb{0pt}\grapho{Rs1NN}}\newcommand{\sourisBSgErgP}{\textipa{reSasoNene}}\newcommand{\sourisBSgErgG}{\cacherGloses{Erg-}souris\cacherGloses{.B}\cacherGloses{xSg}}\newcommand{\sourisBSgAbs}{\strutgb{0pt}\grapho{Hs1NN}}\newcommand{\sourisBSgAbsP}{\textipa{boSasoNene}}\newcommand{\sourisBSgAbsG}{\cacherGloses{Abs-}souris\cacherGloses{.B}\cacherGloses{xSg}}\newcommand{\sourisBSgObl}{\strutgb{0pt}\grapho{hs1NN}}\newcommand{\sourisBSgOblP}{\textipa{koSasoNene}}\newcommand{\sourisBSgOblG}{\cacherGloses{Obl-}souris\cacherGloses{.B}\cacherGloses{xSg}}\newcommand{\sourisBSgDat}{\strutgb{0pt}\grapho{Os1NN}}\newcommand{\sourisBSgDatP}{\textipa{liSasoNene}}\newcommand{\sourisBSgDatG}{\cacherGloses{Dat-}souris\cacherGloses{.B}\cacherGloses{xSg}}\newcommand{\sourisBDuErg}{\strutgb{0pt}\grapho{R1MCy}}\newcommand{\sourisBDuErgP}{\textipa{reSomeNimi}}\newcommand{\sourisBDuErgG}{\cacherGloses{Erg-}souris\cacherGloses{.B}\cacherGloses{xB.Du}}\newcommand{\sourisBDuAbs}{\strutgb{0pt}\grapho{H1MCy}}\newcommand{\sourisBDuAbsP}{\textipa{boSomeNimi}}\newcommand{\sourisBDuAbsG}{\cacherGloses{Abs-}souris\cacherGloses{.B}\cacherGloses{xB.Du}}\newcommand{\sourisBDuObl}{\strutgb{0pt}\grapho{h1MCy}}\newcommand{\sourisBDuOblP}{\textipa{koSomeNimi}}\newcommand{\sourisBDuOblG}{\cacherGloses{Obl-}souris\cacherGloses{.B}\cacherGloses{xB.Du}}\newcommand{\sourisBDuDat}{\strutgb{0pt}\grapho{O1MCy}}\newcommand{\sourisBDuDatP}{\textipa{liSomeNimi}}\newcommand{\sourisBDuDatG}{\cacherGloses{Dat-}souris\cacherGloses{.B}\cacherGloses{xB.Du}}\newcommand{\sourisBPlErg}{\strutgb{0pt}\grapho{R1EBn}}\newcommand{\sourisBPlErgP}{\textipa{resoNomuna}}\newcommand{\sourisBPlErgG}{\cacherGloses{Erg-}souris\cacherGloses{.B}\cacherGloses{xB.Pl}}\newcommand{\sourisBPlAbs}{\strutgb{0pt}\grapho{H1EBn}}\newcommand{\sourisBPlAbsP}{\textipa{bosoNomuna}}\newcommand{\sourisBPlAbsG}{\cacherGloses{Abs-}souris\cacherGloses{.B}\cacherGloses{xB.Pl}}\newcommand{\sourisBPlObl}{\strutgb{0pt}\grapho{h1EBn}}\newcommand{\sourisBPlOblP}{\textipa{kosoNomuna}}\newcommand{\sourisBPlOblG}{\cacherGloses{Obl-}souris\cacherGloses{.B}\cacherGloses{xB.Pl}}\newcommand{\sourisBPlDat}{\strutgb{0pt}\grapho{O1EBn}}\newcommand{\sourisBPlDatP}{\textipa{lisoNomuna}}\newcommand{\sourisBPlDatG}{\cacherGloses{Dat-}souris\cacherGloses{.B}\cacherGloses{xB.Pl}}\newcommand{\chambreBSgErg}{\strutgb{0pt}\grapho{Rkd3N}}\newcommand{\chambreBSgErgP}{\textipa{regadatine}}\newcommand{\chambreBSgErgG}{\cacherGloses{Erg-}chambre\cacherGloses{.B}\cacherGloses{xSg}}\newcommand{\chambreBSgAbs}{\strutgb{0pt}\grapho{Hkd3N}}\newcommand{\chambreBSgAbsP}{\textipa{bogadatine}}\newcommand{\chambreBSgAbsG}{\cacherGloses{Abs-}chambre\cacherGloses{.B}\cacherGloses{xSg}}\newcommand{\chambreBSgObl}{\strutgb{0pt}\grapho{hkd3N}}\newcommand{\chambreBSgOblP}{\textipa{kogadatine}}\newcommand{\chambreBSgOblG}{\cacherGloses{Obl-}chambre\cacherGloses{.B}\cacherGloses{xSg}}\newcommand{\chambreBSgDat}{\strutgb{0pt}\grapho{Okd3N}}\newcommand{\chambreBSgDatP}{\textipa{ligadatine}}\newcommand{\chambreBSgDatG}{\cacherGloses{Dat-}chambre\cacherGloses{.B}\cacherGloses{xSg}}\newcommand{\chambreBDuErg}{\strutgb{0pt}\grapho{RkG3y}}\newcommand{\chambreBDuErgP}{\textipa{regapitimi}}\newcommand{\chambreBDuErgG}{\cacherGloses{Erg-}chambre\cacherGloses{.B}\cacherGloses{xB.Du}}\newcommand{\chambreBDuAbs}{\strutgb{0pt}\grapho{HkG3y}}\newcommand{\chambreBDuAbsP}{\textipa{bogapitimi}}\newcommand{\chambreBDuAbsG}{\cacherGloses{Abs-}chambre\cacherGloses{.B}\cacherGloses{xB.Du}}\newcommand{\chambreBDuObl}{\strutgb{0pt}\grapho{hkG3y}}\newcommand{\chambreBDuOblP}{\textipa{kogapitimi}}\newcommand{\chambreBDuOblG}{\cacherGloses{Obl-}chambre\cacherGloses{.B}\cacherGloses{xB.Du}}\newcommand{\chambreBDuDat}{\strutgb{0pt}\grapho{OkG3y}}\newcommand{\chambreBDuDatP}{\textipa{ligapitimi}}\newcommand{\chambreBDuDatG}{\cacherGloses{Dat-}chambre\cacherGloses{.B}\cacherGloses{xB.Du}}\newcommand{\chambreBPlErg}{\strutgb{0pt}\grapho{RxtBn}}\newcommand{\chambreBPlErgP}{\textipa{redutamuna}}\newcommand{\chambreBPlErgG}{\cacherGloses{Erg-}chambre\cacherGloses{.B}\cacherGloses{xB.Pl}}\newcommand{\chambreBPlAbs}{\strutgb{0pt}\grapho{HxtBn}}\newcommand{\chambreBPlAbsP}{\textipa{bodutamuna}}\newcommand{\chambreBPlAbsG}{\cacherGloses{Abs-}chambre\cacherGloses{.B}\cacherGloses{xB.Pl}}\newcommand{\chambreBPlObl}{\strutgb{0pt}\grapho{hxtBn}}\newcommand{\chambreBPlOblP}{\textipa{kodutamuna}}\newcommand{\chambreBPlOblG}{\cacherGloses{Obl-}chambre\cacherGloses{.B}\cacherGloses{xB.Pl}}\newcommand{\chambreBPlDat}{\strutgb{0pt}\grapho{OxtBn}}\newcommand{\chambreBPlDatP}{\textipa{lidutamuna}}\newcommand{\chambreBPlDatG}{\cacherGloses{Dat-}chambre\cacherGloses{.B}\cacherGloses{xB.Pl}}\newcommand{\autrucheBSgErg}{\strutgb{0pt}\grapho{RmmCT}}\newcommand{\autrucheBSgErgP}{\textipa{remamanite}}\newcommand{\autrucheBSgErgG}{\cacherGloses{Erg-}autruche\cacherGloses{.B}\cacherGloses{xSg}}\newcommand{\autrucheBSgAbs}{\strutgb{0pt}\grapho{HmmCT}}\newcommand{\autrucheBSgAbsP}{\textipa{bomamanite}}\newcommand{\autrucheBSgAbsG}{\cacherGloses{Abs-}autruche\cacherGloses{.B}\cacherGloses{xSg}}\newcommand{\autrucheBSgObl}{\strutgb{0pt}\grapho{hmmCT}}\newcommand{\autrucheBSgOblP}{\textipa{komamanite}}\newcommand{\autrucheBSgOblG}{\cacherGloses{Obl-}autruche\cacherGloses{.B}\cacherGloses{xSg}}\newcommand{\autrucheBSgDat}{\strutgb{0pt}\grapho{OmmCT}}\newcommand{\autrucheBSgDatP}{\textipa{limamanite}}\newcommand{\autrucheBSgDatG}{\cacherGloses{Dat-}autruche\cacherGloses{.B}\cacherGloses{xSg}}\newcommand{\autrucheBDuErg}{\strutgb{0pt}\grapho{RmyCG}}\newcommand{\autrucheBDuErgP}{\textipa{remaminipi}}\newcommand{\autrucheBDuErgG}{\cacherGloses{Erg-}autruche\cacherGloses{.B}\cacherGloses{xB.Du}}\newcommand{\autrucheBDuAbs}{\strutgb{0pt}\grapho{HmyCG}}\newcommand{\autrucheBDuAbsP}{\textipa{bomaminipi}}\newcommand{\autrucheBDuAbsG}{\cacherGloses{Abs-}autruche\cacherGloses{.B}\cacherGloses{xB.Du}}\newcommand{\autrucheBDuObl}{\strutgb{0pt}\grapho{hmyCG}}\newcommand{\autrucheBDuOblP}{\textipa{komaminipi}}\newcommand{\autrucheBDuOblG}{\cacherGloses{Obl-}autruche\cacherGloses{.B}\cacherGloses{xB.Du}}\newcommand{\autrucheBDuDat}{\strutgb{0pt}\grapho{OmyCG}}\newcommand{\autrucheBDuDatP}{\textipa{limaminipi}}\newcommand{\autrucheBDuDatG}{\cacherGloses{Dat-}autruche\cacherGloses{.B}\cacherGloses{xB.Du}}\newcommand{\autrucheBPlErg}{\strutgb{0pt}\grapho{RBnIt}}\newcommand{\autrucheBPlErgP}{\textipa{remunaputa}}\newcommand{\autrucheBPlErgG}{\cacherGloses{Erg-}autruche\cacherGloses{.B}\cacherGloses{xB.Pl}}\newcommand{\autrucheBPlAbs}{\strutgb{0pt}\grapho{HBnIt}}\newcommand{\autrucheBPlAbsP}{\textipa{bomunaputa}}\newcommand{\autrucheBPlAbsG}{\cacherGloses{Abs-}autruche\cacherGloses{.B}\cacherGloses{xB.Pl}}\newcommand{\autrucheBPlObl}{\strutgb{0pt}\grapho{hBnIt}}\newcommand{\autrucheBPlOblP}{\textipa{komunaputa}}\newcommand{\autrucheBPlOblG}{\cacherGloses{Obl-}autruche\cacherGloses{.B}\cacherGloses{xB.Pl}}\newcommand{\autrucheBPlDat}{\strutgb{0pt}\grapho{OBnIt}}\newcommand{\autrucheBPlDatP}{\textipa{limunaputa}}\newcommand{\autrucheBPlDatG}{\cacherGloses{Dat-}autruche\cacherGloses{.B}\cacherGloses{xB.Pl}}\newcommand{\coussinBSgErg}{\strutgb{0pt}\grapho{RsqupM}}\newcommand{\coussinBSgErgP}{\textipa{resafupame}}\newcommand{\coussinBSgErgG}{\cacherGloses{Erg-}coussin\cacherGloses{.B}\cacherGloses{xSg}}\newcommand{\coussinBSgAbs}{\strutgb{0pt}\grapho{HsqupM}}\newcommand{\coussinBSgAbsP}{\textipa{bosafupame}}\newcommand{\coussinBSgAbsG}{\cacherGloses{Abs-}coussin\cacherGloses{.B}\cacherGloses{xSg}}\newcommand{\coussinBSgObl}{\strutgb{0pt}\grapho{hsqupM}}\newcommand{\coussinBSgOblP}{\textipa{kosafupame}}\newcommand{\coussinBSgOblG}{\cacherGloses{Obl-}coussin\cacherGloses{.B}\cacherGloses{xSg}}\newcommand{\coussinBSgDat}{\strutgb{0pt}\grapho{OsqupM}}\newcommand{\coussinBSgDatP}{\textipa{lisafupame}}\newcommand{\coussinBSgDatG}{\cacherGloses{Dat-}coussin\cacherGloses{.B}\cacherGloses{xSg}}\newcommand{\coussinBDuErg}{\strutgb{0pt}\grapho{R2pGy}}\newcommand{\coussinBDuErgP}{\textipa{resupapimi}}\newcommand{\coussinBDuErgG}{\cacherGloses{Erg-}coussin\cacherGloses{.B}\cacherGloses{xB.Du}}\newcommand{\coussinBDuAbs}{\strutgb{0pt}\grapho{H2pGy}}\newcommand{\coussinBDuAbsP}{\textipa{bosupapimi}}\newcommand{\coussinBDuAbsG}{\cacherGloses{Abs-}coussin\cacherGloses{.B}\cacherGloses{xB.Du}}\newcommand{\coussinBDuObl}{\strutgb{0pt}\grapho{h2pGy}}\newcommand{\coussinBDuOblP}{\textipa{kosupapimi}}\newcommand{\coussinBDuOblG}{\cacherGloses{Obl-}coussin\cacherGloses{.B}\cacherGloses{xB.Du}}\newcommand{\coussinBDuDat}{\strutgb{0pt}\grapho{O2pGy}}\newcommand{\coussinBDuDatP}{\textipa{lisupapimi}}\newcommand{\coussinBDuDatG}{\cacherGloses{Dat-}coussin\cacherGloses{.B}\cacherGloses{xB.Du}}\newcommand{\coussinBPlErg}{\strutgb{0pt}\grapho{RquIBm}}\newcommand{\coussinBPlErgP}{\textipa{refupumuma}}\newcommand{\coussinBPlErgG}{\cacherGloses{Erg-}coussin\cacherGloses{.B}\cacherGloses{xB.Pl}}\newcommand{\coussinBPlAbs}{\strutgb{0pt}\grapho{HquIBm}}\newcommand{\coussinBPlAbsP}{\textipa{bofupumuma}}\newcommand{\coussinBPlAbsG}{\cacherGloses{Abs-}coussin\cacherGloses{.B}\cacherGloses{xB.Pl}}\newcommand{\coussinBPlObl}{\strutgb{0pt}\grapho{hquIBm}}\newcommand{\coussinBPlOblP}{\textipa{kofupumuma}}\newcommand{\coussinBPlOblG}{\cacherGloses{Obl-}coussin\cacherGloses{.B}\cacherGloses{xB.Pl}}\newcommand{\coussinBPlDat}{\strutgb{0pt}\grapho{OquIBm}}\newcommand{\coussinBPlDatP}{\textipa{lifupumuma}}\newcommand{\coussinBPlDatG}{\cacherGloses{Dat-}coussin\cacherGloses{.B}\cacherGloses{xB.Pl}}\newcommand{\maisonDSgErg}{\strutgb{0pt}\grapho{Rk5VZ}}\newcommand{\maisonDSgErgP}{\textipa{rekaturuze}}\newcommand{\maisonDSgErgG}{\cacherGloses{Erg-}maison\cacherGloses{.D}\cacherGloses{xSg}}\newcommand{\maisonDSgAbs}{\strutgb{0pt}\grapho{Hk5VZ}}\newcommand{\maisonDSgAbsP}{\textipa{bokaturuze}}\newcommand{\maisonDSgAbsG}{\cacherGloses{Abs-}maison\cacherGloses{.D}\cacherGloses{xSg}}\newcommand{\maisonDSgObl}{\strutgb{0pt}\grapho{hk5VZ}}\newcommand{\maisonDSgOblP}{\textipa{kokaturuze}}\newcommand{\maisonDSgOblG}{\cacherGloses{Obl-}maison\cacherGloses{.D}\cacherGloses{xSg}}\newcommand{\maisonDSgDat}{\strutgb{0pt}\grapho{Ok5VZ}}\newcommand{\maisonDSgDatP}{\textipa{likaturuze}}\newcommand{\maisonDSgDatG}{\cacherGloses{Dat-}maison\cacherGloses{.D}\cacherGloses{xSg}}\newcommand{\maisonDDuErg}{\strutgb{0pt}\grapho{RvwuOqu}}\newcommand{\maisonDDuErgP}{\textipa{rekuwurivu}}\newcommand{\maisonDDuErgG}{\cacherGloses{Erg-}maison\cacherGloses{.D}\cacherGloses{xD.Du}}\newcommand{\maisonDDuAbs}{\strutgb{0pt}\grapho{HvwuOqu}}\newcommand{\maisonDDuAbsP}{\textipa{bokuwurivu}}\newcommand{\maisonDDuAbsG}{\cacherGloses{Abs-}maison\cacherGloses{.D}\cacherGloses{xD.Du}}\newcommand{\maisonDDuObl}{\strutgb{0pt}\grapho{hvwuOqu}}\newcommand{\maisonDDuOblP}{\textipa{kokuwurivu}}\newcommand{\maisonDDuOblG}{\cacherGloses{Obl-}maison\cacherGloses{.D}\cacherGloses{xD.Du}}\newcommand{\maisonDDuDat}{\strutgb{0pt}\grapho{OvwuOqu}}\newcommand{\maisonDDuDatP}{\textipa{likuwurivu}}\newcommand{\maisonDDuDatG}{\cacherGloses{Dat-}maison\cacherGloses{.D}\cacherGloses{xD.Du}}\newcommand{\maisonDPlErg}{\strutgb{0pt}\grapho{R5Vquzu}}\newcommand{\maisonDPlErgP}{\textipa{returufuzu}}\newcommand{\maisonDPlErgG}{\cacherGloses{Erg-}maison\cacherGloses{.D}\cacherGloses{xD.Pl}}\newcommand{\maisonDPlAbs}{\strutgb{0pt}\grapho{H5Vquzu}}\newcommand{\maisonDPlAbsP}{\textipa{boturufuzu}}\newcommand{\maisonDPlAbsG}{\cacherGloses{Abs-}maison\cacherGloses{.D}\cacherGloses{xD.Pl}}\newcommand{\maisonDPlObl}{\strutgb{0pt}\grapho{h5Vquzu}}\newcommand{\maisonDPlOblP}{\textipa{koturufuzu}}\newcommand{\maisonDPlOblG}{\cacherGloses{Obl-}maison\cacherGloses{.D}\cacherGloses{xD.Pl}}\newcommand{\maisonDPlDat}{\strutgb{0pt}\grapho{O5Vquzu}}\newcommand{\maisonDPlDatP}{\textipa{liturufuzu}}\newcommand{\maisonDPlDatG}{\cacherGloses{Dat-}maison\cacherGloses{.D}\cacherGloses{xD.Pl}}\newcommand{\tableDSgErg}{\strutgb{0pt}\grapho{RnFzN}}\newcommand{\tableDSgErgP}{\textipa{reNanuzane}}\newcommand{\tableDSgErgG}{\cacherGloses{Erg-}table\cacherGloses{.D}\cacherGloses{xSg}}\newcommand{\tableDSgAbs}{\strutgb{0pt}\grapho{HnFzN}}\newcommand{\tableDSgAbsP}{\textipa{boNanuzane}}\newcommand{\tableDSgAbsG}{\cacherGloses{Abs-}table\cacherGloses{.D}\cacherGloses{xSg}}\newcommand{\tableDSgObl}{\strutgb{0pt}\grapho{hnFzN}}\newcommand{\tableDSgOblP}{\textipa{koNanuzane}}\newcommand{\tableDSgOblG}{\cacherGloses{Obl-}table\cacherGloses{.D}\cacherGloses{xSg}}\newcommand{\tableDSgDat}{\strutgb{0pt}\grapho{OnFzN}}\newcommand{\tableDSgDatP}{\textipa{liNanuzane}}\newcommand{\tableDSgDatG}{\cacherGloses{Dat-}table\cacherGloses{.D}\cacherGloses{xSg}}\newcommand{\tableDDuErg}{\strutgb{0pt}\grapho{RFqziB}}\newcommand{\tableDDuErgP}{\textipa{reNufazimu}}\newcommand{\tableDDuErgG}{\cacherGloses{Erg-}table\cacherGloses{.D}\cacherGloses{xD.Du}}\newcommand{\tableDDuAbs}{\strutgb{0pt}\grapho{HFqziB}}\newcommand{\tableDDuAbsP}{\textipa{boNufazimu}}\newcommand{\tableDDuAbsG}{\cacherGloses{Abs-}table\cacherGloses{.D}\cacherGloses{xD.Du}}\newcommand{\tableDDuObl}{\strutgb{0pt}\grapho{hFqziB}}\newcommand{\tableDDuOblP}{\textipa{koNufazimu}}\newcommand{\tableDDuOblG}{\cacherGloses{Obl-}table\cacherGloses{.D}\cacherGloses{xD.Du}}\newcommand{\tableDDuDat}{\strutgb{0pt}\grapho{OFqziB}}\newcommand{\tableDDuDatP}{\textipa{liNufazimu}}\newcommand{\tableDDuDatG}{\cacherGloses{Dat-}table\cacherGloses{.D}\cacherGloses{xD.Du}}\newcommand{\tableDPlErg}{\strutgb{0pt}\grapho{RFzuBF}}\newcommand{\tableDPlErgP}{\textipa{renuzumunu}}\newcommand{\tableDPlErgG}{\cacherGloses{Erg-}table\cacherGloses{.D}\cacherGloses{xD.Pl}}\newcommand{\tableDPlAbs}{\strutgb{0pt}\grapho{HFzuBF}}\newcommand{\tableDPlAbsP}{\textipa{bonuzumunu}}\newcommand{\tableDPlAbsG}{\cacherGloses{Abs-}table\cacherGloses{.D}\cacherGloses{xD.Pl}}\newcommand{\tableDPlObl}{\strutgb{0pt}\grapho{hFzuBF}}\newcommand{\tableDPlOblP}{\textipa{konuzumunu}}\newcommand{\tableDPlOblG}{\cacherGloses{Obl-}table\cacherGloses{.D}\cacherGloses{xD.Pl}}\newcommand{\tableDPlDat}{\strutgb{0pt}\grapho{OFzuBF}}\newcommand{\tableDPlDatP}{\textipa{linuzumunu}}\newcommand{\tableDPlDatG}{\cacherGloses{Dat-}table\cacherGloses{.D}\cacherGloses{xD.Pl}}\newcommand{\garconDSgErg}{\strutgb{0pt}\grapho{RmASZ}}\newcommand{\garconDSgErgP}{\textipa{remamoseZe}}\newcommand{\garconDSgErgG}{\cacherGloses{Erg-}garçon\cacherGloses{.D}\cacherGloses{xSg}}\newcommand{\garconDSgAbs}{\strutgb{0pt}\grapho{HmASZ}}\newcommand{\garconDSgAbsP}{\textipa{bomamoseZe}}\newcommand{\garconDSgAbsG}{\cacherGloses{Abs-}garçon\cacherGloses{.D}\cacherGloses{xSg}}\newcommand{\garconDSgObl}{\strutgb{0pt}\grapho{hmASZ}}\newcommand{\garconDSgOblP}{\textipa{komamoseZe}}\newcommand{\garconDSgOblG}{\cacherGloses{Obl-}garçon\cacherGloses{.D}\cacherGloses{xSg}}\newcommand{\garconDSgDat}{\strutgb{0pt}\grapho{OmASZ}}\newcommand{\garconDSgDatP}{\textipa{limamoseZe}}\newcommand{\garconDSgDatG}{\cacherGloses{Dat-}garçon\cacherGloses{.D}\cacherGloses{xSg}}\newcommand{\garconDDuErg}{\strutgb{0pt}\grapho{RAQYzu}}\newcommand{\garconDDuErgP}{\textipa{remofesizu}}\newcommand{\garconDDuErgG}{\cacherGloses{Erg-}garçon\cacherGloses{.D}\cacherGloses{xD.Du}}\newcommand{\garconDDuAbs}{\strutgb{0pt}\grapho{HAQYzu}}\newcommand{\garconDDuAbsP}{\textipa{bomofesizu}}\newcommand{\garconDDuAbsG}{\cacherGloses{Abs-}garçon\cacherGloses{.D}\cacherGloses{xD.Du}}\newcommand{\garconDDuObl}{\strutgb{0pt}\grapho{hAQYzu}}\newcommand{\garconDDuOblP}{\textipa{komofesizu}}\newcommand{\garconDDuOblG}{\cacherGloses{Obl-}garçon\cacherGloses{.D}\cacherGloses{xD.Du}}\newcommand{\garconDDuDat}{\strutgb{0pt}\grapho{OAQYzu}}\newcommand{\garconDDuDatP}{\textipa{limofesizu}}\newcommand{\garconDDuDatG}{\cacherGloses{Dat-}garçon\cacherGloses{.D}\cacherGloses{xD.Du}}\newcommand{\garconDPlErg}{\strutgb{0pt}\grapho{RA1quzu}}\newcommand{\garconDPlErgP}{\textipa{remosovuZu}}\newcommand{\garconDPlErgG}{\cacherGloses{Erg-}garçon\cacherGloses{.D}\cacherGloses{xD.Pl}}\newcommand{\garconDPlAbs}{\strutgb{0pt}\grapho{HA1quzu}}\newcommand{\garconDPlAbsP}{\textipa{bomosovuZu}}\newcommand{\garconDPlAbsG}{\cacherGloses{Abs-}garçon\cacherGloses{.D}\cacherGloses{xD.Pl}}\newcommand{\garconDPlObl}{\strutgb{0pt}\grapho{hA1quzu}}\newcommand{\garconDPlOblP}{\textipa{komosovuZu}}\newcommand{\garconDPlOblG}{\cacherGloses{Obl-}garçon\cacherGloses{.D}\cacherGloses{xD.Pl}}\newcommand{\garconDPlDat}{\strutgb{0pt}\grapho{OA1quzu}}\newcommand{\garconDPlDatP}{\textipa{limosovuZu}}\newcommand{\garconDPlDatG}{\cacherGloses{Dat-}garçon\cacherGloses{.D}\cacherGloses{xD.Pl}}\newcommand{\litDSgErg}{\strutgb{0pt}\grapho{RkgHD}}\newcommand{\litDSgErgP}{\textipa{regadobode}}\newcommand{\litDSgErgG}{\cacherGloses{Erg-}lit\cacherGloses{.D}\cacherGloses{xSg}}\newcommand{\litDSgAbs}{\strutgb{0pt}\grapho{HkgHD}}\newcommand{\litDSgAbsP}{\textipa{bogadobode}}\newcommand{\litDSgAbsG}{\cacherGloses{Abs-}lit\cacherGloses{.D}\cacherGloses{xSg}}\newcommand{\litDSgObl}{\strutgb{0pt}\grapho{hkgHD}}\newcommand{\litDSgOblP}{\textipa{kogadobode}}\newcommand{\litDSgOblG}{\cacherGloses{Obl-}lit\cacherGloses{.D}\cacherGloses{xSg}}\newcommand{\litDSgDat}{\strutgb{0pt}\grapho{OkgHD}}\newcommand{\litDSgDatP}{\textipa{ligadobode}}\newcommand{\litDSgDatG}{\cacherGloses{Dat-}lit\cacherGloses{.D}\cacherGloses{xSg}}\newcommand{\litDDuErg}{\strutgb{0pt}\grapho{RhHGI}}\newcommand{\litDDuErgP}{\textipa{regopobibu}}\newcommand{\litDDuErgG}{\cacherGloses{Erg-}lit\cacherGloses{.D}\cacherGloses{xD.Du}}\newcommand{\litDDuAbs}{\strutgb{0pt}\grapho{HhHGI}}\newcommand{\litDDuAbsP}{\textipa{bogopobibu}}\newcommand{\litDDuAbsG}{\cacherGloses{Abs-}lit\cacherGloses{.D}\cacherGloses{xD.Du}}\newcommand{\litDDuObl}{\strutgb{0pt}\grapho{hhHGI}}\newcommand{\litDDuOblP}{\textipa{kogopobibu}}\newcommand{\litDDuOblG}{\cacherGloses{Obl-}lit\cacherGloses{.D}\cacherGloses{xD.Du}}\newcommand{\litDDuDat}{\strutgb{0pt}\grapho{OhHGI}}\newcommand{\litDDuDatP}{\textipa{ligopobibu}}\newcommand{\litDDuDatG}{\cacherGloses{Dat-}lit\cacherGloses{.D}\cacherGloses{xD.Du}}\newcommand{\litDPlErg}{\strutgb{0pt}\grapho{RgHIx}}\newcommand{\litDPlErgP}{\textipa{redobopudu}}\newcommand{\litDPlErgG}{\cacherGloses{Erg-}lit\cacherGloses{.D}\cacherGloses{xD.Pl}}\newcommand{\litDPlAbs}{\strutgb{0pt}\grapho{HgHIx}}\newcommand{\litDPlAbsP}{\textipa{bodobopudu}}\newcommand{\litDPlAbsG}{\cacherGloses{Abs-}lit\cacherGloses{.D}\cacherGloses{xD.Pl}}\newcommand{\litDPlObl}{\strutgb{0pt}\grapho{hgHIx}}\newcommand{\litDPlOblP}{\textipa{kodobopudu}}\newcommand{\litDPlOblG}{\cacherGloses{Obl-}lit\cacherGloses{.D}\cacherGloses{xD.Pl}}\newcommand{\litDPlDat}{\strutgb{0pt}\grapho{OgHIx}}\newcommand{\litDPlDatP}{\textipa{lidobopudu}}\newcommand{\litDPlDatG}{\cacherGloses{Dat-}lit\cacherGloses{.D}\cacherGloses{xD.Pl}}\newcommand{\cuisineDSgErg}{\strutgb{0pt}\grapho{RpIVP}}\newcommand{\cuisineDSgErgP}{\textipa{repapulupe}}\newcommand{\cuisineDSgErgG}{\cacherGloses{Erg-}cuisine\cacherGloses{.D}\cacherGloses{xSg}}\newcommand{\cuisineDSgAbs}{\strutgb{0pt}\grapho{HpIVP}}\newcommand{\cuisineDSgAbsP}{\textipa{bopapulupe}}\newcommand{\cuisineDSgAbsG}{\cacherGloses{Abs-}cuisine\cacherGloses{.D}\cacherGloses{xSg}}\newcommand{\cuisineDSgObl}{\strutgb{0pt}\grapho{hpIVP}}\newcommand{\cuisineDSgOblP}{\textipa{kopapulupe}}\newcommand{\cuisineDSgOblG}{\cacherGloses{Obl-}cuisine\cacherGloses{.D}\cacherGloses{xSg}}\newcommand{\cuisineDSgDat}{\strutgb{0pt}\grapho{OpIVP}}\newcommand{\cuisineDSgDatP}{\textipa{lipapulupe}}\newcommand{\cuisineDSgDatG}{\cacherGloses{Dat-}cuisine\cacherGloses{.D}\cacherGloses{xSg}}\newcommand{\cuisineDDuErg}{\strutgb{0pt}\grapho{RIwuOI}}\newcommand{\cuisineDDuErgP}{\textipa{repuwulipu}}\newcommand{\cuisineDDuErgG}{\cacherGloses{Erg-}cuisine\cacherGloses{.D}\cacherGloses{xD.Du}}\newcommand{\cuisineDDuAbs}{\strutgb{0pt}\grapho{HIwuOI}}\newcommand{\cuisineDDuAbsP}{\textipa{bopuwulipu}}\newcommand{\cuisineDDuAbsG}{\cacherGloses{Abs-}cuisine\cacherGloses{.D}\cacherGloses{xD.Du}}\newcommand{\cuisineDDuObl}{\strutgb{0pt}\grapho{hIwuOI}}\newcommand{\cuisineDDuOblP}{\textipa{kopuwulipu}}\newcommand{\cuisineDDuOblG}{\cacherGloses{Obl-}cuisine\cacherGloses{.D}\cacherGloses{xD.Du}}\newcommand{\cuisineDDuDat}{\strutgb{0pt}\grapho{OIwuOI}}\newcommand{\cuisineDDuDatP}{\textipa{lipuwulipu}}\newcommand{\cuisineDDuDatG}{\cacherGloses{Dat-}cuisine\cacherGloses{.D}\cacherGloses{xD.Du}}\newcommand{\cuisineDPlErg}{\strutgb{0pt}\grapho{RIVII}}\newcommand{\cuisineDPlErgP}{\textipa{repulupupu}}\newcommand{\cuisineDPlErgG}{\cacherGloses{Erg-}cuisine\cacherGloses{.D}\cacherGloses{xD.Pl}}\newcommand{\cuisineDPlAbs}{\strutgb{0pt}\grapho{HIVII}}\newcommand{\cuisineDPlAbsP}{\textipa{bopulupupu}}\newcommand{\cuisineDPlAbsG}{\cacherGloses{Abs-}cuisine\cacherGloses{.D}\cacherGloses{xD.Pl}}\newcommand{\cuisineDPlObl}{\strutgb{0pt}\grapho{hIVII}}\newcommand{\cuisineDPlOblP}{\textipa{kopulupupu}}\newcommand{\cuisineDPlOblG}{\cacherGloses{Obl-}cuisine\cacherGloses{.D}\cacherGloses{xD.Pl}}\newcommand{\cuisineDPlDat}{\strutgb{0pt}\grapho{OIVII}}\newcommand{\cuisineDPlDatP}{\textipa{lipulupupu}}\newcommand{\cuisineDPlDatG}{\cacherGloses{Dat-}cuisine\cacherGloses{.D}\cacherGloses{xD.Pl}}\newcommand{\NabilDSgErg}{\strutgb{0pt}\grapho{RnBpR}}\newcommand{\NabilDSgErgP}{\textipa{renamubale}}\newcommand{\NabilDSgErgG}{\cacherGloses{Erg-}Nabil\cacherGloses{.D}\cacherGloses{xSg}}\newcommand{\NabilDSgAbs}{\strutgb{0pt}\grapho{HnBpR}}\newcommand{\NabilDSgAbsP}{\textipa{bonamubale}}\newcommand{\NabilDSgAbsG}{\cacherGloses{Abs-}Nabil\cacherGloses{.D}\cacherGloses{xSg}}\newcommand{\NabilDSgObl}{\strutgb{0pt}\grapho{hnBpR}}\newcommand{\NabilDSgOblP}{\textipa{konamubale}}\newcommand{\NabilDSgOblG}{\cacherGloses{Obl-}Nabil\cacherGloses{.D}\cacherGloses{xSg}}\newcommand{\NabilDSgDat}{\strutgb{0pt}\grapho{OnBpR}}\newcommand{\NabilDSgDatP}{\textipa{linamubale}}\newcommand{\NabilDSgDatG}{\cacherGloses{Dat-}Nabil\cacherGloses{.D}\cacherGloses{xSg}}\newcommand{\NabilDDuErg}{\strutgb{0pt}\grapho{RFpGV}}\newcommand{\NabilDDuErgP}{\textipa{renupabiru}}\newcommand{\NabilDDuErgG}{\cacherGloses{Erg-}Nabil\cacherGloses{.D}\cacherGloses{xD.Du}}\newcommand{\NabilDDuAbs}{\strutgb{0pt}\grapho{HFpGV}}\newcommand{\NabilDDuAbsP}{\textipa{bonupabiru}}\newcommand{\NabilDDuAbsG}{\cacherGloses{Abs-}Nabil\cacherGloses{.D}\cacherGloses{xD.Du}}\newcommand{\NabilDDuObl}{\strutgb{0pt}\grapho{hFpGV}}\newcommand{\NabilDDuOblP}{\textipa{konupabiru}}\newcommand{\NabilDDuOblG}{\cacherGloses{Obl-}Nabil\cacherGloses{.D}\cacherGloses{xD.Du}}\newcommand{\NabilDDuDat}{\strutgb{0pt}\grapho{OFpGV}}\newcommand{\NabilDDuDatP}{\textipa{linupabiru}}\newcommand{\NabilDDuDatG}{\cacherGloses{Dat-}Nabil\cacherGloses{.D}\cacherGloses{xD.Du}}\newcommand{\NabilDPlErg}{\strutgb{0pt}\grapho{RBIwuV}}\newcommand{\NabilDPlErgP}{\textipa{remubuwulu}}\newcommand{\NabilDPlErgG}{\cacherGloses{Erg-}Nabil\cacherGloses{.D}\cacherGloses{xD.Pl}}\newcommand{\NabilDPlAbs}{\strutgb{0pt}\grapho{HBIwuV}}\newcommand{\NabilDPlAbsP}{\textipa{bomubuwulu}}\newcommand{\NabilDPlAbsG}{\cacherGloses{Abs-}Nabil\cacherGloses{.D}\cacherGloses{xD.Pl}}\newcommand{\NabilDPlObl}{\strutgb{0pt}\grapho{hBIwuV}}\newcommand{\NabilDPlOblP}{\textipa{komubuwulu}}\newcommand{\NabilDPlOblG}{\cacherGloses{Obl-}Nabil\cacherGloses{.D}\cacherGloses{xD.Pl}}\newcommand{\NabilDPlDat}{\strutgb{0pt}\grapho{OBIwuV}}\newcommand{\NabilDPlDatP}{\textipa{limubuwulu}}\newcommand{\NabilDPlDatG}{\cacherGloses{Dat-}Nabil\cacherGloses{.D}\cacherGloses{xD.Pl}}\newcommand{\chatDSgErg}{\strutgb{0pt}\grapho{RrwuBK}}\newcommand{\chatDSgErgP}{\textipa{rerawumuke}}\newcommand{\chatDSgErgG}{\cacherGloses{Erg-}chat\cacherGloses{.D}\cacherGloses{xSg}}\newcommand{\chatDSgAbs}{\strutgb{0pt}\grapho{HrwuBK}}\newcommand{\chatDSgAbsP}{\textipa{borawumuke}}\newcommand{\chatDSgAbsG}{\cacherGloses{Abs-}chat\cacherGloses{.D}\cacherGloses{xSg}}\newcommand{\chatDSgObl}{\strutgb{0pt}\grapho{hrwuBK}}\newcommand{\chatDSgOblP}{\textipa{korawumuke}}\newcommand{\chatDSgOblG}{\cacherGloses{Obl-}chat\cacherGloses{.D}\cacherGloses{xSg}}\newcommand{\chatDSgDat}{\strutgb{0pt}\grapho{OrwuBK}}\newcommand{\chatDSgDatP}{\textipa{lirawumuke}}\newcommand{\chatDSgDatG}{\cacherGloses{Dat-}chat\cacherGloses{.D}\cacherGloses{xSg}}\newcommand{\chatDDuErg}{\strutgb{0pt}\grapho{RVBy5}}\newcommand{\chatDDuErgP}{\textipa{rerumumitu}}\newcommand{\chatDDuErgG}{\cacherGloses{Erg-}chat\cacherGloses{.D}\cacherGloses{xD.Du}}\newcommand{\chatDDuAbs}{\strutgb{0pt}\grapho{HVBy5}}\newcommand{\chatDDuAbsP}{\textipa{borumumitu}}\newcommand{\chatDDuAbsG}{\cacherGloses{Abs-}chat\cacherGloses{.D}\cacherGloses{xD.Du}}\newcommand{\chatDDuObl}{\strutgb{0pt}\grapho{hVBy5}}\newcommand{\chatDDuOblP}{\textipa{korumumitu}}\newcommand{\chatDDuOblG}{\cacherGloses{Obl-}chat\cacherGloses{.D}\cacherGloses{xD.Du}}\newcommand{\chatDDuDat}{\strutgb{0pt}\grapho{OVBy5}}\newcommand{\chatDDuDatP}{\textipa{lirumumitu}}\newcommand{\chatDDuDatG}{\cacherGloses{Dat-}chat\cacherGloses{.D}\cacherGloses{xD.Du}}\newcommand{\chatDPlErg}{\strutgb{0pt}\grapho{RwuBIv}}\newcommand{\chatDPlErgP}{\textipa{rewumupuku}}\newcommand{\chatDPlErgG}{\cacherGloses{Erg-}chat\cacherGloses{.D}\cacherGloses{xD.Pl}}\newcommand{\chatDPlAbs}{\strutgb{0pt}\grapho{HwuBIv}}\newcommand{\chatDPlAbsP}{\textipa{bowumupuku}}\newcommand{\chatDPlAbsG}{\cacherGloses{Abs-}chat\cacherGloses{.D}\cacherGloses{xD.Pl}}\newcommand{\chatDPlObl}{\strutgb{0pt}\grapho{hwuBIv}}\newcommand{\chatDPlOblP}{\textipa{kowumupuku}}\newcommand{\chatDPlOblG}{\cacherGloses{Obl-}chat\cacherGloses{.D}\cacherGloses{xD.Pl}}\newcommand{\chatDPlDat}{\strutgb{0pt}\grapho{OwuBIv}}\newcommand{\chatDPlDatP}{\textipa{liwumupuku}}\newcommand{\chatDPlDatG}{\cacherGloses{Dat-}chat\cacherGloses{.D}\cacherGloses{xD.Pl}}\newcommand{\INDSgErg}{\strutgb{0pt}\grapho{hm}}\newcommand{\INDSgErgP}{\textipa{kom}}\newcommand{\INDSgErgG}{IND\cacherGloses{-Sg}\cacherGloses{-Erg}}\newcommand{\INDSgAbs}{\strutgb{0pt}\grapho{hk}}\newcommand{\INDSgAbsP}{\textipa{kok}}\newcommand{\INDSgAbsG}{IND\cacherGloses{-Sg}\cacherGloses{-Abs}}\newcommand{\INDSgObl}{\strutgb{0pt}\grapho{hk}}\newcommand{\INDSgOblP}{\textipa{kog}}\newcommand{\INDSgOblG}{IND\cacherGloses{-Sg}\cacherGloses{-Obl}}\newcommand{\INDSgDat}{\strutgb{0pt}\grapho{ht}}\newcommand{\INDSgDatP}{\textipa{kot}}\newcommand{\INDSgDatG}{IND\cacherGloses{-Sg}\cacherGloses{-Dat}}\newcommand{\INDDuErg}{\strutgb{0pt}\grapho{km}}\newcommand{\INDDuErgP}{\textipa{kam}}\newcommand{\INDDuErgG}{IND\cacherGloses{-Du}\cacherGloses{-Erg}}\newcommand{\INDDuAbs}{\strutgb{0pt}\grapho{kk}}\newcommand{\INDDuAbsP}{\textipa{kak}}\newcommand{\INDDuAbsG}{IND\cacherGloses{-Du}\cacherGloses{-Abs}}\newcommand{\INDDuObl}{\strutgb{0pt}\grapho{kk}}\newcommand{\INDDuOblP}{\textipa{kag}}\newcommand{\INDDuOblG}{IND\cacherGloses{-Du}\cacherGloses{-Obl}}\newcommand{\INDDuDat}{\strutgb{0pt}\grapho{kt}}\newcommand{\INDDuDatP}{\textipa{kat}}\newcommand{\INDDuDatG}{IND\cacherGloses{-Du}\cacherGloses{-Dat}}\newcommand{\INDPlErg}{\strutgb{0pt}\grapho{cm}}\newcommand{\INDPlErgP}{\textipa{kim}}\newcommand{\INDPlErgG}{IND\cacherGloses{-Pl}\cacherGloses{-Erg}}\newcommand{\INDPlAbs}{\strutgb{0pt}\grapho{ck}}\newcommand{\INDPlAbsP}{\textipa{kik}}\newcommand{\INDPlAbsG}{IND\cacherGloses{-Pl}\cacherGloses{-Abs}}\newcommand{\INDPlObl}{\strutgb{0pt}\grapho{ck}}\newcommand{\INDPlOblP}{\textipa{kig}}\newcommand{\INDPlOblG}{IND\cacherGloses{-Pl}\cacherGloses{-Obl}}\newcommand{\INDPlDat}{\strutgb{0pt}\grapho{ct}}\newcommand{\INDPlDatP}{\textipa{kit}}\newcommand{\INDPlDatG}{IND\cacherGloses{-Pl}\cacherGloses{-Dat}}\newcommand{\DEFSgErg}{\strutgb{0pt}\grapho{Hm}}\newcommand{\DEFSgErgP}{\textipa{bom}}\newcommand{\DEFSgErgG}{DEF\cacherGloses{-Sg}\cacherGloses{-Erg}}\newcommand{\DEFSgAbs}{\strutgb{0pt}\grapho{Hk}}\newcommand{\DEFSgAbsP}{\textipa{bok}}\newcommand{\DEFSgAbsG}{DEF\cacherGloses{-Sg}\cacherGloses{-Abs}}\newcommand{\DEFSgObl}{\strutgb{0pt}\grapho{Hk}}\newcommand{\DEFSgOblP}{\textipa{bog}}\newcommand{\DEFSgOblG}{DEF\cacherGloses{-Sg}\cacherGloses{-Obl}}\newcommand{\DEFSgDat}{\strutgb{0pt}\grapho{Ht}}\newcommand{\DEFSgDatP}{\textipa{bot}}\newcommand{\DEFSgDatG}{DEF\cacherGloses{-Sg}\cacherGloses{-Dat}}\newcommand{\DEFDuErg}{\strutgb{0pt}\grapho{pm}}\newcommand{\DEFDuErgP}{\textipa{bam}}\newcommand{\DEFDuErgG}{DEF\cacherGloses{-Du}\cacherGloses{-Erg}}\newcommand{\DEFDuAbs}{\strutgb{0pt}\grapho{pk}}\newcommand{\DEFDuAbsP}{\textipa{bak}}\newcommand{\DEFDuAbsG}{DEF\cacherGloses{-Du}\cacherGloses{-Abs}}\newcommand{\DEFDuObl}{\strutgb{0pt}\grapho{pk}}\newcommand{\DEFDuOblP}{\textipa{bag}}\newcommand{\DEFDuOblG}{DEF\cacherGloses{-Du}\cacherGloses{-Obl}}\newcommand{\DEFDuDat}{\strutgb{0pt}\grapho{pt}}\newcommand{\DEFDuDatP}{\textipa{bat}}\newcommand{\DEFDuDatG}{DEF\cacherGloses{-Du}\cacherGloses{-Dat}}\newcommand{\DEFPlErg}{\strutgb{0pt}\grapho{Gm}}\newcommand{\DEFPlErgP}{\textipa{bim}}\newcommand{\DEFPlErgG}{DEF\cacherGloses{-Pl}\cacherGloses{-Erg}}\newcommand{\DEFPlAbs}{\strutgb{0pt}\grapho{Gk}}\newcommand{\DEFPlAbsP}{\textipa{bik}}\newcommand{\DEFPlAbsG}{DEF\cacherGloses{-Pl}\cacherGloses{-Abs}}\newcommand{\DEFPlObl}{\strutgb{0pt}\grapho{Gk}}\newcommand{\DEFPlOblP}{\textipa{big}}\newcommand{\DEFPlOblG}{DEF\cacherGloses{-Pl}\cacherGloses{-Obl}}\newcommand{\DEFPlDat}{\strutgb{0pt}\grapho{Gt}}\newcommand{\DEFPlDatP}{\textipa{bit}}\newcommand{\DEFPlDatG}{DEF\cacherGloses{-Pl}\cacherGloses{-Dat}}\newcommand{\DEMSgErg}{\strutgb{0pt}\grapho{Um}}\newcommand{\DEMSgErgP}{\textipa{lom}}\newcommand{\DEMSgErgG}{DEM\cacherGloses{-Sg}\cacherGloses{-Erg}}\newcommand{\DEMSgAbs}{\strutgb{0pt}\grapho{Uk}}\newcommand{\DEMSgAbsP}{\textipa{lok}}\newcommand{\DEMSgAbsG}{DEM\cacherGloses{-Sg}\cacherGloses{-Abs}}\newcommand{\DEMSgObl}{\strutgb{0pt}\grapho{Uk}}\newcommand{\DEMSgOblP}{\textipa{log}}\newcommand{\DEMSgOblG}{DEM\cacherGloses{-Sg}\cacherGloses{-Obl}}\newcommand{\DEMSgDat}{\strutgb{0pt}\grapho{Ut}}\newcommand{\DEMSgDatP}{\textipa{lot}}\newcommand{\DEMSgDatG}{DEM\cacherGloses{-Sg}\cacherGloses{-Dat}}\newcommand{\DEMDuErg}{\strutgb{0pt}\grapho{rm}}\newcommand{\DEMDuErgP}{\textipa{lam}}\newcommand{\DEMDuErgG}{DEM\cacherGloses{-Du}\cacherGloses{-Erg}}\newcommand{\DEMDuAbs}{\strutgb{0pt}\grapho{rk}}\newcommand{\DEMDuAbsP}{\textipa{lak}}\newcommand{\DEMDuAbsG}{DEM\cacherGloses{-Du}\cacherGloses{-Abs}}\newcommand{\DEMDuObl}{\strutgb{0pt}\grapho{rk}}\newcommand{\DEMDuOblP}{\textipa{lag}}\newcommand{\DEMDuOblG}{DEM\cacherGloses{-Du}\cacherGloses{-Obl}}\newcommand{\DEMDuDat}{\strutgb{0pt}\grapho{rt}}\newcommand{\DEMDuDatP}{\textipa{lat}}\newcommand{\DEMDuDatG}{DEM\cacherGloses{-Du}\cacherGloses{-Dat}}\newcommand{\DEMPlErg}{\strutgb{0pt}\grapho{Om}}\newcommand{\DEMPlErgP}{\textipa{lim}}\newcommand{\DEMPlErgG}{DEM\cacherGloses{-Pl}\cacherGloses{-Erg}}\newcommand{\DEMPlAbs}{\strutgb{0pt}\grapho{Ok}}\newcommand{\DEMPlAbsP}{\textipa{lik}}\newcommand{\DEMPlAbsG}{DEM\cacherGloses{-Pl}\cacherGloses{-Abs}}\newcommand{\DEMPlObl}{\strutgb{0pt}\grapho{Ok}}\newcommand{\DEMPlOblP}{\textipa{lig}}\newcommand{\DEMPlOblG}{DEM\cacherGloses{-Pl}\cacherGloses{-Obl}}\newcommand{\DEMPlDat}{\strutgb{0pt}\grapho{Ot}}\newcommand{\DEMPlDatP}{\textipa{lit}}\newcommand{\DEMPlDatG}{DEM\cacherGloses{-Pl}\cacherGloses{-Dat}}\newcommand{\POUR}{\strutgb{0pt}\grapho{rn}}\newcommand{\POURP}{\textipa{laN}}\newcommand{\POURG}{pour}\newcommand{\DEVANT}{\strutgb{0pt}\grapho{xr}}\newcommand{\DEVANTP}{\textipa{dul}}\newcommand{\DEVANTG}{devant}\newcommand{\SUR}{\strutgb{0pt}\grapho{rp}}\newcommand{\SURP}{\textipa{lab}}\newcommand{\SURG}{sur}\newcommand{\DANS}{\strutgb{0pt}\grapho{kr}}\newcommand{\DANSP}{\textipa{kal}}\newcommand{\DANSG}{dans}\newcommand{\SOUS}{\strutgb{0pt}\grapho{jn}}\newcommand{\SOUSP}{\textipa{jan}}\newcommand{\SOUSG}{sous}\newcommand{\A}{\strutgb{0pt}\grapho{Sd}}\newcommand{\AP}{\textipa{sed}}\newcommand{\AG}{à}\newcommand{\AVEC}{\strutgb{0pt}\grapho{pk}}\newcommand{\AVECP}{\textipa{bak}}\newcommand{\AVECG}{avec}\newcommand{\DE}{\strutgb{0pt}\grapho{Yr}}\newcommand{\DEP}{\textipa{sil}}\newcommand{\DEG}{de}\newcommand{\tomberViPrsASg}{\strutgb{0pt}\grapho{pOnyqu}}\newcommand{\tomberViPrsASgP}{\textipa{parinamifu}}\newcommand{\tomberViPrsASgG}{\cacherGloses{Sg-}tomber\cacherGloses{.VI}\cacherGloses{xPRS.A}}\newcommand{\tomberViPrsADu}{\strutgb{0pt}\grapho{sOnyqu}}\newcommand{\tomberViPrsADuP}{\textipa{sarinamifu}}\newcommand{\tomberViPrsADuG}{\cacherGloses{Du-}tomber\cacherGloses{.VI}\cacherGloses{xPRS.A}}\newcommand{\tomberViPrsAPl}{\strutgb{0pt}\grapho{dOnyqu}}\newcommand{\tomberViPrsAPlP}{\textipa{darinamifu}}\newcommand{\tomberViPrsAPlG}{\cacherGloses{Pl-}tomber\cacherGloses{.VI}\cacherGloses{xPRS.A}}\newcommand{\tomberViPrsBSg}{\strutgb{0pt}\grapho{prnyqu}}\newcommand{\tomberViPrsBSgP}{\textipa{paranamifu}}\newcommand{\tomberViPrsBSgG}{\cacherGloses{Sg-}tomber\cacherGloses{.VI}\cacherGloses{xPRS.B}}\newcommand{\tomberViPrsBDu}{\strutgb{0pt}\grapho{srnyqu}}\newcommand{\tomberViPrsBDuP}{\textipa{saranamifu}}\newcommand{\tomberViPrsBDuG}{\cacherGloses{Du-}tomber\cacherGloses{.VI}\cacherGloses{xPRS.B}}\newcommand{\tomberViPrsBPl}{\strutgb{0pt}\grapho{drnyqu}}\newcommand{\tomberViPrsBPlP}{\textipa{daranamifu}}\newcommand{\tomberViPrsBPlG}{\cacherGloses{Pl-}tomber\cacherGloses{.VI}\cacherGloses{xPRS.B}}\newcommand{\tomberViPrsCSg}{\strutgb{0pt}\grapho{pUnyqu}}\newcommand{\tomberViPrsCSgP}{\textipa{paronamifu}}\newcommand{\tomberViPrsCSgG}{\cacherGloses{Sg-}tomber\cacherGloses{.VI}\cacherGloses{xPRS.C}}\newcommand{\tomberViPrsCDu}{\strutgb{0pt}\grapho{sUnyqu}}\newcommand{\tomberViPrsCDuP}{\textipa{saronamifu}}\newcommand{\tomberViPrsCDuG}{\cacherGloses{Du-}tomber\cacherGloses{.VI}\cacherGloses{xPRS.C}}\newcommand{\tomberViPrsCPl}{\strutgb{0pt}\grapho{dUnyqu}}\newcommand{\tomberViPrsCPlP}{\textipa{daronamifu}}\newcommand{\tomberViPrsCPlG}{\cacherGloses{Pl-}tomber\cacherGloses{.VI}\cacherGloses{xPRS.C}}\newcommand{\tomberViPrsDSg}{\strutgb{0pt}\grapho{pRnyqu}}\newcommand{\tomberViPrsDSgP}{\textipa{parenamifu}}\newcommand{\tomberViPrsDSgG}{\cacherGloses{Sg-}tomber\cacherGloses{.VI}\cacherGloses{xPRS.D}}\newcommand{\tomberViPrsDDu}{\strutgb{0pt}\grapho{sRnyqu}}\newcommand{\tomberViPrsDDuP}{\textipa{sarenamifu}}\newcommand{\tomberViPrsDDuG}{\cacherGloses{Du-}tomber\cacherGloses{.VI}\cacherGloses{xPRS.D}}\newcommand{\tomberViPrsDPl}{\strutgb{0pt}\grapho{dRnyqu}}\newcommand{\tomberViPrsDPlP}{\textipa{darenamifu}}\newcommand{\tomberViPrsDPlG}{\cacherGloses{Pl-}tomber\cacherGloses{.VI}\cacherGloses{xPRS.D}}\newcommand{\tomberViPstASg}{\strutgb{0pt}\grapho{GwUnX}}\newcommand{\tomberViPstASgP}{\textipa{piwaronafi}}\newcommand{\tomberViPstASgG}{\cacherGloses{Sg-}tomber\cacherGloses{.VI}\cacherGloses{xPST.A}}\newcommand{\tomberViPstADu}{\strutgb{0pt}\grapho{YwUnX}}\newcommand{\tomberViPstADuP}{\textipa{siwaronafi}}\newcommand{\tomberViPstADuG}{\cacherGloses{Du-}tomber\cacherGloses{.VI}\cacherGloses{xPST.A}}\newcommand{\tomberViPstAPl}{\strutgb{0pt}\grapho{fwUnX}}\newcommand{\tomberViPstAPlP}{\textipa{diwaronafi}}\newcommand{\tomberViPstAPlG}{\cacherGloses{Pl-}tomber\cacherGloses{.VI}\cacherGloses{xPST.A}}\newcommand{\tomberViPstBSg}{\strutgb{0pt}\grapho{pwUFX}}\newcommand{\tomberViPstBSgP}{\textipa{pawaronufi}}\newcommand{\tomberViPstBSgG}{\cacherGloses{Sg-}tomber\cacherGloses{.VI}\cacherGloses{xPST.B}}\newcommand{\tomberViPstBDu}{\strutgb{0pt}\grapho{swUFX}}\newcommand{\tomberViPstBDuP}{\textipa{sawaronufi}}\newcommand{\tomberViPstBDuG}{\cacherGloses{Du-}tomber\cacherGloses{.VI}\cacherGloses{xPST.B}}\newcommand{\tomberViPstBPl}{\strutgb{0pt}\grapho{dwUFX}}\newcommand{\tomberViPstBPlP}{\textipa{dawaronufi}}\newcommand{\tomberViPstBPlG}{\cacherGloses{Pl-}tomber\cacherGloses{.VI}\cacherGloses{xPST.B}}\newcommand{\tomberViPstCSg}{\strutgb{0pt}\grapho{HwUEX}}\newcommand{\tomberViPstCSgP}{\textipa{powaronofi}}\newcommand{\tomberViPstCSgG}{\cacherGloses{Sg-}tomber\cacherGloses{.VI}\cacherGloses{xPST.C}}\newcommand{\tomberViPstCDu}{\strutgb{0pt}\grapho{1wUEX}}\newcommand{\tomberViPstCDuP}{\textipa{sowaronofi}}\newcommand{\tomberViPstCDuG}{\cacherGloses{Du-}tomber\cacherGloses{.VI}\cacherGloses{xPST.C}}\newcommand{\tomberViPstCPl}{\strutgb{0pt}\grapho{gwUEX}}\newcommand{\tomberViPstCPlP}{\textipa{dowaronofi}}\newcommand{\tomberViPstCPlG}{\cacherGloses{Pl-}tomber\cacherGloses{.VI}\cacherGloses{xPST.C}}\newcommand{\tomberViPstDSg}{\strutgb{0pt}\grapho{PwUEX}}\newcommand{\tomberViPstDSgP}{\textipa{pewaronofi}}\newcommand{\tomberViPstDSgG}{\cacherGloses{Sg-}tomber\cacherGloses{.VI}\cacherGloses{xPST.D}}\newcommand{\tomberViPstDDu}{\strutgb{0pt}\grapho{SwUEX}}\newcommand{\tomberViPstDDuP}{\textipa{sewaronofi}}\newcommand{\tomberViPstDDuG}{\cacherGloses{Du-}tomber\cacherGloses{.VI}\cacherGloses{xPST.D}}\newcommand{\tomberViPstDPl}{\strutgb{0pt}\grapho{DwUEX}}\newcommand{\tomberViPstDPlP}{\textipa{dewaronofi}}\newcommand{\tomberViPstDPlG}{\cacherGloses{Pl-}tomber\cacherGloses{.VI}\cacherGloses{xPST.D}}\newcommand{\entrerViPrsASg}{\strutgb{0pt}\grapho{pGtGV}}\newcommand{\entrerViPrsASgP}{\textipa{pabitapiru}}\newcommand{\entrerViPrsASgG}{\cacherGloses{Sg-}entrer\cacherGloses{.VI}\cacherGloses{xPRS.A}}\newcommand{\entrerViPrsADu}{\strutgb{0pt}\grapho{sGtGV}}\newcommand{\entrerViPrsADuP}{\textipa{sabitapiru}}\newcommand{\entrerViPrsADuG}{\cacherGloses{Du-}entrer\cacherGloses{.VI}\cacherGloses{xPRS.A}}\newcommand{\entrerViPrsAPl}{\strutgb{0pt}\grapho{dGtGV}}\newcommand{\entrerViPrsAPlP}{\textipa{dabitapiru}}\newcommand{\entrerViPrsAPlG}{\cacherGloses{Pl-}entrer\cacherGloses{.VI}\cacherGloses{xPRS.A}}\newcommand{\entrerViPrsBSg}{\strutgb{0pt}\grapho{pptGV}}\newcommand{\entrerViPrsBSgP}{\textipa{pabatapiru}}\newcommand{\entrerViPrsBSgG}{\cacherGloses{Sg-}entrer\cacherGloses{.VI}\cacherGloses{xPRS.B}}\newcommand{\entrerViPrsBDu}{\strutgb{0pt}\grapho{sptGV}}\newcommand{\entrerViPrsBDuP}{\textipa{sabatapiru}}\newcommand{\entrerViPrsBDuG}{\cacherGloses{Du-}entrer\cacherGloses{.VI}\cacherGloses{xPRS.B}}\newcommand{\entrerViPrsBPl}{\strutgb{0pt}\grapho{dptGV}}\newcommand{\entrerViPrsBPlP}{\textipa{dabatapiru}}\newcommand{\entrerViPrsBPlG}{\cacherGloses{Pl-}entrer\cacherGloses{.VI}\cacherGloses{xPRS.B}}\newcommand{\entrerViPrsCSg}{\strutgb{0pt}\grapho{pHtGV}}\newcommand{\entrerViPrsCSgP}{\textipa{pabotapiru}}\newcommand{\entrerViPrsCSgG}{\cacherGloses{Sg-}entrer\cacherGloses{.VI}\cacherGloses{xPRS.C}}\newcommand{\entrerViPrsCDu}{\strutgb{0pt}\grapho{sHtGV}}\newcommand{\entrerViPrsCDuP}{\textipa{sabotapiru}}\newcommand{\entrerViPrsCDuG}{\cacherGloses{Du-}entrer\cacherGloses{.VI}\cacherGloses{xPRS.C}}\newcommand{\entrerViPrsCPl}{\strutgb{0pt}\grapho{dHtGV}}\newcommand{\entrerViPrsCPlP}{\textipa{dabotapiru}}\newcommand{\entrerViPrsCPlG}{\cacherGloses{Pl-}entrer\cacherGloses{.VI}\cacherGloses{xPRS.C}}\newcommand{\entrerViPrsDSg}{\strutgb{0pt}\grapho{pPtGV}}\newcommand{\entrerViPrsDSgP}{\textipa{pabetapiru}}\newcommand{\entrerViPrsDSgG}{\cacherGloses{Sg-}entrer\cacherGloses{.VI}\cacherGloses{xPRS.D}}\newcommand{\entrerViPrsDDu}{\strutgb{0pt}\grapho{sPtGV}}\newcommand{\entrerViPrsDDuP}{\textipa{sabetapiru}}\newcommand{\entrerViPrsDDuG}{\cacherGloses{Du-}entrer\cacherGloses{.VI}\cacherGloses{xPRS.D}}\newcommand{\entrerViPrsDPl}{\strutgb{0pt}\grapho{dPtGV}}\newcommand{\entrerViPrsDPlP}{\textipa{dabetapiru}}\newcommand{\entrerViPrsDPlG}{\cacherGloses{Pl-}entrer\cacherGloses{.VI}\cacherGloses{xPRS.D}}\newcommand{\entrerViPstASg}{\strutgb{0pt}\grapho{GpHtO}}\newcommand{\entrerViPstASgP}{\textipa{pipabotari}}\newcommand{\entrerViPstASgG}{\cacherGloses{Sg-}entrer\cacherGloses{.VI}\cacherGloses{xPST.A}}\newcommand{\entrerViPstADu}{\strutgb{0pt}\grapho{YpHtO}}\newcommand{\entrerViPstADuP}{\textipa{sipabotari}}\newcommand{\entrerViPstADuG}{\cacherGloses{Du-}entrer\cacherGloses{.VI}\cacherGloses{xPST.A}}\newcommand{\entrerViPstAPl}{\strutgb{0pt}\grapho{fpHtO}}\newcommand{\entrerViPstAPlP}{\textipa{dipabotari}}\newcommand{\entrerViPstAPlG}{\cacherGloses{Pl-}entrer\cacherGloses{.VI}\cacherGloses{xPST.A}}\newcommand{\entrerViPstBSg}{\strutgb{0pt}\grapho{ppH5O}}\newcommand{\entrerViPstBSgP}{\textipa{papaboturi}}\newcommand{\entrerViPstBSgG}{\cacherGloses{Sg-}entrer\cacherGloses{.VI}\cacherGloses{xPST.B}}\newcommand{\entrerViPstBDu}{\strutgb{0pt}\grapho{spH5O}}\newcommand{\entrerViPstBDuP}{\textipa{sapaboturi}}\newcommand{\entrerViPstBDuG}{\cacherGloses{Du-}entrer\cacherGloses{.VI}\cacherGloses{xPST.B}}\newcommand{\entrerViPstBPl}{\strutgb{0pt}\grapho{dpH5O}}\newcommand{\entrerViPstBPlP}{\textipa{dapaboturi}}\newcommand{\entrerViPstBPlG}{\cacherGloses{Pl-}entrer\cacherGloses{.VI}\cacherGloses{xPST.B}}\newcommand{\entrerViPstCSg}{\strutgb{0pt}\grapho{HpH4O}}\newcommand{\entrerViPstCSgP}{\textipa{popabotori}}\newcommand{\entrerViPstCSgG}{\cacherGloses{Sg-}entrer\cacherGloses{.VI}\cacherGloses{xPST.C}}\newcommand{\entrerViPstCDu}{\strutgb{0pt}\grapho{1pH4O}}\newcommand{\entrerViPstCDuP}{\textipa{sopabotori}}\newcommand{\entrerViPstCDuG}{\cacherGloses{Du-}entrer\cacherGloses{.VI}\cacherGloses{xPST.C}}\newcommand{\entrerViPstCPl}{\strutgb{0pt}\grapho{gpH4O}}\newcommand{\entrerViPstCPlP}{\textipa{dopabotori}}\newcommand{\entrerViPstCPlG}{\cacherGloses{Pl-}entrer\cacherGloses{.VI}\cacherGloses{xPST.C}}\newcommand{\entrerViPstDSg}{\strutgb{0pt}\grapho{PpH4O}}\newcommand{\entrerViPstDSgP}{\textipa{pepabotori}}\newcommand{\entrerViPstDSgG}{\cacherGloses{Sg-}entrer\cacherGloses{.VI}\cacherGloses{xPST.D}}\newcommand{\entrerViPstDDu}{\strutgb{0pt}\grapho{SpH4O}}\newcommand{\entrerViPstDDuP}{\textipa{sepabotori}}\newcommand{\entrerViPstDDuG}{\cacherGloses{Du-}entrer\cacherGloses{.VI}\cacherGloses{xPST.D}}\newcommand{\entrerViPstDPl}{\strutgb{0pt}\grapho{DpH4O}}\newcommand{\entrerViPstDPlP}{\textipa{depabotori}}\newcommand{\entrerViPstDPlG}{\cacherGloses{Pl-}entrer\cacherGloses{.VI}\cacherGloses{xPST.D}}\newcommand{\arriverViPrsASg}{\strutgb{0pt}\grapho{pYqX5}}\newcommand{\arriverViPrsASgP}{\textipa{pasivafitu}}\newcommand{\arriverViPrsASgG}{\cacherGloses{Sg-}arriver\cacherGloses{.VI}\cacherGloses{xPRS.A}}\newcommand{\arriverViPrsADu}{\strutgb{0pt}\grapho{sYqX5}}\newcommand{\arriverViPrsADuP}{\textipa{sasivafitu}}\newcommand{\arriverViPrsADuG}{\cacherGloses{Du-}arriver\cacherGloses{.VI}\cacherGloses{xPRS.A}}\newcommand{\arriverViPrsAPl}{\strutgb{0pt}\grapho{dYqX5}}\newcommand{\arriverViPrsAPlP}{\textipa{dasivafitu}}\newcommand{\arriverViPrsAPlG}{\cacherGloses{Pl-}arriver\cacherGloses{.VI}\cacherGloses{xPRS.A}}\newcommand{\arriverViPrsBSg}{\strutgb{0pt}\grapho{psqX5}}\newcommand{\arriverViPrsBSgP}{\textipa{pasavafitu}}\newcommand{\arriverViPrsBSgG}{\cacherGloses{Sg-}arriver\cacherGloses{.VI}\cacherGloses{xPRS.B}}\newcommand{\arriverViPrsBDu}{\strutgb{0pt}\grapho{ssqX5}}\newcommand{\arriverViPrsBDuP}{\textipa{sasavafitu}}\newcommand{\arriverViPrsBDuG}{\cacherGloses{Du-}arriver\cacherGloses{.VI}\cacherGloses{xPRS.B}}\newcommand{\arriverViPrsBPl}{\strutgb{0pt}\grapho{dsqX5}}\newcommand{\arriverViPrsBPlP}{\textipa{dasavafitu}}\newcommand{\arriverViPrsBPlG}{\cacherGloses{Pl-}arriver\cacherGloses{.VI}\cacherGloses{xPRS.B}}\newcommand{\arriverViPrsCSg}{\strutgb{0pt}\grapho{p1qX5}}\newcommand{\arriverViPrsCSgP}{\textipa{pasovafitu}}\newcommand{\arriverViPrsCSgG}{\cacherGloses{Sg-}arriver\cacherGloses{.VI}\cacherGloses{xPRS.C}}\newcommand{\arriverViPrsCDu}{\strutgb{0pt}\grapho{s1qX5}}\newcommand{\arriverViPrsCDuP}{\textipa{sasovafitu}}\newcommand{\arriverViPrsCDuG}{\cacherGloses{Du-}arriver\cacherGloses{.VI}\cacherGloses{xPRS.C}}\newcommand{\arriverViPrsCPl}{\strutgb{0pt}\grapho{d1qX5}}\newcommand{\arriverViPrsCPlP}{\textipa{dasovafitu}}\newcommand{\arriverViPrsCPlG}{\cacherGloses{Pl-}arriver\cacherGloses{.VI}\cacherGloses{xPRS.C}}\newcommand{\arriverViPrsDSg}{\strutgb{0pt}\grapho{pSqX5}}\newcommand{\arriverViPrsDSgP}{\textipa{pasevafitu}}\newcommand{\arriverViPrsDSgG}{\cacherGloses{Sg-}arriver\cacherGloses{.VI}\cacherGloses{xPRS.D}}\newcommand{\arriverViPrsDDu}{\strutgb{0pt}\grapho{sSqX5}}\newcommand{\arriverViPrsDDuP}{\textipa{sasevafitu}}\newcommand{\arriverViPrsDDuG}{\cacherGloses{Du-}arriver\cacherGloses{.VI}\cacherGloses{xPRS.D}}\newcommand{\arriverViPrsDPl}{\strutgb{0pt}\grapho{dSqX5}}\newcommand{\arriverViPrsDPlP}{\textipa{dasevafitu}}\newcommand{\arriverViPrsDPlG}{\cacherGloses{Pl-}arriver\cacherGloses{.VI}\cacherGloses{xPRS.D}}\newcommand{\arriverViPstASg}{\strutgb{0pt}\grapho{Gq1q3}}\newcommand{\arriverViPstASgP}{\textipa{pifasovati}}\newcommand{\arriverViPstASgG}{\cacherGloses{Sg-}arriver\cacherGloses{.VI}\cacherGloses{xPST.A}}\newcommand{\arriverViPstADu}{\strutgb{0pt}\grapho{Yq1q3}}\newcommand{\arriverViPstADuP}{\textipa{sifasovati}}\newcommand{\arriverViPstADuG}{\cacherGloses{Du-}arriver\cacherGloses{.VI}\cacherGloses{xPST.A}}\newcommand{\arriverViPstAPl}{\strutgb{0pt}\grapho{fq1q3}}\newcommand{\arriverViPstAPlP}{\textipa{difasovati}}\newcommand{\arriverViPstAPlG}{\cacherGloses{Pl-}arriver\cacherGloses{.VI}\cacherGloses{xPST.A}}\newcommand{\arriverViPstBSg}{\strutgb{0pt}\grapho{pq1qu3}}\newcommand{\arriverViPstBSgP}{\textipa{pafasovuti}}\newcommand{\arriverViPstBSgG}{\cacherGloses{Sg-}arriver\cacherGloses{.VI}\cacherGloses{xPST.B}}\newcommand{\arriverViPstBDu}{\strutgb{0pt}\grapho{sq1qu3}}\newcommand{\arriverViPstBDuP}{\textipa{safasovuti}}\newcommand{\arriverViPstBDuG}{\cacherGloses{Du-}arriver\cacherGloses{.VI}\cacherGloses{xPST.B}}\newcommand{\arriverViPstBPl}{\strutgb{0pt}\grapho{dq1qu3}}\newcommand{\arriverViPstBPlP}{\textipa{dafasovuti}}\newcommand{\arriverViPstBPlG}{\cacherGloses{Pl-}arriver\cacherGloses{.VI}\cacherGloses{xPST.B}}\newcommand{\arriverViPstCSg}{\strutgb{0pt}\grapho{Hq183}}\newcommand{\arriverViPstCSgP}{\textipa{pofasovoti}}\newcommand{\arriverViPstCSgG}{\cacherGloses{Sg-}arriver\cacherGloses{.VI}\cacherGloses{xPST.C}}\newcommand{\arriverViPstCDu}{\strutgb{0pt}\grapho{1q183}}\newcommand{\arriverViPstCDuP}{\textipa{sofasovoti}}\newcommand{\arriverViPstCDuG}{\cacherGloses{Du-}arriver\cacherGloses{.VI}\cacherGloses{xPST.C}}\newcommand{\arriverViPstCPl}{\strutgb{0pt}\grapho{gq183}}\newcommand{\arriverViPstCPlP}{\textipa{dofasovoti}}\newcommand{\arriverViPstCPlG}{\cacherGloses{Pl-}arriver\cacherGloses{.VI}\cacherGloses{xPST.C}}\newcommand{\arriverViPstDSg}{\strutgb{0pt}\grapho{Pq183}}\newcommand{\arriverViPstDSgP}{\textipa{pefasovoti}}\newcommand{\arriverViPstDSgG}{\cacherGloses{Sg-}arriver\cacherGloses{.VI}\cacherGloses{xPST.D}}\newcommand{\arriverViPstDDu}{\strutgb{0pt}\grapho{Sq183}}\newcommand{\arriverViPstDDuP}{\textipa{sefasovoti}}\newcommand{\arriverViPstDDuG}{\cacherGloses{Du-}arriver\cacherGloses{.VI}\cacherGloses{xPST.D}}\newcommand{\arriverViPstDPl}{\strutgb{0pt}\grapho{Dq183}}\newcommand{\arriverViPstDPlP}{\textipa{defasovoti}}\newcommand{\arriverViPstDPlG}{\cacherGloses{Pl-}arriver\cacherGloses{.VI}\cacherGloses{xPST.D}}\newcommand{\dormirViPrsASg}{\strutgb{0pt}\grapho{pGr6F}}\newcommand{\dormirViPrsASgP}{\textipa{pabirawiNu}}\newcommand{\dormirViPrsASgG}{\cacherGloses{Sg-}dormir\cacherGloses{.VI}\cacherGloses{xPRS.A}}\newcommand{\dormirViPrsADu}{\strutgb{0pt}\grapho{sGr6F}}\newcommand{\dormirViPrsADuP}{\textipa{sabirawiNu}}\newcommand{\dormirViPrsADuG}{\cacherGloses{Du-}dormir\cacherGloses{.VI}\cacherGloses{xPRS.A}}\newcommand{\dormirViPrsAPl}{\strutgb{0pt}\grapho{dGr6F}}\newcommand{\dormirViPrsAPlP}{\textipa{dabirawiNu}}\newcommand{\dormirViPrsAPlG}{\cacherGloses{Pl-}dormir\cacherGloses{.VI}\cacherGloses{xPRS.A}}\newcommand{\dormirViPrsBSg}{\strutgb{0pt}\grapho{ppr6F}}\newcommand{\dormirViPrsBSgP}{\textipa{pabarawiNu}}\newcommand{\dormirViPrsBSgG}{\cacherGloses{Sg-}dormir\cacherGloses{.VI}\cacherGloses{xPRS.B}}\newcommand{\dormirViPrsBDu}{\strutgb{0pt}\grapho{spr6F}}\newcommand{\dormirViPrsBDuP}{\textipa{sabarawiNu}}\newcommand{\dormirViPrsBDuG}{\cacherGloses{Du-}dormir\cacherGloses{.VI}\cacherGloses{xPRS.B}}\newcommand{\dormirViPrsBPl}{\strutgb{0pt}\grapho{dpr6F}}\newcommand{\dormirViPrsBPlP}{\textipa{dabarawiNu}}\newcommand{\dormirViPrsBPlG}{\cacherGloses{Pl-}dormir\cacherGloses{.VI}\cacherGloses{xPRS.B}}\newcommand{\dormirViPrsCSg}{\strutgb{0pt}\grapho{pHr6F}}\newcommand{\dormirViPrsCSgP}{\textipa{paborawiNu}}\newcommand{\dormirViPrsCSgG}{\cacherGloses{Sg-}dormir\cacherGloses{.VI}\cacherGloses{xPRS.C}}\newcommand{\dormirViPrsCDu}{\strutgb{0pt}\grapho{sHr6F}}\newcommand{\dormirViPrsCDuP}{\textipa{saborawiNu}}\newcommand{\dormirViPrsCDuG}{\cacherGloses{Du-}dormir\cacherGloses{.VI}\cacherGloses{xPRS.C}}\newcommand{\dormirViPrsCPl}{\strutgb{0pt}\grapho{dHr6F}}\newcommand{\dormirViPrsCPlP}{\textipa{daborawiNu}}\newcommand{\dormirViPrsCPlG}{\cacherGloses{Pl-}dormir\cacherGloses{.VI}\cacherGloses{xPRS.C}}\newcommand{\dormirViPrsDSg}{\strutgb{0pt}\grapho{pPr6F}}\newcommand{\dormirViPrsDSgP}{\textipa{paberawiNu}}\newcommand{\dormirViPrsDSgG}{\cacherGloses{Sg-}dormir\cacherGloses{.VI}\cacherGloses{xPRS.D}}\newcommand{\dormirViPrsDDu}{\strutgb{0pt}\grapho{sPr6F}}\newcommand{\dormirViPrsDDuP}{\textipa{saberawiNu}}\newcommand{\dormirViPrsDDuG}{\cacherGloses{Du-}dormir\cacherGloses{.VI}\cacherGloses{xPRS.D}}\newcommand{\dormirViPrsDPl}{\strutgb{0pt}\grapho{dPr6F}}\newcommand{\dormirViPrsDPlP}{\textipa{daberawiNu}}\newcommand{\dormirViPrsDPlG}{\cacherGloses{Pl-}dormir\cacherGloses{.VI}\cacherGloses{xPRS.D}}\newcommand{\dormirViPstASg}{\strutgb{0pt}\grapho{GpHrC}}\newcommand{\dormirViPstASgP}{\textipa{pipaboraNi}}\newcommand{\dormirViPstASgG}{\cacherGloses{Sg-}dormir\cacherGloses{.VI}\cacherGloses{xPST.A}}\newcommand{\dormirViPstADu}{\strutgb{0pt}\grapho{YpHrC}}\newcommand{\dormirViPstADuP}{\textipa{sipaboraNi}}\newcommand{\dormirViPstADuG}{\cacherGloses{Du-}dormir\cacherGloses{.VI}\cacherGloses{xPST.A}}\newcommand{\dormirViPstAPl}{\strutgb{0pt}\grapho{fpHrC}}\newcommand{\dormirViPstAPlP}{\textipa{dipaboraNi}}\newcommand{\dormirViPstAPlG}{\cacherGloses{Pl-}dormir\cacherGloses{.VI}\cacherGloses{xPST.A}}\newcommand{\dormirViPstBSg}{\strutgb{0pt}\grapho{ppHVC}}\newcommand{\dormirViPstBSgP}{\textipa{papaboruNi}}\newcommand{\dormirViPstBSgG}{\cacherGloses{Sg-}dormir\cacherGloses{.VI}\cacherGloses{xPST.B}}\newcommand{\dormirViPstBDu}{\strutgb{0pt}\grapho{spHVC}}\newcommand{\dormirViPstBDuP}{\textipa{sapaboruNi}}\newcommand{\dormirViPstBDuG}{\cacherGloses{Du-}dormir\cacherGloses{.VI}\cacherGloses{xPST.B}}\newcommand{\dormirViPstBPl}{\strutgb{0pt}\grapho{dpHVC}}\newcommand{\dormirViPstBPlP}{\textipa{dapaboruNi}}\newcommand{\dormirViPstBPlG}{\cacherGloses{Pl-}dormir\cacherGloses{.VI}\cacherGloses{xPST.B}}\newcommand{\dormirViPstCSg}{\strutgb{0pt}\grapho{HpHUC}}\newcommand{\dormirViPstCSgP}{\textipa{popaboroNi}}\newcommand{\dormirViPstCSgG}{\cacherGloses{Sg-}dormir\cacherGloses{.VI}\cacherGloses{xPST.C}}\newcommand{\dormirViPstCDu}{\strutgb{0pt}\grapho{1pHUC}}\newcommand{\dormirViPstCDuP}{\textipa{sopaboroNi}}\newcommand{\dormirViPstCDuG}{\cacherGloses{Du-}dormir\cacherGloses{.VI}\cacherGloses{xPST.C}}\newcommand{\dormirViPstCPl}{\strutgb{0pt}\grapho{gpHUC}}\newcommand{\dormirViPstCPlP}{\textipa{dopaboroNi}}\newcommand{\dormirViPstCPlG}{\cacherGloses{Pl-}dormir\cacherGloses{.VI}\cacherGloses{xPST.C}}\newcommand{\dormirViPstDSg}{\strutgb{0pt}\grapho{PpHUC}}\newcommand{\dormirViPstDSgP}{\textipa{pepaboroNi}}\newcommand{\dormirViPstDSgG}{\cacherGloses{Sg-}dormir\cacherGloses{.VI}\cacherGloses{xPST.D}}\newcommand{\dormirViPstDDu}{\strutgb{0pt}\grapho{SpHUC}}\newcommand{\dormirViPstDDuP}{\textipa{sepaboroNi}}\newcommand{\dormirViPstDDuG}{\cacherGloses{Du-}dormir\cacherGloses{.VI}\cacherGloses{xPST.D}}\newcommand{\dormirViPstDPl}{\strutgb{0pt}\grapho{DpHUC}}\newcommand{\dormirViPstDPlP}{\textipa{depaboroNi}}\newcommand{\dormirViPstDPlG}{\cacherGloses{Pl-}dormir\cacherGloses{.VI}\cacherGloses{xPST.D}}\newcommand{\boireVtPrsASg}{\strutgb{0pt}\grapho{HYrWqu}}\newcommand{\boireVtPrsASgP}{\textipa{poSirawevu}}\newcommand{\boireVtPrsASgG}{\cacherGloses{Sg-}boire\cacherGloses{.VT}\cacherGloses{xPRS.A}}\newcommand{\boireVtPrsADu}{\strutgb{0pt}\grapho{1YrWqu}}\newcommand{\boireVtPrsADuP}{\textipa{soSirawevu}}\newcommand{\boireVtPrsADuG}{\cacherGloses{Du-}boire\cacherGloses{.VT}\cacherGloses{xPRS.A}}\newcommand{\boireVtPrsAPl}{\strutgb{0pt}\grapho{gYrWqu}}\newcommand{\boireVtPrsAPlP}{\textipa{doSirawevu}}\newcommand{\boireVtPrsAPlG}{\cacherGloses{Pl-}boire\cacherGloses{.VT}\cacherGloses{xPRS.A}}\newcommand{\boireVtPrsBSg}{\strutgb{0pt}\grapho{HsrWqu}}\newcommand{\boireVtPrsBSgP}{\textipa{poSarawevu}}\newcommand{\boireVtPrsBSgG}{\cacherGloses{Sg-}boire\cacherGloses{.VT}\cacherGloses{xPRS.B}}\newcommand{\boireVtPrsBDu}{\strutgb{0pt}\grapho{1srWqu}}\newcommand{\boireVtPrsBDuP}{\textipa{soSarawevu}}\newcommand{\boireVtPrsBDuG}{\cacherGloses{Du-}boire\cacherGloses{.VT}\cacherGloses{xPRS.B}}\newcommand{\boireVtPrsBPl}{\strutgb{0pt}\grapho{gsrWqu}}\newcommand{\boireVtPrsBPlP}{\textipa{doSarawevu}}\newcommand{\boireVtPrsBPlG}{\cacherGloses{Pl-}boire\cacherGloses{.VT}\cacherGloses{xPRS.B}}\newcommand{\boireVtPrsCSg}{\strutgb{0pt}\grapho{H1rWqu}}\newcommand{\boireVtPrsCSgP}{\textipa{poSorawevu}}\newcommand{\boireVtPrsCSgG}{\cacherGloses{Sg-}boire\cacherGloses{.VT}\cacherGloses{xPRS.C}}\newcommand{\boireVtPrsCDu}{\strutgb{0pt}\grapho{11rWqu}}\newcommand{\boireVtPrsCDuP}{\textipa{soSorawevu}}\newcommand{\boireVtPrsCDuG}{\cacherGloses{Du-}boire\cacherGloses{.VT}\cacherGloses{xPRS.C}}\newcommand{\boireVtPrsCPl}{\strutgb{0pt}\grapho{g1rWqu}}\newcommand{\boireVtPrsCPlP}{\textipa{doSorawevu}}\newcommand{\boireVtPrsCPlG}{\cacherGloses{Pl-}boire\cacherGloses{.VT}\cacherGloses{xPRS.C}}\newcommand{\boireVtPrsDSg}{\strutgb{0pt}\grapho{HSrWqu}}\newcommand{\boireVtPrsDSgP}{\textipa{poSerawevu}}\newcommand{\boireVtPrsDSgG}{\cacherGloses{Sg-}boire\cacherGloses{.VT}\cacherGloses{xPRS.D}}\newcommand{\boireVtPrsDDu}{\strutgb{0pt}\grapho{1SrWqu}}\newcommand{\boireVtPrsDDuP}{\textipa{soSerawevu}}\newcommand{\boireVtPrsDDuG}{\cacherGloses{Du-}boire\cacherGloses{.VT}\cacherGloses{xPRS.D}}\newcommand{\boireVtPrsDPl}{\strutgb{0pt}\grapho{gSrWqu}}\newcommand{\boireVtPrsDPlP}{\textipa{doSerawevu}}\newcommand{\boireVtPrsDPlG}{\cacherGloses{Pl-}boire\cacherGloses{.VT}\cacherGloses{xPRS.D}}\newcommand{\boireVtPstASg}{\strutgb{0pt}\grapho{G11rQ}}\newcommand{\boireVtPstASgP}{\textipa{pisoSorave}}\newcommand{\boireVtPstASgG}{\cacherGloses{Sg-}boire\cacherGloses{.VT}\cacherGloses{xPST.A}}\newcommand{\boireVtPstADu}{\strutgb{0pt}\grapho{Y11rQ}}\newcommand{\boireVtPstADuP}{\textipa{sisoSorave}}\newcommand{\boireVtPstADuG}{\cacherGloses{Du-}boire\cacherGloses{.VT}\cacherGloses{xPST.A}}\newcommand{\boireVtPstAPl}{\strutgb{0pt}\grapho{f11rQ}}\newcommand{\boireVtPstAPlP}{\textipa{disoSorave}}\newcommand{\boireVtPstAPlG}{\cacherGloses{Pl-}boire\cacherGloses{.VT}\cacherGloses{xPST.A}}\newcommand{\boireVtPstBSg}{\strutgb{0pt}\grapho{p11VQ}}\newcommand{\boireVtPstBSgP}{\textipa{pasoSoruve}}\newcommand{\boireVtPstBSgG}{\cacherGloses{Sg-}boire\cacherGloses{.VT}\cacherGloses{xPST.B}}\newcommand{\boireVtPstBDu}{\strutgb{0pt}\grapho{s11VQ}}\newcommand{\boireVtPstBDuP}{\textipa{sasoSoruve}}\newcommand{\boireVtPstBDuG}{\cacherGloses{Du-}boire\cacherGloses{.VT}\cacherGloses{xPST.B}}\newcommand{\boireVtPstBPl}{\strutgb{0pt}\grapho{d11VQ}}\newcommand{\boireVtPstBPlP}{\textipa{dasoSoruve}}\newcommand{\boireVtPstBPlG}{\cacherGloses{Pl-}boire\cacherGloses{.VT}\cacherGloses{xPST.B}}\newcommand{\boireVtPstCSg}{\strutgb{0pt}\grapho{H11UQ}}\newcommand{\boireVtPstCSgP}{\textipa{posoSorove}}\newcommand{\boireVtPstCSgG}{\cacherGloses{Sg-}boire\cacherGloses{.VT}\cacherGloses{xPST.C}}\newcommand{\boireVtPstCDu}{\strutgb{0pt}\grapho{111UQ}}\newcommand{\boireVtPstCDuP}{\textipa{sosoSorove}}\newcommand{\boireVtPstCDuG}{\cacherGloses{Du-}boire\cacherGloses{.VT}\cacherGloses{xPST.C}}\newcommand{\boireVtPstCPl}{\strutgb{0pt}\grapho{g11UQ}}\newcommand{\boireVtPstCPlP}{\textipa{dosoSorove}}\newcommand{\boireVtPstCPlG}{\cacherGloses{Pl-}boire\cacherGloses{.VT}\cacherGloses{xPST.C}}\newcommand{\boireVtPstDSg}{\strutgb{0pt}\grapho{P11UQ}}\newcommand{\boireVtPstDSgP}{\textipa{pesoSorove}}\newcommand{\boireVtPstDSgG}{\cacherGloses{Sg-}boire\cacherGloses{.VT}\cacherGloses{xPST.D}}\newcommand{\boireVtPstDDu}{\strutgb{0pt}\grapho{S11UQ}}\newcommand{\boireVtPstDDuP}{\textipa{sesoSorove}}\newcommand{\boireVtPstDDuG}{\cacherGloses{Du-}boire\cacherGloses{.VT}\cacherGloses{xPST.D}}\newcommand{\boireVtPstDPl}{\strutgb{0pt}\grapho{D11UQ}}\newcommand{\boireVtPstDPlP}{\textipa{desoSorove}}\newcommand{\boireVtPstDPlG}{\cacherGloses{Pl-}boire\cacherGloses{.VT}\cacherGloses{xPST.D}}\newcommand{\supporterVtPrsASg}{\strutgb{0pt}\grapho{HCkTB}}\newcommand{\supporterVtPrsASgP}{\textipa{ponikatemu}}\newcommand{\supporterVtPrsASgG}{\cacherGloses{Sg-}supporter\cacherGloses{.VT}\cacherGloses{xPRS.A}}\newcommand{\supporterVtPrsADu}{\strutgb{0pt}\grapho{1CkTB}}\newcommand{\supporterVtPrsADuP}{\textipa{sonikatemu}}\newcommand{\supporterVtPrsADuG}{\cacherGloses{Du-}supporter\cacherGloses{.VT}\cacherGloses{xPRS.A}}\newcommand{\supporterVtPrsAPl}{\strutgb{0pt}\grapho{gCkTB}}\newcommand{\supporterVtPrsAPlP}{\textipa{donikatemu}}\newcommand{\supporterVtPrsAPlG}{\cacherGloses{Pl-}supporter\cacherGloses{.VT}\cacherGloses{xPRS.A}}\newcommand{\supporterVtPrsBSg}{\strutgb{0pt}\grapho{HnkTB}}\newcommand{\supporterVtPrsBSgP}{\textipa{ponakatemu}}\newcommand{\supporterVtPrsBSgG}{\cacherGloses{Sg-}supporter\cacherGloses{.VT}\cacherGloses{xPRS.B}}\newcommand{\supporterVtPrsBDu}{\strutgb{0pt}\grapho{1nkTB}}\newcommand{\supporterVtPrsBDuP}{\textipa{sonakatemu}}\newcommand{\supporterVtPrsBDuG}{\cacherGloses{Du-}supporter\cacherGloses{.VT}\cacherGloses{xPRS.B}}\newcommand{\supporterVtPrsBPl}{\strutgb{0pt}\grapho{gnkTB}}\newcommand{\supporterVtPrsBPlP}{\textipa{donakatemu}}\newcommand{\supporterVtPrsBPlG}{\cacherGloses{Pl-}supporter\cacherGloses{.VT}\cacherGloses{xPRS.B}}\newcommand{\supporterVtPrsCSg}{\strutgb{0pt}\grapho{HEkTB}}\newcommand{\supporterVtPrsCSgP}{\textipa{ponokatemu}}\newcommand{\supporterVtPrsCSgG}{\cacherGloses{Sg-}supporter\cacherGloses{.VT}\cacherGloses{xPRS.C}}\newcommand{\supporterVtPrsCDu}{\strutgb{0pt}\grapho{1EkTB}}\newcommand{\supporterVtPrsCDuP}{\textipa{sonokatemu}}\newcommand{\supporterVtPrsCDuG}{\cacherGloses{Du-}supporter\cacherGloses{.VT}\cacherGloses{xPRS.C}}\newcommand{\supporterVtPrsCPl}{\strutgb{0pt}\grapho{gEkTB}}\newcommand{\supporterVtPrsCPlP}{\textipa{donokatemu}}\newcommand{\supporterVtPrsCPlG}{\cacherGloses{Pl-}supporter\cacherGloses{.VT}\cacherGloses{xPRS.C}}\newcommand{\supporterVtPrsDSg}{\strutgb{0pt}\grapho{HNkTB}}\newcommand{\supporterVtPrsDSgP}{\textipa{ponekatemu}}\newcommand{\supporterVtPrsDSgG}{\cacherGloses{Sg-}supporter\cacherGloses{.VT}\cacherGloses{xPRS.D}}\newcommand{\supporterVtPrsDDu}{\strutgb{0pt}\grapho{1NkTB}}\newcommand{\supporterVtPrsDDuP}{\textipa{sonekatemu}}\newcommand{\supporterVtPrsDDuG}{\cacherGloses{Du-}supporter\cacherGloses{.VT}\cacherGloses{xPRS.D}}\newcommand{\supporterVtPrsDPl}{\strutgb{0pt}\grapho{gNkTB}}\newcommand{\supporterVtPrsDPlP}{\textipa{donekatemu}}\newcommand{\supporterVtPrsDPlG}{\cacherGloses{Pl-}supporter\cacherGloses{.VT}\cacherGloses{xPRS.D}}\newcommand{\supporterVtPstASg}{\strutgb{0pt}\grapho{GAEkM}}\newcommand{\supporterVtPstASgP}{\textipa{pimonokame}}\newcommand{\supporterVtPstASgG}{\cacherGloses{Sg-}supporter\cacherGloses{.VT}\cacherGloses{xPST.A}}\newcommand{\supporterVtPstADu}{\strutgb{0pt}\grapho{YAEkM}}\newcommand{\supporterVtPstADuP}{\textipa{simonokame}}\newcommand{\supporterVtPstADuG}{\cacherGloses{Du-}supporter\cacherGloses{.VT}\cacherGloses{xPST.A}}\newcommand{\supporterVtPstAPl}{\strutgb{0pt}\grapho{fAEkM}}\newcommand{\supporterVtPstAPlP}{\textipa{dimonokame}}\newcommand{\supporterVtPstAPlG}{\cacherGloses{Pl-}supporter\cacherGloses{.VT}\cacherGloses{xPST.A}}\newcommand{\supporterVtPstBSg}{\strutgb{0pt}\grapho{pAEvM}}\newcommand{\supporterVtPstBSgP}{\textipa{pamonokume}}\newcommand{\supporterVtPstBSgG}{\cacherGloses{Sg-}supporter\cacherGloses{.VT}\cacherGloses{xPST.B}}\newcommand{\supporterVtPstBDu}{\strutgb{0pt}\grapho{sAEvM}}\newcommand{\supporterVtPstBDuP}{\textipa{samonokume}}\newcommand{\supporterVtPstBDuG}{\cacherGloses{Du-}supporter\cacherGloses{.VT}\cacherGloses{xPST.B}}\newcommand{\supporterVtPstBPl}{\strutgb{0pt}\grapho{dAEvM}}\newcommand{\supporterVtPstBPlP}{\textipa{damonokume}}\newcommand{\supporterVtPstBPlG}{\cacherGloses{Pl-}supporter\cacherGloses{.VT}\cacherGloses{xPST.B}}\newcommand{\supporterVtPstCSg}{\strutgb{0pt}\grapho{HAEhM}}\newcommand{\supporterVtPstCSgP}{\textipa{pomonokome}}\newcommand{\supporterVtPstCSgG}{\cacherGloses{Sg-}supporter\cacherGloses{.VT}\cacherGloses{xPST.C}}\newcommand{\supporterVtPstCDu}{\strutgb{0pt}\grapho{1AEhM}}\newcommand{\supporterVtPstCDuP}{\textipa{somonokome}}\newcommand{\supporterVtPstCDuG}{\cacherGloses{Du-}supporter\cacherGloses{.VT}\cacherGloses{xPST.C}}\newcommand{\supporterVtPstCPl}{\strutgb{0pt}\grapho{gAEhM}}\newcommand{\supporterVtPstCPlP}{\textipa{domonokome}}\newcommand{\supporterVtPstCPlG}{\cacherGloses{Pl-}supporter\cacherGloses{.VT}\cacherGloses{xPST.C}}\newcommand{\supporterVtPstDSg}{\strutgb{0pt}\grapho{PAEhM}}\newcommand{\supporterVtPstDSgP}{\textipa{pemonokome}}\newcommand{\supporterVtPstDSgG}{\cacherGloses{Sg-}supporter\cacherGloses{.VT}\cacherGloses{xPST.D}}\newcommand{\supporterVtPstDDu}{\strutgb{0pt}\grapho{SAEhM}}\newcommand{\supporterVtPstDDuP}{\textipa{semonokome}}\newcommand{\supporterVtPstDDuG}{\cacherGloses{Du-}supporter\cacherGloses{.VT}\cacherGloses{xPST.D}}\newcommand{\supporterVtPstDPl}{\strutgb{0pt}\grapho{DAEhM}}\newcommand{\supporterVtPstDPlP}{\textipa{demonokome}}\newcommand{\supporterVtPstDPlG}{\cacherGloses{Pl-}supporter\cacherGloses{.VT}\cacherGloses{xPST.D}}\newcommand{\acheterVtPrsASg}{\strutgb{0pt}\grapho{HcrWL}}\newcommand{\acheterVtPrsASgP}{\textipa{pogiraweju}}\newcommand{\acheterVtPrsASgG}{\cacherGloses{Sg-}acheter\cacherGloses{.VT}\cacherGloses{xPRS.A}}\newcommand{\acheterVtPrsADu}{\strutgb{0pt}\grapho{1crWL}}\newcommand{\acheterVtPrsADuP}{\textipa{sogiraweju}}\newcommand{\acheterVtPrsADuG}{\cacherGloses{Du-}acheter\cacherGloses{.VT}\cacherGloses{xPRS.A}}\newcommand{\acheterVtPrsAPl}{\strutgb{0pt}\grapho{gcrWL}}\newcommand{\acheterVtPrsAPlP}{\textipa{dogiraweju}}\newcommand{\acheterVtPrsAPlG}{\cacherGloses{Pl-}acheter\cacherGloses{.VT}\cacherGloses{xPRS.A}}\newcommand{\acheterVtPrsBSg}{\strutgb{0pt}\grapho{HkrWL}}\newcommand{\acheterVtPrsBSgP}{\textipa{pogaraweju}}\newcommand{\acheterVtPrsBSgG}{\cacherGloses{Sg-}acheter\cacherGloses{.VT}\cacherGloses{xPRS.B}}\newcommand{\acheterVtPrsBDu}{\strutgb{0pt}\grapho{1krWL}}\newcommand{\acheterVtPrsBDuP}{\textipa{sogaraweju}}\newcommand{\acheterVtPrsBDuG}{\cacherGloses{Du-}acheter\cacherGloses{.VT}\cacherGloses{xPRS.B}}\newcommand{\acheterVtPrsBPl}{\strutgb{0pt}\grapho{gkrWL}}\newcommand{\acheterVtPrsBPlP}{\textipa{dogaraweju}}\newcommand{\acheterVtPrsBPlG}{\cacherGloses{Pl-}acheter\cacherGloses{.VT}\cacherGloses{xPRS.B}}\newcommand{\acheterVtPrsCSg}{\strutgb{0pt}\grapho{HhrWL}}\newcommand{\acheterVtPrsCSgP}{\textipa{pogoraweju}}\newcommand{\acheterVtPrsCSgG}{\cacherGloses{Sg-}acheter\cacherGloses{.VT}\cacherGloses{xPRS.C}}\newcommand{\acheterVtPrsCDu}{\strutgb{0pt}\grapho{1hrWL}}\newcommand{\acheterVtPrsCDuP}{\textipa{sogoraweju}}\newcommand{\acheterVtPrsCDuG}{\cacherGloses{Du-}acheter\cacherGloses{.VT}\cacherGloses{xPRS.C}}\newcommand{\acheterVtPrsCPl}{\strutgb{0pt}\grapho{ghrWL}}\newcommand{\acheterVtPrsCPlP}{\textipa{dogoraweju}}\newcommand{\acheterVtPrsCPlG}{\cacherGloses{Pl-}acheter\cacherGloses{.VT}\cacherGloses{xPRS.C}}\newcommand{\acheterVtPrsDSg}{\strutgb{0pt}\grapho{HKrWL}}\newcommand{\acheterVtPrsDSgP}{\textipa{pogeraweju}}\newcommand{\acheterVtPrsDSgG}{\cacherGloses{Sg-}acheter\cacherGloses{.VT}\cacherGloses{xPRS.D}}\newcommand{\acheterVtPrsDDu}{\strutgb{0pt}\grapho{1KrWL}}\newcommand{\acheterVtPrsDDuP}{\textipa{sogeraweju}}\newcommand{\acheterVtPrsDDuG}{\cacherGloses{Du-}acheter\cacherGloses{.VT}\cacherGloses{xPRS.D}}\newcommand{\acheterVtPrsDPl}{\strutgb{0pt}\grapho{gKrWL}}\newcommand{\acheterVtPrsDPlP}{\textipa{dogeraweju}}\newcommand{\acheterVtPrsDPlG}{\cacherGloses{Pl-}acheter\cacherGloses{.VT}\cacherGloses{xPRS.D}}\newcommand{\acheterVtPstASg}{\strutgb{0pt}\grapho{GghrJ}}\newcommand{\acheterVtPstASgP}{\textipa{pidogoraje}}\newcommand{\acheterVtPstASgG}{\cacherGloses{Sg-}acheter\cacherGloses{.VT}\cacherGloses{xPST.A}}\newcommand{\acheterVtPstADu}{\strutgb{0pt}\grapho{YghrJ}}\newcommand{\acheterVtPstADuP}{\textipa{sidogoraje}}\newcommand{\acheterVtPstADuG}{\cacherGloses{Du-}acheter\cacherGloses{.VT}\cacherGloses{xPST.A}}\newcommand{\acheterVtPstAPl}{\strutgb{0pt}\grapho{fghrJ}}\newcommand{\acheterVtPstAPlP}{\textipa{didogoraje}}\newcommand{\acheterVtPstAPlG}{\cacherGloses{Pl-}acheter\cacherGloses{.VT}\cacherGloses{xPST.A}}\newcommand{\acheterVtPstBSg}{\strutgb{0pt}\grapho{pghVJ}}\newcommand{\acheterVtPstBSgP}{\textipa{padogoruje}}\newcommand{\acheterVtPstBSgG}{\cacherGloses{Sg-}acheter\cacherGloses{.VT}\cacherGloses{xPST.B}}\newcommand{\acheterVtPstBDu}{\strutgb{0pt}\grapho{sghVJ}}\newcommand{\acheterVtPstBDuP}{\textipa{sadogoruje}}\newcommand{\acheterVtPstBDuG}{\cacherGloses{Du-}acheter\cacherGloses{.VT}\cacherGloses{xPST.B}}\newcommand{\acheterVtPstBPl}{\strutgb{0pt}\grapho{dghVJ}}\newcommand{\acheterVtPstBPlP}{\textipa{dadogoruje}}\newcommand{\acheterVtPstBPlG}{\cacherGloses{Pl-}acheter\cacherGloses{.VT}\cacherGloses{xPST.B}}\newcommand{\acheterVtPstCSg}{\strutgb{0pt}\grapho{HghUJ}}\newcommand{\acheterVtPstCSgP}{\textipa{podogoroje}}\newcommand{\acheterVtPstCSgG}{\cacherGloses{Sg-}acheter\cacherGloses{.VT}\cacherGloses{xPST.C}}\newcommand{\acheterVtPstCDu}{\strutgb{0pt}\grapho{1ghUJ}}\newcommand{\acheterVtPstCDuP}{\textipa{sodogoroje}}\newcommand{\acheterVtPstCDuG}{\cacherGloses{Du-}acheter\cacherGloses{.VT}\cacherGloses{xPST.C}}\newcommand{\acheterVtPstCPl}{\strutgb{0pt}\grapho{gghUJ}}\newcommand{\acheterVtPstCPlP}{\textipa{dodogoroje}}\newcommand{\acheterVtPstCPlG}{\cacherGloses{Pl-}acheter\cacherGloses{.VT}\cacherGloses{xPST.C}}\newcommand{\acheterVtPstDSg}{\strutgb{0pt}\grapho{PghUJ}}\newcommand{\acheterVtPstDSgP}{\textipa{pedogoroje}}\newcommand{\acheterVtPstDSgG}{\cacherGloses{Sg-}acheter\cacherGloses{.VT}\cacherGloses{xPST.D}}\newcommand{\acheterVtPstDDu}{\strutgb{0pt}\grapho{SghUJ}}\newcommand{\acheterVtPstDDuP}{\textipa{sedogoroje}}\newcommand{\acheterVtPstDDuG}{\cacherGloses{Du-}acheter\cacherGloses{.VT}\cacherGloses{xPST.D}}\newcommand{\acheterVtPstDPl}{\strutgb{0pt}\grapho{DghUJ}}\newcommand{\acheterVtPstDPlP}{\textipa{dedogoroje}}\newcommand{\acheterVtPstDPlG}{\cacherGloses{Pl-}acheter\cacherGloses{.VT}\cacherGloses{xPST.D}}\newcommand{\mangerVtPrsASg}{\strutgb{0pt}\grapho{HGrRx}}\newcommand{\mangerVtPrsASgP}{\textipa{popilaredu}}\newcommand{\mangerVtPrsASgG}{\cacherGloses{Sg-}manger\cacherGloses{.VT}\cacherGloses{xPRS.A}}\newcommand{\mangerVtPrsADu}{\strutgb{0pt}\grapho{1GrRx}}\newcommand{\mangerVtPrsADuP}{\textipa{sopilaredu}}\newcommand{\mangerVtPrsADuG}{\cacherGloses{Du-}manger\cacherGloses{.VT}\cacherGloses{xPRS.A}}\newcommand{\mangerVtPrsAPl}{\strutgb{0pt}\grapho{gGrRx}}\newcommand{\mangerVtPrsAPlP}{\textipa{dopilaredu}}\newcommand{\mangerVtPrsAPlG}{\cacherGloses{Pl-}manger\cacherGloses{.VT}\cacherGloses{xPRS.A}}\newcommand{\mangerVtPrsBSg}{\strutgb{0pt}\grapho{HprRx}}\newcommand{\mangerVtPrsBSgP}{\textipa{popalaredu}}\newcommand{\mangerVtPrsBSgG}{\cacherGloses{Sg-}manger\cacherGloses{.VT}\cacherGloses{xPRS.B}}\newcommand{\mangerVtPrsBDu}{\strutgb{0pt}\grapho{1prRx}}\newcommand{\mangerVtPrsBDuP}{\textipa{sopalaredu}}\newcommand{\mangerVtPrsBDuG}{\cacherGloses{Du-}manger\cacherGloses{.VT}\cacherGloses{xPRS.B}}\newcommand{\mangerVtPrsBPl}{\strutgb{0pt}\grapho{gprRx}}\newcommand{\mangerVtPrsBPlP}{\textipa{dopalaredu}}\newcommand{\mangerVtPrsBPlG}{\cacherGloses{Pl-}manger\cacherGloses{.VT}\cacherGloses{xPRS.B}}\newcommand{\mangerVtPrsCSg}{\strutgb{0pt}\grapho{HHrRx}}\newcommand{\mangerVtPrsCSgP}{\textipa{popolaredu}}\newcommand{\mangerVtPrsCSgG}{\cacherGloses{Sg-}manger\cacherGloses{.VT}\cacherGloses{xPRS.C}}\newcommand{\mangerVtPrsCDu}{\strutgb{0pt}\grapho{1HrRx}}\newcommand{\mangerVtPrsCDuP}{\textipa{sopolaredu}}\newcommand{\mangerVtPrsCDuG}{\cacherGloses{Du-}manger\cacherGloses{.VT}\cacherGloses{xPRS.C}}\newcommand{\mangerVtPrsCPl}{\strutgb{0pt}\grapho{gHrRx}}\newcommand{\mangerVtPrsCPlP}{\textipa{dopolaredu}}\newcommand{\mangerVtPrsCPlG}{\cacherGloses{Pl-}manger\cacherGloses{.VT}\cacherGloses{xPRS.C}}\newcommand{\mangerVtPrsDSg}{\strutgb{0pt}\grapho{HPrRx}}\newcommand{\mangerVtPrsDSgP}{\textipa{popelaredu}}\newcommand{\mangerVtPrsDSgG}{\cacherGloses{Sg-}manger\cacherGloses{.VT}\cacherGloses{xPRS.D}}\newcommand{\mangerVtPrsDDu}{\strutgb{0pt}\grapho{1PrRx}}\newcommand{\mangerVtPrsDDuP}{\textipa{sopelaredu}}\newcommand{\mangerVtPrsDDuG}{\cacherGloses{Du-}manger\cacherGloses{.VT}\cacherGloses{xPRS.D}}\newcommand{\mangerVtPrsDPl}{\strutgb{0pt}\grapho{gPrRx}}\newcommand{\mangerVtPrsDPlP}{\textipa{dopelaredu}}\newcommand{\mangerVtPrsDPlG}{\cacherGloses{Pl-}manger\cacherGloses{.VT}\cacherGloses{xPRS.D}}\newcommand{\mangerVtPstASg}{\strutgb{0pt}\grapho{GHHrD}}\newcommand{\mangerVtPstASgP}{\textipa{pipopolade}}\newcommand{\mangerVtPstASgG}{\cacherGloses{Sg-}manger\cacherGloses{.VT}\cacherGloses{xPST.A}}\newcommand{\mangerVtPstADu}{\strutgb{0pt}\grapho{YHHrD}}\newcommand{\mangerVtPstADuP}{\textipa{sipopolade}}\newcommand{\mangerVtPstADuG}{\cacherGloses{Du-}manger\cacherGloses{.VT}\cacherGloses{xPST.A}}\newcommand{\mangerVtPstAPl}{\strutgb{0pt}\grapho{fHHrD}}\newcommand{\mangerVtPstAPlP}{\textipa{dipopolade}}\newcommand{\mangerVtPstAPlG}{\cacherGloses{Pl-}manger\cacherGloses{.VT}\cacherGloses{xPST.A}}\newcommand{\mangerVtPstBSg}{\strutgb{0pt}\grapho{pHHVD}}\newcommand{\mangerVtPstBSgP}{\textipa{papopolude}}\newcommand{\mangerVtPstBSgG}{\cacherGloses{Sg-}manger\cacherGloses{.VT}\cacherGloses{xPST.B}}\newcommand{\mangerVtPstBDu}{\strutgb{0pt}\grapho{sHHVD}}\newcommand{\mangerVtPstBDuP}{\textipa{sapopolude}}\newcommand{\mangerVtPstBDuG}{\cacherGloses{Du-}manger\cacherGloses{.VT}\cacherGloses{xPST.B}}\newcommand{\mangerVtPstBPl}{\strutgb{0pt}\grapho{dHHVD}}\newcommand{\mangerVtPstBPlP}{\textipa{dapopolude}}\newcommand{\mangerVtPstBPlG}{\cacherGloses{Pl-}manger\cacherGloses{.VT}\cacherGloses{xPST.B}}\newcommand{\mangerVtPstCSg}{\strutgb{0pt}\grapho{HHHUD}}\newcommand{\mangerVtPstCSgP}{\textipa{popopolode}}\newcommand{\mangerVtPstCSgG}{\cacherGloses{Sg-}manger\cacherGloses{.VT}\cacherGloses{xPST.C}}\newcommand{\mangerVtPstCDu}{\strutgb{0pt}\grapho{1HHUD}}\newcommand{\mangerVtPstCDuP}{\textipa{sopopolode}}\newcommand{\mangerVtPstCDuG}{\cacherGloses{Du-}manger\cacherGloses{.VT}\cacherGloses{xPST.C}}\newcommand{\mangerVtPstCPl}{\strutgb{0pt}\grapho{gHHUD}}\newcommand{\mangerVtPstCPlP}{\textipa{dopopolode}}\newcommand{\mangerVtPstCPlG}{\cacherGloses{Pl-}manger\cacherGloses{.VT}\cacherGloses{xPST.C}}\newcommand{\mangerVtPstDSg}{\strutgb{0pt}\grapho{PHHUD}}\newcommand{\mangerVtPstDSgP}{\textipa{pepopolode}}\newcommand{\mangerVtPstDSgG}{\cacherGloses{Sg-}manger\cacherGloses{.VT}\cacherGloses{xPST.D}}\newcommand{\mangerVtPstDDu}{\strutgb{0pt}\grapho{SHHUD}}\newcommand{\mangerVtPstDDuP}{\textipa{sepopolode}}\newcommand{\mangerVtPstDDuG}{\cacherGloses{Du-}manger\cacherGloses{.VT}\cacherGloses{xPST.D}}\newcommand{\mangerVtPstDPl}{\strutgb{0pt}\grapho{DHHUD}}\newcommand{\mangerVtPstDPlP}{\textipa{depopolode}}\newcommand{\mangerVtPstDPlG}{\cacherGloses{Pl-}manger\cacherGloses{.VT}\cacherGloses{xPST.D}}\newcommand{\chasserVtPrsASg}{\strutgb{0pt}\grapho{HfnN5}}\newcommand{\chasserVtPrsASgP}{\textipa{podiNanetu}}\newcommand{\chasserVtPrsASgG}{\cacherGloses{Sg-}chasser\cacherGloses{.VT}\cacherGloses{xPRS.A}}\newcommand{\chasserVtPrsADu}{\strutgb{0pt}\grapho{1fnN5}}\newcommand{\chasserVtPrsADuP}{\textipa{sodiNanetu}}\newcommand{\chasserVtPrsADuG}{\cacherGloses{Du-}chasser\cacherGloses{.VT}\cacherGloses{xPRS.A}}\newcommand{\chasserVtPrsAPl}{\strutgb{0pt}\grapho{gfnN5}}\newcommand{\chasserVtPrsAPlP}{\textipa{dodiNanetu}}\newcommand{\chasserVtPrsAPlG}{\cacherGloses{Pl-}chasser\cacherGloses{.VT}\cacherGloses{xPRS.A}}\newcommand{\chasserVtPrsBSg}{\strutgb{0pt}\grapho{HdnN5}}\newcommand{\chasserVtPrsBSgP}{\textipa{podaNanetu}}\newcommand{\chasserVtPrsBSgG}{\cacherGloses{Sg-}chasser\cacherGloses{.VT}\cacherGloses{xPRS.B}}\newcommand{\chasserVtPrsBDu}{\strutgb{0pt}\grapho{1dnN5}}\newcommand{\chasserVtPrsBDuP}{\textipa{sodaNanetu}}\newcommand{\chasserVtPrsBDuG}{\cacherGloses{Du-}chasser\cacherGloses{.VT}\cacherGloses{xPRS.B}}\newcommand{\chasserVtPrsBPl}{\strutgb{0pt}\grapho{gdnN5}}\newcommand{\chasserVtPrsBPlP}{\textipa{dodaNanetu}}\newcommand{\chasserVtPrsBPlG}{\cacherGloses{Pl-}chasser\cacherGloses{.VT}\cacherGloses{xPRS.B}}\newcommand{\chasserVtPrsCSg}{\strutgb{0pt}\grapho{HgnN5}}\newcommand{\chasserVtPrsCSgP}{\textipa{podoNanetu}}\newcommand{\chasserVtPrsCSgG}{\cacherGloses{Sg-}chasser\cacherGloses{.VT}\cacherGloses{xPRS.C}}\newcommand{\chasserVtPrsCDu}{\strutgb{0pt}\grapho{1gnN5}}\newcommand{\chasserVtPrsCDuP}{\textipa{sodoNanetu}}\newcommand{\chasserVtPrsCDuG}{\cacherGloses{Du-}chasser\cacherGloses{.VT}\cacherGloses{xPRS.C}}\newcommand{\chasserVtPrsCPl}{\strutgb{0pt}\grapho{ggnN5}}\newcommand{\chasserVtPrsCPlP}{\textipa{dodoNanetu}}\newcommand{\chasserVtPrsCPlG}{\cacherGloses{Pl-}chasser\cacherGloses{.VT}\cacherGloses{xPRS.C}}\newcommand{\chasserVtPrsDSg}{\strutgb{0pt}\grapho{HDnN5}}\newcommand{\chasserVtPrsDSgP}{\textipa{podeNanetu}}\newcommand{\chasserVtPrsDSgG}{\cacherGloses{Sg-}chasser\cacherGloses{.VT}\cacherGloses{xPRS.D}}\newcommand{\chasserVtPrsDDu}{\strutgb{0pt}\grapho{1DnN5}}\newcommand{\chasserVtPrsDDuP}{\textipa{sodeNanetu}}\newcommand{\chasserVtPrsDDuG}{\cacherGloses{Du-}chasser\cacherGloses{.VT}\cacherGloses{xPRS.D}}\newcommand{\chasserVtPrsDPl}{\strutgb{0pt}\grapho{gDnN5}}\newcommand{\chasserVtPrsDPlP}{\textipa{dodeNanetu}}\newcommand{\chasserVtPrsDPlG}{\cacherGloses{Pl-}chasser\cacherGloses{.VT}\cacherGloses{xPRS.D}}\newcommand{\chasserVtPstASg}{\strutgb{0pt}\grapho{GHgnT}}\newcommand{\chasserVtPstASgP}{\textipa{pibodoNate}}\newcommand{\chasserVtPstASgG}{\cacherGloses{Sg-}chasser\cacherGloses{.VT}\cacherGloses{xPST.A}}\newcommand{\chasserVtPstADu}{\strutgb{0pt}\grapho{YHgnT}}\newcommand{\chasserVtPstADuP}{\textipa{sibodoNate}}\newcommand{\chasserVtPstADuG}{\cacherGloses{Du-}chasser\cacherGloses{.VT}\cacherGloses{xPST.A}}\newcommand{\chasserVtPstAPl}{\strutgb{0pt}\grapho{fHgnT}}\newcommand{\chasserVtPstAPlP}{\textipa{dibodoNate}}\newcommand{\chasserVtPstAPlG}{\cacherGloses{Pl-}chasser\cacherGloses{.VT}\cacherGloses{xPST.A}}\newcommand{\chasserVtPstBSg}{\strutgb{0pt}\grapho{pHgFT}}\newcommand{\chasserVtPstBSgP}{\textipa{pabodoNute}}\newcommand{\chasserVtPstBSgG}{\cacherGloses{Sg-}chasser\cacherGloses{.VT}\cacherGloses{xPST.B}}\newcommand{\chasserVtPstBDu}{\strutgb{0pt}\grapho{sHgFT}}\newcommand{\chasserVtPstBDuP}{\textipa{sabodoNute}}\newcommand{\chasserVtPstBDuG}{\cacherGloses{Du-}chasser\cacherGloses{.VT}\cacherGloses{xPST.B}}\newcommand{\chasserVtPstBPl}{\strutgb{0pt}\grapho{dHgFT}}\newcommand{\chasserVtPstBPlP}{\textipa{dabodoNute}}\newcommand{\chasserVtPstBPlG}{\cacherGloses{Pl-}chasser\cacherGloses{.VT}\cacherGloses{xPST.B}}\newcommand{\chasserVtPstCSg}{\strutgb{0pt}\grapho{HHgET}}\newcommand{\chasserVtPstCSgP}{\textipa{pobodoNote}}\newcommand{\chasserVtPstCSgG}{\cacherGloses{Sg-}chasser\cacherGloses{.VT}\cacherGloses{xPST.C}}\newcommand{\chasserVtPstCDu}{\strutgb{0pt}\grapho{1HgET}}\newcommand{\chasserVtPstCDuP}{\textipa{sobodoNote}}\newcommand{\chasserVtPstCDuG}{\cacherGloses{Du-}chasser\cacherGloses{.VT}\cacherGloses{xPST.C}}\newcommand{\chasserVtPstCPl}{\strutgb{0pt}\grapho{gHgET}}\newcommand{\chasserVtPstCPlP}{\textipa{dobodoNote}}\newcommand{\chasserVtPstCPlG}{\cacherGloses{Pl-}chasser\cacherGloses{.VT}\cacherGloses{xPST.C}}\newcommand{\chasserVtPstDSg}{\strutgb{0pt}\grapho{PHgET}}\newcommand{\chasserVtPstDSgP}{\textipa{pebodoNote}}\newcommand{\chasserVtPstDSgG}{\cacherGloses{Sg-}chasser\cacherGloses{.VT}\cacherGloses{xPST.D}}\newcommand{\chasserVtPstDDu}{\strutgb{0pt}\grapho{SHgET}}\newcommand{\chasserVtPstDDuP}{\textipa{sebodoNote}}\newcommand{\chasserVtPstDDuG}{\cacherGloses{Du-}chasser\cacherGloses{.VT}\cacherGloses{xPST.D}}\newcommand{\chasserVtPstDPl}{\strutgb{0pt}\grapho{DHgET}}\newcommand{\chasserVtPstDPlP}{\textipa{debodoNote}}\newcommand{\chasserVtPstDPlG}{\cacherGloses{Pl-}chasser\cacherGloses{.VT}\cacherGloses{xPST.D}}\newcommand{\donnerVdPrsASg}{\strutgb{0pt}\grapho{IGsqv}}\newcommand{\donnerVdPrsASgP}{\textipa{pupisafagu}}\newcommand{\donnerVdPrsASgG}{\cacherGloses{Sg-}donner\cacherGloses{.VD}\cacherGloses{xPRS.A}}\newcommand{\donnerVdPrsADu}{\strutgb{0pt}\grapho{2Gsqv}}\newcommand{\donnerVdPrsADuP}{\textipa{supisafagu}}\newcommand{\donnerVdPrsADuG}{\cacherGloses{Du-}donner\cacherGloses{.VD}\cacherGloses{xPRS.A}}\newcommand{\donnerVdPrsAPl}{\strutgb{0pt}\grapho{xGsqv}}\newcommand{\donnerVdPrsAPlP}{\textipa{dupisafagu}}\newcommand{\donnerVdPrsAPlG}{\cacherGloses{Pl-}donner\cacherGloses{.VD}\cacherGloses{xPRS.A}}\newcommand{\donnerVdPrsBSg}{\strutgb{0pt}\grapho{Ipsqv}}\newcommand{\donnerVdPrsBSgP}{\textipa{pupasafagu}}\newcommand{\donnerVdPrsBSgG}{\cacherGloses{Sg-}donner\cacherGloses{.VD}\cacherGloses{xPRS.B}}\newcommand{\donnerVdPrsBDu}{\strutgb{0pt}\grapho{2psqv}}\newcommand{\donnerVdPrsBDuP}{\textipa{supasafagu}}\newcommand{\donnerVdPrsBDuG}{\cacherGloses{Du-}donner\cacherGloses{.VD}\cacherGloses{xPRS.B}}\newcommand{\donnerVdPrsBPl}{\strutgb{0pt}\grapho{xpsqv}}\newcommand{\donnerVdPrsBPlP}{\textipa{dupasafagu}}\newcommand{\donnerVdPrsBPlG}{\cacherGloses{Pl-}donner\cacherGloses{.VD}\cacherGloses{xPRS.B}}\newcommand{\donnerVdPrsCSg}{\strutgb{0pt}\grapho{IHsqv}}\newcommand{\donnerVdPrsCSgP}{\textipa{puposafagu}}\newcommand{\donnerVdPrsCSgG}{\cacherGloses{Sg-}donner\cacherGloses{.VD}\cacherGloses{xPRS.C}}\newcommand{\donnerVdPrsCDu}{\strutgb{0pt}\grapho{2Hsqv}}\newcommand{\donnerVdPrsCDuP}{\textipa{suposafagu}}\newcommand{\donnerVdPrsCDuG}{\cacherGloses{Du-}donner\cacherGloses{.VD}\cacherGloses{xPRS.C}}\newcommand{\donnerVdPrsCPl}{\strutgb{0pt}\grapho{xHsqv}}\newcommand{\donnerVdPrsCPlP}{\textipa{duposafagu}}\newcommand{\donnerVdPrsCPlG}{\cacherGloses{Pl-}donner\cacherGloses{.VD}\cacherGloses{xPRS.C}}\newcommand{\donnerVdPrsDSg}{\strutgb{0pt}\grapho{IPsqv}}\newcommand{\donnerVdPrsDSgP}{\textipa{pupesafagu}}\newcommand{\donnerVdPrsDSgG}{\cacherGloses{Sg-}donner\cacherGloses{.VD}\cacherGloses{xPRS.D}}\newcommand{\donnerVdPrsDDu}{\strutgb{0pt}\grapho{2Psqv}}\newcommand{\donnerVdPrsDDuP}{\textipa{supesafagu}}\newcommand{\donnerVdPrsDDuG}{\cacherGloses{Du-}donner\cacherGloses{.VD}\cacherGloses{xPRS.D}}\newcommand{\donnerVdPrsDPl}{\strutgb{0pt}\grapho{xPsqv}}\newcommand{\donnerVdPrsDPlP}{\textipa{dupesafagu}}\newcommand{\donnerVdPrsDPlG}{\cacherGloses{Pl-}donner\cacherGloses{.VD}\cacherGloses{xPRS.D}}\newcommand{\donnerVdPstASg}{\strutgb{0pt}\grapho{GIHsk}}\newcommand{\donnerVdPstASgP}{\textipa{pipuposaga}}\newcommand{\donnerVdPstASgG}{\cacherGloses{Sg-}donner\cacherGloses{.VD}\cacherGloses{xPST.A}}\newcommand{\donnerVdPstADu}{\strutgb{0pt}\grapho{YIHsk}}\newcommand{\donnerVdPstADuP}{\textipa{sipuposaga}}\newcommand{\donnerVdPstADuG}{\cacherGloses{Du-}donner\cacherGloses{.VD}\cacherGloses{xPST.A}}\newcommand{\donnerVdPstAPl}{\strutgb{0pt}\grapho{fIHsk}}\newcommand{\donnerVdPstAPlP}{\textipa{dipuposaga}}\newcommand{\donnerVdPstAPlG}{\cacherGloses{Pl-}donner\cacherGloses{.VD}\cacherGloses{xPST.A}}\newcommand{\donnerVdPstBSg}{\strutgb{0pt}\grapho{pIH2k}}\newcommand{\donnerVdPstBSgP}{\textipa{papuposuga}}\newcommand{\donnerVdPstBSgG}{\cacherGloses{Sg-}donner\cacherGloses{.VD}\cacherGloses{xPST.B}}\newcommand{\donnerVdPstBDu}{\strutgb{0pt}\grapho{sIH2k}}\newcommand{\donnerVdPstBDuP}{\textipa{sapuposuga}}\newcommand{\donnerVdPstBDuG}{\cacherGloses{Du-}donner\cacherGloses{.VD}\cacherGloses{xPST.B}}\newcommand{\donnerVdPstBPl}{\strutgb{0pt}\grapho{dIH2k}}\newcommand{\donnerVdPstBPlP}{\textipa{dapuposuga}}\newcommand{\donnerVdPstBPlG}{\cacherGloses{Pl-}donner\cacherGloses{.VD}\cacherGloses{xPST.B}}\newcommand{\donnerVdPstCSg}{\strutgb{0pt}\grapho{HIH1k}}\newcommand{\donnerVdPstCSgP}{\textipa{popuposoga}}\newcommand{\donnerVdPstCSgG}{\cacherGloses{Sg-}donner\cacherGloses{.VD}\cacherGloses{xPST.C}}\newcommand{\donnerVdPstCDu}{\strutgb{0pt}\grapho{1IH1k}}\newcommand{\donnerVdPstCDuP}{\textipa{sopuposoga}}\newcommand{\donnerVdPstCDuG}{\cacherGloses{Du-}donner\cacherGloses{.VD}\cacherGloses{xPST.C}}\newcommand{\donnerVdPstCPl}{\strutgb{0pt}\grapho{gIH1k}}\newcommand{\donnerVdPstCPlP}{\textipa{dopuposoga}}\newcommand{\donnerVdPstCPlG}{\cacherGloses{Pl-}donner\cacherGloses{.VD}\cacherGloses{xPST.C}}\newcommand{\donnerVdPstDSg}{\strutgb{0pt}\grapho{PIH1k}}\newcommand{\donnerVdPstDSgP}{\textipa{pepuposoga}}\newcommand{\donnerVdPstDSgG}{\cacherGloses{Sg-}donner\cacherGloses{.VD}\cacherGloses{xPST.D}}\newcommand{\donnerVdPstDDu}{\strutgb{0pt}\grapho{SIH1k}}\newcommand{\donnerVdPstDDuP}{\textipa{sepuposoga}}\newcommand{\donnerVdPstDDuG}{\cacherGloses{Du-}donner\cacherGloses{.VD}\cacherGloses{xPST.D}}\newcommand{\donnerVdPstDPl}{\strutgb{0pt}\grapho{DIH1k}}\newcommand{\donnerVdPstDPlP}{\textipa{depuposoga}}\newcommand{\donnerVdPstDPlG}{\cacherGloses{Pl-}donner\cacherGloses{.VD}\cacherGloses{xPST.D}}\newcommand{\lancerVdPrsASg}{\strutgb{0pt}\grapho{IGrwV}}\newcommand{\lancerVdPrsASgP}{\textipa{pupirawalu}}\newcommand{\lancerVdPrsASgG}{\cacherGloses{Sg-}lancer\cacherGloses{.VD}\cacherGloses{xPRS.A}}\newcommand{\lancerVdPrsADu}{\strutgb{0pt}\grapho{2GrwV}}\newcommand{\lancerVdPrsADuP}{\textipa{supirawalu}}\newcommand{\lancerVdPrsADuG}{\cacherGloses{Du-}lancer\cacherGloses{.VD}\cacherGloses{xPRS.A}}\newcommand{\lancerVdPrsAPl}{\strutgb{0pt}\grapho{xGrwV}}\newcommand{\lancerVdPrsAPlP}{\textipa{dupirawalu}}\newcommand{\lancerVdPrsAPlG}{\cacherGloses{Pl-}lancer\cacherGloses{.VD}\cacherGloses{xPRS.A}}\newcommand{\lancerVdPrsBSg}{\strutgb{0pt}\grapho{IprwV}}\newcommand{\lancerVdPrsBSgP}{\textipa{puparawalu}}\newcommand{\lancerVdPrsBSgG}{\cacherGloses{Sg-}lancer\cacherGloses{.VD}\cacherGloses{xPRS.B}}\newcommand{\lancerVdPrsBDu}{\strutgb{0pt}\grapho{2prwV}}\newcommand{\lancerVdPrsBDuP}{\textipa{suparawalu}}\newcommand{\lancerVdPrsBDuG}{\cacherGloses{Du-}lancer\cacherGloses{.VD}\cacherGloses{xPRS.B}}\newcommand{\lancerVdPrsBPl}{\strutgb{0pt}\grapho{xprwV}}\newcommand{\lancerVdPrsBPlP}{\textipa{duparawalu}}\newcommand{\lancerVdPrsBPlG}{\cacherGloses{Pl-}lancer\cacherGloses{.VD}\cacherGloses{xPRS.B}}\newcommand{\lancerVdPrsCSg}{\strutgb{0pt}\grapho{IHrwV}}\newcommand{\lancerVdPrsCSgP}{\textipa{puporawalu}}\newcommand{\lancerVdPrsCSgG}{\cacherGloses{Sg-}lancer\cacherGloses{.VD}\cacherGloses{xPRS.C}}\newcommand{\lancerVdPrsCDu}{\strutgb{0pt}\grapho{2HrwV}}\newcommand{\lancerVdPrsCDuP}{\textipa{suporawalu}}\newcommand{\lancerVdPrsCDuG}{\cacherGloses{Du-}lancer\cacherGloses{.VD}\cacherGloses{xPRS.C}}\newcommand{\lancerVdPrsCPl}{\strutgb{0pt}\grapho{xHrwV}}\newcommand{\lancerVdPrsCPlP}{\textipa{duporawalu}}\newcommand{\lancerVdPrsCPlG}{\cacherGloses{Pl-}lancer\cacherGloses{.VD}\cacherGloses{xPRS.C}}\newcommand{\lancerVdPrsDSg}{\strutgb{0pt}\grapho{IPrwV}}\newcommand{\lancerVdPrsDSgP}{\textipa{puperawalu}}\newcommand{\lancerVdPrsDSgG}{\cacherGloses{Sg-}lancer\cacherGloses{.VD}\cacherGloses{xPRS.D}}\newcommand{\lancerVdPrsDDu}{\strutgb{0pt}\grapho{2PrwV}}\newcommand{\lancerVdPrsDDuP}{\textipa{superawalu}}\newcommand{\lancerVdPrsDDuG}{\cacherGloses{Du-}lancer\cacherGloses{.VD}\cacherGloses{xPRS.D}}\newcommand{\lancerVdPrsDPl}{\strutgb{0pt}\grapho{xPrwV}}\newcommand{\lancerVdPrsDPlP}{\textipa{duperawalu}}\newcommand{\lancerVdPrsDPlG}{\cacherGloses{Pl-}lancer\cacherGloses{.VD}\cacherGloses{xPRS.D}}\newcommand{\lancerVdPstASg}{\strutgb{0pt}\grapho{GIHrr}}\newcommand{\lancerVdPstASgP}{\textipa{pipuporala}}\newcommand{\lancerVdPstASgG}{\cacherGloses{Sg-}lancer\cacherGloses{.VD}\cacherGloses{xPST.A}}\newcommand{\lancerVdPstADu}{\strutgb{0pt}\grapho{YIHrr}}\newcommand{\lancerVdPstADuP}{\textipa{sipuporala}}\newcommand{\lancerVdPstADuG}{\cacherGloses{Du-}lancer\cacherGloses{.VD}\cacherGloses{xPST.A}}\newcommand{\lancerVdPstAPl}{\strutgb{0pt}\grapho{fIHrr}}\newcommand{\lancerVdPstAPlP}{\textipa{dipuporala}}\newcommand{\lancerVdPstAPlG}{\cacherGloses{Pl-}lancer\cacherGloses{.VD}\cacherGloses{xPST.A}}\newcommand{\lancerVdPstBSg}{\strutgb{0pt}\grapho{pIHVr}}\newcommand{\lancerVdPstBSgP}{\textipa{papuporula}}\newcommand{\lancerVdPstBSgG}{\cacherGloses{Sg-}lancer\cacherGloses{.VD}\cacherGloses{xPST.B}}\newcommand{\lancerVdPstBDu}{\strutgb{0pt}\grapho{sIHVr}}\newcommand{\lancerVdPstBDuP}{\textipa{sapuporula}}\newcommand{\lancerVdPstBDuG}{\cacherGloses{Du-}lancer\cacherGloses{.VD}\cacherGloses{xPST.B}}\newcommand{\lancerVdPstBPl}{\strutgb{0pt}\grapho{dIHVr}}\newcommand{\lancerVdPstBPlP}{\textipa{dapuporula}}\newcommand{\lancerVdPstBPlG}{\cacherGloses{Pl-}lancer\cacherGloses{.VD}\cacherGloses{xPST.B}}\newcommand{\lancerVdPstCSg}{\strutgb{0pt}\grapho{HIHUr}}\newcommand{\lancerVdPstCSgP}{\textipa{popuporola}}\newcommand{\lancerVdPstCSgG}{\cacherGloses{Sg-}lancer\cacherGloses{.VD}\cacherGloses{xPST.C}}\newcommand{\lancerVdPstCDu}{\strutgb{0pt}\grapho{1IHUr}}\newcommand{\lancerVdPstCDuP}{\textipa{sopuporola}}\newcommand{\lancerVdPstCDuG}{\cacherGloses{Du-}lancer\cacherGloses{.VD}\cacherGloses{xPST.C}}\newcommand{\lancerVdPstCPl}{\strutgb{0pt}\grapho{gIHUr}}\newcommand{\lancerVdPstCPlP}{\textipa{dopuporola}}\newcommand{\lancerVdPstCPlG}{\cacherGloses{Pl-}lancer\cacherGloses{.VD}\cacherGloses{xPST.C}}\newcommand{\lancerVdPstDSg}{\strutgb{0pt}\grapho{PIHUr}}\newcommand{\lancerVdPstDSgP}{\textipa{pepuporola}}\newcommand{\lancerVdPstDSgG}{\cacherGloses{Sg-}lancer\cacherGloses{.VD}\cacherGloses{xPST.D}}\newcommand{\lancerVdPstDDu}{\strutgb{0pt}\grapho{SIHUr}}\newcommand{\lancerVdPstDDuP}{\textipa{sepuporola}}\newcommand{\lancerVdPstDDuG}{\cacherGloses{Du-}lancer\cacherGloses{.VD}\cacherGloses{xPST.D}}\newcommand{\lancerVdPstDPl}{\strutgb{0pt}\grapho{DIHUr}}\newcommand{\lancerVdPstDPlP}{\textipa{depuporola}}\newcommand{\lancerVdPstDPlG}{\cacherGloses{Pl-}lancer\cacherGloses{.VD}\cacherGloses{xPST.D}}\newcommand{\offrirVdPrsASg}{\strutgb{0pt}\grapho{IOdpF}}\newcommand{\offrirVdPrsASgP}{\textipa{puridabanu}}\newcommand{\offrirVdPrsASgG}{\cacherGloses{Sg-}offrir\cacherGloses{.VD}\cacherGloses{xPRS.A}}\newcommand{\offrirVdPrsADu}{\strutgb{0pt}\grapho{2OdpF}}\newcommand{\offrirVdPrsADuP}{\textipa{suridabanu}}\newcommand{\offrirVdPrsADuG}{\cacherGloses{Du-}offrir\cacherGloses{.VD}\cacherGloses{xPRS.A}}\newcommand{\offrirVdPrsAPl}{\strutgb{0pt}\grapho{xOdpF}}\newcommand{\offrirVdPrsAPlP}{\textipa{duridabanu}}\newcommand{\offrirVdPrsAPlG}{\cacherGloses{Pl-}offrir\cacherGloses{.VD}\cacherGloses{xPRS.A}}\newcommand{\offrirVdPrsBSg}{\strutgb{0pt}\grapho{IrdpF}}\newcommand{\offrirVdPrsBSgP}{\textipa{puradabanu}}\newcommand{\offrirVdPrsBSgG}{\cacherGloses{Sg-}offrir\cacherGloses{.VD}\cacherGloses{xPRS.B}}\newcommand{\offrirVdPrsBDu}{\strutgb{0pt}\grapho{2rdpF}}\newcommand{\offrirVdPrsBDuP}{\textipa{suradabanu}}\newcommand{\offrirVdPrsBDuG}{\cacherGloses{Du-}offrir\cacherGloses{.VD}\cacherGloses{xPRS.B}}\newcommand{\offrirVdPrsBPl}{\strutgb{0pt}\grapho{xrdpF}}\newcommand{\offrirVdPrsBPlP}{\textipa{duradabanu}}\newcommand{\offrirVdPrsBPlG}{\cacherGloses{Pl-}offrir\cacherGloses{.VD}\cacherGloses{xPRS.B}}\newcommand{\offrirVdPrsCSg}{\strutgb{0pt}\grapho{IUdpF}}\newcommand{\offrirVdPrsCSgP}{\textipa{purodabanu}}\newcommand{\offrirVdPrsCSgG}{\cacherGloses{Sg-}offrir\cacherGloses{.VD}\cacherGloses{xPRS.C}}\newcommand{\offrirVdPrsCDu}{\strutgb{0pt}\grapho{2UdpF}}\newcommand{\offrirVdPrsCDuP}{\textipa{surodabanu}}\newcommand{\offrirVdPrsCDuG}{\cacherGloses{Du-}offrir\cacherGloses{.VD}\cacherGloses{xPRS.C}}\newcommand{\offrirVdPrsCPl}{\strutgb{0pt}\grapho{xUdpF}}\newcommand{\offrirVdPrsCPlP}{\textipa{durodabanu}}\newcommand{\offrirVdPrsCPlG}{\cacherGloses{Pl-}offrir\cacherGloses{.VD}\cacherGloses{xPRS.C}}\newcommand{\offrirVdPrsDSg}{\strutgb{0pt}\grapho{IRdpF}}\newcommand{\offrirVdPrsDSgP}{\textipa{puredabanu}}\newcommand{\offrirVdPrsDSgG}{\cacherGloses{Sg-}offrir\cacherGloses{.VD}\cacherGloses{xPRS.D}}\newcommand{\offrirVdPrsDDu}{\strutgb{0pt}\grapho{2RdpF}}\newcommand{\offrirVdPrsDDuP}{\textipa{suredabanu}}\newcommand{\offrirVdPrsDDuG}{\cacherGloses{Du-}offrir\cacherGloses{.VD}\cacherGloses{xPRS.D}}\newcommand{\offrirVdPrsDPl}{\strutgb{0pt}\grapho{xRdpF}}\newcommand{\offrirVdPrsDPlP}{\textipa{duredabanu}}\newcommand{\offrirVdPrsDPlG}{\cacherGloses{Pl-}offrir\cacherGloses{.VD}\cacherGloses{xPRS.D}}\newcommand{\offrirVdPstASg}{\strutgb{0pt}\grapho{GwuUdn}}\newcommand{\offrirVdPstASgP}{\textipa{piwurodana}}\newcommand{\offrirVdPstASgG}{\cacherGloses{Sg-}offrir\cacherGloses{.VD}\cacherGloses{xPST.A}}\newcommand{\offrirVdPstADu}{\strutgb{0pt}\grapho{YwuUdn}}\newcommand{\offrirVdPstADuP}{\textipa{siwurodana}}\newcommand{\offrirVdPstADuG}{\cacherGloses{Du-}offrir\cacherGloses{.VD}\cacherGloses{xPST.A}}\newcommand{\offrirVdPstAPl}{\strutgb{0pt}\grapho{fwuUdn}}\newcommand{\offrirVdPstAPlP}{\textipa{diwurodana}}\newcommand{\offrirVdPstAPlG}{\cacherGloses{Pl-}offrir\cacherGloses{.VD}\cacherGloses{xPST.A}}\newcommand{\offrirVdPstBSg}{\strutgb{0pt}\grapho{pwuUxn}}\newcommand{\offrirVdPstBSgP}{\textipa{pawuroduna}}\newcommand{\offrirVdPstBSgG}{\cacherGloses{Sg-}offrir\cacherGloses{.VD}\cacherGloses{xPST.B}}\newcommand{\offrirVdPstBDu}{\strutgb{0pt}\grapho{swuUxn}}\newcommand{\offrirVdPstBDuP}{\textipa{sawuroduna}}\newcommand{\offrirVdPstBDuG}{\cacherGloses{Du-}offrir\cacherGloses{.VD}\cacherGloses{xPST.B}}\newcommand{\offrirVdPstBPl}{\strutgb{0pt}\grapho{dwuUxn}}\newcommand{\offrirVdPstBPlP}{\textipa{dawuroduna}}\newcommand{\offrirVdPstBPlG}{\cacherGloses{Pl-}offrir\cacherGloses{.VD}\cacherGloses{xPST.B}}\newcommand{\offrirVdPstCSg}{\strutgb{0pt}\grapho{HwuUgn}}\newcommand{\offrirVdPstCSgP}{\textipa{powurodona}}\newcommand{\offrirVdPstCSgG}{\cacherGloses{Sg-}offrir\cacherGloses{.VD}\cacherGloses{xPST.C}}\newcommand{\offrirVdPstCDu}{\strutgb{0pt}\grapho{1wuUgn}}\newcommand{\offrirVdPstCDuP}{\textipa{sowurodona}}\newcommand{\offrirVdPstCDuG}{\cacherGloses{Du-}offrir\cacherGloses{.VD}\cacherGloses{xPST.C}}\newcommand{\offrirVdPstCPl}{\strutgb{0pt}\grapho{gwuUgn}}\newcommand{\offrirVdPstCPlP}{\textipa{dowurodona}}\newcommand{\offrirVdPstCPlG}{\cacherGloses{Pl-}offrir\cacherGloses{.VD}\cacherGloses{xPST.C}}\newcommand{\offrirVdPstDSg}{\strutgb{0pt}\grapho{PwuUgn}}\newcommand{\offrirVdPstDSgP}{\textipa{pewurodona}}\newcommand{\offrirVdPstDSgG}{\cacherGloses{Sg-}offrir\cacherGloses{.VD}\cacherGloses{xPST.D}}\newcommand{\offrirVdPstDDu}{\strutgb{0pt}\grapho{SwuUgn}}\newcommand{\offrirVdPstDDuP}{\textipa{sewurodona}}\newcommand{\offrirVdPstDDuG}{\cacherGloses{Du-}offrir\cacherGloses{.VD}\cacherGloses{xPST.D}}\newcommand{\offrirVdPstDPl}{\strutgb{0pt}\grapho{DwuUgn}}\newcommand{\offrirVdPstDPlP}{\textipa{dewurodona}}\newcommand{\offrirVdPstDPlG}{\cacherGloses{Pl-}offrir\cacherGloses{.VD}\cacherGloses{xPST.D}}\newcommand{\montrerVdPrsASg}{\strutgb{0pt}\grapho{IGnmI}}\newcommand{\montrerVdPrsASgP}{\textipa{pupinamabu}}\newcommand{\montrerVdPrsASgG}{\cacherGloses{Sg-}montrer\cacherGloses{.VD}\cacherGloses{xPRS.A}}\newcommand{\montrerVdPrsADu}{\strutgb{0pt}\grapho{2GnmI}}\newcommand{\montrerVdPrsADuP}{\textipa{supinamabu}}\newcommand{\montrerVdPrsADuG}{\cacherGloses{Du-}montrer\cacherGloses{.VD}\cacherGloses{xPRS.A}}\newcommand{\montrerVdPrsAPl}{\strutgb{0pt}\grapho{xGnmI}}\newcommand{\montrerVdPrsAPlP}{\textipa{dupinamabu}}\newcommand{\montrerVdPrsAPlG}{\cacherGloses{Pl-}montrer\cacherGloses{.VD}\cacherGloses{xPRS.A}}\newcommand{\montrerVdPrsBSg}{\strutgb{0pt}\grapho{IpnmI}}\newcommand{\montrerVdPrsBSgP}{\textipa{pupanamabu}}\newcommand{\montrerVdPrsBSgG}{\cacherGloses{Sg-}montrer\cacherGloses{.VD}\cacherGloses{xPRS.B}}\newcommand{\montrerVdPrsBDu}{\strutgb{0pt}\grapho{2pnmI}}\newcommand{\montrerVdPrsBDuP}{\textipa{supanamabu}}\newcommand{\montrerVdPrsBDuG}{\cacherGloses{Du-}montrer\cacherGloses{.VD}\cacherGloses{xPRS.B}}\newcommand{\montrerVdPrsBPl}{\strutgb{0pt}\grapho{xpnmI}}\newcommand{\montrerVdPrsBPlP}{\textipa{dupanamabu}}\newcommand{\montrerVdPrsBPlG}{\cacherGloses{Pl-}montrer\cacherGloses{.VD}\cacherGloses{xPRS.B}}\newcommand{\montrerVdPrsCSg}{\strutgb{0pt}\grapho{IHnmI}}\newcommand{\montrerVdPrsCSgP}{\textipa{puponamabu}}\newcommand{\montrerVdPrsCSgG}{\cacherGloses{Sg-}montrer\cacherGloses{.VD}\cacherGloses{xPRS.C}}\newcommand{\montrerVdPrsCDu}{\strutgb{0pt}\grapho{2HnmI}}\newcommand{\montrerVdPrsCDuP}{\textipa{suponamabu}}\newcommand{\montrerVdPrsCDuG}{\cacherGloses{Du-}montrer\cacherGloses{.VD}\cacherGloses{xPRS.C}}\newcommand{\montrerVdPrsCPl}{\strutgb{0pt}\grapho{xHnmI}}\newcommand{\montrerVdPrsCPlP}{\textipa{duponamabu}}\newcommand{\montrerVdPrsCPlG}{\cacherGloses{Pl-}montrer\cacherGloses{.VD}\cacherGloses{xPRS.C}}\newcommand{\montrerVdPrsDSg}{\strutgb{0pt}\grapho{IPnmI}}\newcommand{\montrerVdPrsDSgP}{\textipa{pupenamabu}}\newcommand{\montrerVdPrsDSgG}{\cacherGloses{Sg-}montrer\cacherGloses{.VD}\cacherGloses{xPRS.D}}\newcommand{\montrerVdPrsDDu}{\strutgb{0pt}\grapho{2PnmI}}\newcommand{\montrerVdPrsDDuP}{\textipa{supenamabu}}\newcommand{\montrerVdPrsDDuG}{\cacherGloses{Du-}montrer\cacherGloses{.VD}\cacherGloses{xPRS.D}}\newcommand{\montrerVdPrsDPl}{\strutgb{0pt}\grapho{xPnmI}}\newcommand{\montrerVdPrsDPlP}{\textipa{dupenamabu}}\newcommand{\montrerVdPrsDPlG}{\cacherGloses{Pl-}montrer\cacherGloses{.VD}\cacherGloses{xPRS.D}}\newcommand{\montrerVdPstASg}{\strutgb{0pt}\grapho{GIHnp}}\newcommand{\montrerVdPstASgP}{\textipa{pipuponaba}}\newcommand{\montrerVdPstASgG}{\cacherGloses{Sg-}montrer\cacherGloses{.VD}\cacherGloses{xPST.A}}\newcommand{\montrerVdPstADu}{\strutgb{0pt}\grapho{YIHnp}}\newcommand{\montrerVdPstADuP}{\textipa{sipuponaba}}\newcommand{\montrerVdPstADuG}{\cacherGloses{Du-}montrer\cacherGloses{.VD}\cacherGloses{xPST.A}}\newcommand{\montrerVdPstAPl}{\strutgb{0pt}\grapho{fIHnp}}\newcommand{\montrerVdPstAPlP}{\textipa{dipuponaba}}\newcommand{\montrerVdPstAPlG}{\cacherGloses{Pl-}montrer\cacherGloses{.VD}\cacherGloses{xPST.A}}\newcommand{\montrerVdPstBSg}{\strutgb{0pt}\grapho{pIHFp}}\newcommand{\montrerVdPstBSgP}{\textipa{papuponuba}}\newcommand{\montrerVdPstBSgG}{\cacherGloses{Sg-}montrer\cacherGloses{.VD}\cacherGloses{xPST.B}}\newcommand{\montrerVdPstBDu}{\strutgb{0pt}\grapho{sIHFp}}\newcommand{\montrerVdPstBDuP}{\textipa{sapuponuba}}\newcommand{\montrerVdPstBDuG}{\cacherGloses{Du-}montrer\cacherGloses{.VD}\cacherGloses{xPST.B}}\newcommand{\montrerVdPstBPl}{\strutgb{0pt}\grapho{dIHFp}}\newcommand{\montrerVdPstBPlP}{\textipa{dapuponuba}}\newcommand{\montrerVdPstBPlG}{\cacherGloses{Pl-}montrer\cacherGloses{.VD}\cacherGloses{xPST.B}}\newcommand{\montrerVdPstCSg}{\strutgb{0pt}\grapho{HIHEp}}\newcommand{\montrerVdPstCSgP}{\textipa{popuponoba}}\newcommand{\montrerVdPstCSgG}{\cacherGloses{Sg-}montrer\cacherGloses{.VD}\cacherGloses{xPST.C}}\newcommand{\montrerVdPstCDu}{\strutgb{0pt}\grapho{1IHEp}}\newcommand{\montrerVdPstCDuP}{\textipa{sopuponoba}}\newcommand{\montrerVdPstCDuG}{\cacherGloses{Du-}montrer\cacherGloses{.VD}\cacherGloses{xPST.C}}\newcommand{\montrerVdPstCPl}{\strutgb{0pt}\grapho{gIHEp}}\newcommand{\montrerVdPstCPlP}{\textipa{dopuponoba}}\newcommand{\montrerVdPstCPlG}{\cacherGloses{Pl-}montrer\cacherGloses{.VD}\cacherGloses{xPST.C}}\newcommand{\montrerVdPstDSg}{\strutgb{0pt}\grapho{PIHEp}}\newcommand{\montrerVdPstDSgP}{\textipa{pepuponoba}}\newcommand{\montrerVdPstDSgG}{\cacherGloses{Sg-}montrer\cacherGloses{.VD}\cacherGloses{xPST.D}}\newcommand{\montrerVdPstDDu}{\strutgb{0pt}\grapho{SIHEp}}\newcommand{\montrerVdPstDDuP}{\textipa{sepuponoba}}\newcommand{\montrerVdPstDDuG}{\cacherGloses{Du-}montrer\cacherGloses{.VD}\cacherGloses{xPST.D}}\newcommand{\montrerVdPstDPl}{\strutgb{0pt}\grapho{DIHEp}}\newcommand{\montrerVdPstDPlP}{\textipa{depuponoba}}\newcommand{\montrerVdPstDPlG}{\cacherGloses{Pl-}montrer\cacherGloses{.VD}\cacherGloses{xPST.D}}
%% version : __file__
%% traitement : 140914-1748
\begin{exe}
\ex\glll
   \DEFSgErg{}   \chasseurCSgErg{}    \INDSgDat{}   \NicoleBSgDat{}   \INDSgAbs{}   \coyoteCSgAbs{}  \offrirVdPstCSg{} \\
   \DEFSgErgP{}   \chasseurCSgErgP{}    \INDSgDatP{}   \NicoleBSgDatP{}   \INDSgAbsP{}   \coyoteCSgAbsP{}  \offrirVdPstCSgP{} \\
   \DEFSgErgG{}   \chasseurCSgErgG{}    \INDSgDatG{}   \NicoleBSgDatG{}   \INDSgAbsG{}   \coyoteCSgAbsG{}  \offrirVdPstCSgG{} \\
 Le chasseur offrait un coyote à Nicole
\ex\glll
   \DEFDuAbs{}   \autrucheBDuAbs{}    \DEFSgObl{}   \grandASg{}   \infirmiereASgObl{}   \AVEC{}  \dormirViPrsBDu{} \\
   \DEFDuAbsP{}   \autrucheBDuAbsP{}    \DEFSgOblP{}   \grandASgP{}   \infirmiereASgOblP{}   \AVECP{}  \dormirViPrsBDuP{} \\
   \DEFDuAbsG{}   \autrucheBDuAbsG{}    \DEFSgOblG{}   \grandASgG{}   \infirmiereASgOblG{}   \AVECG{}  \dormirViPrsBDuG{} \\
 Les deux autruches dorment avec la grande infirmière
\ex\glll
   \DEFDuAbs{}   \grandCDu{}   \noirCDu{}   \cafeCDuAbs{}    \DEFSgObl{}    \DEFSgObl{}   \grandCSg{}   \filleCSgObl{}   \DE{}   \petitDSg{}   \litDSgObl{}   \DANS{}  \tomberViPrsCDu{} \\
   \DEFDuAbsP{}   \grandCDuP{}   \noirCDuP{}   \cafeCDuAbsP{}    \DEFSgOblP{}    \DEFSgOblP{}   \grandCSgP{}   \filleCSgOblP{}   \DEP{}   \petitDSgP{}   \litDSgOblP{}   \DANSP{}  \tomberViPrsCDuP{} \\
   \DEFDuAbsG{}   \grandCDuG{}   \noirCDuG{}   \cafeCDuAbsG{}    \DEFSgOblG{}    \DEFSgOblG{}   \grandCSgG{}   \filleCSgOblG{}   \DEG{}   \petitDSgG{}   \litDSgOblG{}   \DANSG{}  \tomberViPrsCDuG{} \\
 Les deux grands cafés noirs tombent dans le petit lit de la grande fille
\ex\glll
   \INDSgErg{}   \grosCSg{}   \blancCSg{}   \chasseurCSgErg{}   \INDPlAbs{}   \noirCPl{}   \cafeCPlAbs{}  \acheterVtPrsCPl{} \\
   \INDSgErgP{}   \grosCSgP{}   \blancCSgP{}   \chasseurCSgErgP{}   \INDPlAbsP{}   \noirCPlP{}   \cafeCPlAbsP{}  \acheterVtPrsCPlP{} \\
   \INDSgErgG{}   \grosCSgG{}   \blancCSgG{}   \chasseurCSgErgG{}   \INDPlAbsG{}   \noirCPlG{}   \cafeCPlAbsG{}  \acheterVtPrsCPlG{} \\
 Un gros chasseur blanc achète des cafés noirs
\ex\glll
   \DEFPlErg{}   \troisDPl{}   \noirDPl{}   \chatDPlErg{}   \DEMDuAbs{}   \blancBDu{}   \sourisBDuAbs{}  \mangerVtPrsBDu{} \\
   \DEFPlErgP{}   \troisDPlP{}   \noirDPlP{}   \chatDPlErgP{}   \DEMDuAbsP{}   \blancBDuP{}   \sourisBDuAbsP{}  \mangerVtPrsBDuP{} \\
   \DEFPlErgG{}   \troisDPlG{}   \noirDPlG{}   \chatDPlErgG{}   \DEMDuAbsG{}   \blancBDuG{}   \sourisBDuAbsG{}  \mangerVtPrsBDuG{} \\
 Les trois chats noirs mangent ces deux souris blanches
\ex\glll
    \DEFSgObl{}   \litDSgObl{}   \SOUS{}   \INDDuErg{}   \sourisBDuErg{}    \INDSgDat{}   \autrucheBSgDat{}   \INDSgAbs{}   \oeufCSgAbs{}  \lancerVdPrsCSg{} \\
    \DEFSgOblP{}   \litDSgOblP{}   \SOUSP{}   \INDDuErgP{}   \sourisBDuErgP{}    \INDSgDatP{}   \autrucheBSgDatP{}   \INDSgAbsP{}   \oeufCSgAbsP{}  \lancerVdPrsCSgP{} \\
    \DEFSgOblG{}   \litDSgOblG{}   \SOUSG{}   \INDDuErgG{}   \sourisBDuErgG{}    \INDSgDatG{}   \autrucheBSgDatG{}   \INDSgAbsG{}   \oeufCSgAbsG{}  \lancerVdPrsCSgG{} \\
 Deux souris lancent un oeuf à une autruche sous le lit
\ex\glll
   \DEFSgErg{}   \litDSgErg{}   \INDDuAbs{}   \grosDDu{}   \blancDDu{}   \chatDDuAbs{}  \supporterVtPstDDu{} \\
   \DEFSgErgP{}   \litDSgErgP{}   \INDDuAbsP{}   \grosDDuP{}   \blancDDuP{}   \chatDDuAbsP{}  \supporterVtPstDDuP{} \\
   \DEFSgErgG{}   \litDSgErgG{}   \INDDuAbsG{}   \grosDDuG{}   \blancDDuG{}   \chatDDuAbsG{}  \supporterVtPstDDuG{} \\
 Le lit supportait deux gros chats blancs
\ex\glll
   \INDDuErg{}    \DEMSgObl{}   \maisonDSgObl{}   \DE{}   \chasseurCDuErg{}   \DEFPlAbs{}    \INDSgObl{}   \NicoleBSgObl{}   \DE{}   \quatreCPl{}   \oeufCPlAbs{}  \acheterVtPstCPl{} \\
   \INDDuErgP{}    \DEMSgOblP{}   \maisonDSgOblP{}   \DEP{}   \chasseurCDuErgP{}   \DEFPlAbsP{}    \INDSgOblP{}   \NicoleBSgOblP{}   \DEP{}   \quatreCPlP{}   \oeufCPlAbsP{}  \acheterVtPstCPlP{} \\
   \INDDuErgG{}    \DEMSgOblG{}   \maisonDSgOblG{}   \DEG{}   \chasseurCDuErgG{}   \DEFPlAbsG{}    \INDSgOblG{}   \NicoleBSgOblG{}   \DEG{}   \quatreCPlG{}   \oeufCPlAbsG{}  \acheterVtPstCPlG{} \\
 Deux chasseurs de cette maison achetaient les quatre oeufs de Nicole
\ex\glll
   \INDSgAbs{}   \jauneDSg{}   \chatDSgAbs{}    \DEFSgObl{}   \maisonDSgObl{}   \DEVANT{}  \dormirViPstDSg{} \\
   \INDSgAbsP{}   \jauneDSgP{}   \chatDSgAbsP{}    \DEFSgOblP{}   \maisonDSgOblP{}   \DEVANTP{}  \dormirViPstDSgP{} \\
   \INDSgAbsG{}   \jauneDSgG{}   \chatDSgAbsG{}    \DEFSgOblG{}   \maisonDSgOblG{}   \DEVANTG{}  \dormirViPstDSgG{} \\
 Un chat jaune dormait devant la maison
\ex\glll
    \INDSgObl{}   \NicoleBSgObl{}   \AVEC{}   \INDDuAbs{}   \petitCDu{}   \chasseurCDuAbs{}  \arriverViPrsCDu{} \\
    \INDSgOblP{}   \NicoleBSgOblP{}   \AVECP{}   \INDDuAbsP{}   \petitCDuP{}   \chasseurCDuAbsP{}  \arriverViPrsCDuP{} \\
    \INDSgOblG{}   \NicoleBSgOblG{}   \AVECG{}   \INDDuAbsG{}   \petitCDuG{}   \chasseurCDuAbsG{}  \arriverViPrsCDuG{} \\
 Deux petits chasseurs arrivent avec Nicole
\ex\glll
   \INDSgErg{}   \NabilDSgErg{}    \DEFSgObl{}   \chambreBSgObl{}   \POUR{}   \INDSgAbs{}   \litDSgAbs{}  \acheterVtPstDSg{} \\
   \INDSgErgP{}   \NabilDSgErgP{}    \DEFSgOblP{}   \chambreBSgOblP{}   \POURP{}   \INDSgAbsP{}   \litDSgAbsP{}  \acheterVtPstDSgP{} \\
   \INDSgErgG{}   \NabilDSgErgG{}    \DEFSgOblG{}   \chambreBSgOblG{}   \POURG{}   \INDSgAbsG{}   \litDSgAbsG{}  \acheterVtPstDSgG{} \\
 Nabil achetait un lit pour la chambre
\ex\glll
   \DEMDuAbs{}   \grandDDu{}   \garconDDuAbs{}    \DEFSgObl{}   \litDSgObl{}   \SOUS{}  \dormirViPrsDDu{} \\
   \DEMDuAbsP{}   \grandDDuP{}   \garconDDuAbsP{}    \DEFSgOblP{}   \litDSgOblP{}   \SOUSP{}  \dormirViPrsDDuP{} \\
   \DEMDuAbsG{}   \grandDDuG{}   \garconDDuAbsG{}    \DEFSgOblG{}   \litDSgOblG{}   \SOUSG{}  \dormirViPrsDDuG{} \\
 Ces deux grands garçons dorment sous le lit
\ex\glll
   \INDPlErg{}   \grandDPl{}   \garconDPlErg{}    \DEFPlObl{}    \DEFSgObl{}   \maisonDSgObl{}   \DE{}   \filleCPlObl{}   \A{}   \INDPlAbs{}   \sourisBPlAbs{}  \montrerVdPstBPl{} \\
   \INDPlErgP{}   \grandDPlP{}   \garconDPlErgP{}    \DEFPlOblP{}    \DEFSgOblP{}   \maisonDSgOblP{}   \DEP{}   \filleCPlOblP{}   \AP{}   \INDPlAbsP{}   \sourisBPlAbsP{}  \montrerVdPstBPlP{} \\
   \INDPlErgG{}   \grandDPlG{}   \garconDPlErgG{}    \DEFPlOblG{}    \DEFSgOblG{}   \maisonDSgOblG{}   \DEG{}   \filleCPlOblG{}   \AG{}   \INDPlAbsG{}   \sourisBPlAbsG{}  \montrerVdPstBPlG{} \\
 Des grands garçons montraient des souris aux filles de la maison
\ex\glll
   \DEFPlErg{}    \DEFSgObl{}   \grandASg{}   \plaineASgObl{}   \DE{}   \coyoteCPlErg{}    \DEFPlObl{}   \chatDPlObl{}   \A{}   \DEFPlAbs{}   \oeufCPlAbs{}  \montrerVdPrsCPl{} \\
   \DEFPlErgP{}    \DEFSgOblP{}   \grandASgP{}   \plaineASgOblP{}   \DEP{}   \coyoteCPlErgP{}    \DEFPlOblP{}   \chatDPlOblP{}   \AP{}   \DEFPlAbsP{}   \oeufCPlAbsP{}  \montrerVdPrsCPlP{} \\
   \DEFPlErgG{}    \DEFSgOblG{}   \grandASgG{}   \plaineASgOblG{}   \DEG{}   \coyoteCPlErgG{}    \DEFPlOblG{}   \chatDPlOblG{}   \AG{}   \DEFPlAbsG{}   \oeufCPlAbsG{}  \montrerVdPrsCPlG{} \\
 Les coyotes de la grande plaine montrent les oeufs aux chats
\ex\glll
    \INDSgObl{}   \NicoleBSgObl{}   \DEVANT{}   \DEFPlAbs{}    \DEFPlObl{}   \autrucheBPlObl{}   \DE{}   \oeufCPlAbs{}    \DEFSgObl{}   \maisonDSgObl{}   \SOUS{}  \tomberViPstCPl{} \\
    \INDSgOblP{}   \NicoleBSgOblP{}   \DEVANTP{}   \DEFPlAbsP{}    \DEFPlOblP{}   \autrucheBPlOblP{}   \DEP{}   \oeufCPlAbsP{}    \DEFSgOblP{}   \maisonDSgOblP{}   \SOUSP{}  \tomberViPstCPlP{} \\
    \INDSgOblG{}   \NicoleBSgOblG{}   \DEVANTG{}   \DEFPlAbsG{}    \DEFPlOblG{}   \autrucheBPlOblG{}   \DEG{}   \oeufCPlAbsG{}    \DEFSgOblG{}   \maisonDSgOblG{}   \SOUSG{}  \tomberViPstCPlG{} \\
 Les oeufs des autruches tombaient sous la maison devant Nicole
\ex\glll
   \DEMDuErg{}   \grandCDu{}   \maigreCDu{}   \chasseurCDuErg{}    \INDPlObl{}   \coussinBPlObl{}   \AVEC{}   \DEFPlAbs{}   \litDPlAbs{}  \supporterVtPrsDPl{} \\
   \DEMDuErgP{}   \grandCDuP{}   \maigreCDuP{}   \chasseurCDuErgP{}    \INDPlOblP{}   \coussinBPlOblP{}   \AVECP{}   \DEFPlAbsP{}   \litDPlAbsP{}  \supporterVtPrsDPlP{} \\
   \DEMDuErgG{}   \grandCDuG{}   \maigreCDuG{}   \chasseurCDuErgG{}    \INDPlOblG{}   \coussinBPlOblG{}   \AVECG{}   \DEFPlAbsG{}   \litDPlAbsG{}  \supporterVtPrsDPlG{} \\
 Ces deux grands chasseurs maigres supportent les lits avec des coussins
\ex\glll
    \DEFSgObl{}   \cuisineDSgObl{}   \DANS{}   \DEFSgErg{}    \DEFSgObl{}   \chasseurCSgObl{}   \DE{}   \filleCSgErg{}    \DEFPlObl{}    \DEFSgObl{}   \plaineASgObl{}   \DE{}   \infirmiereAPlObl{}   \A{}   \INDDuAbs{}   \autrucheBDuAbs{}  \donnerVdPrsBDu{} \\
    \DEFSgOblP{}   \cuisineDSgOblP{}   \DANSP{}   \DEFSgErgP{}    \DEFSgOblP{}   \chasseurCSgOblP{}   \DEP{}   \filleCSgErgP{}    \DEFPlOblP{}    \DEFSgOblP{}   \plaineASgOblP{}   \DEP{}   \infirmiereAPlOblP{}   \AP{}   \INDDuAbsP{}   \autrucheBDuAbsP{}  \donnerVdPrsBDuP{} \\
    \DEFSgOblG{}   \cuisineDSgOblG{}   \DANSG{}   \DEFSgErgG{}    \DEFSgOblG{}   \chasseurCSgOblG{}   \DEG{}   \filleCSgErgG{}    \DEFPlOblG{}    \DEFSgOblG{}   \plaineASgOblG{}   \DEG{}   \infirmiereAPlOblG{}   \AG{}   \INDDuAbsG{}   \autrucheBDuAbsG{}  \donnerVdPrsBDuG{} \\
 La fille du chasseur donne deux autruches aux infirmières de la plaine dans la cuisine
\ex\glll
    \DEMSgObl{}   \tableDSgObl{}   \SOUS{}   \INDDuErg{}   \sourisBDuErg{}   \INDPlAbs{}   \theBPlAbs{}  \boireVtPstBPl{} \\
    \DEMSgOblP{}   \tableDSgOblP{}   \SOUSP{}   \INDDuErgP{}   \sourisBDuErgP{}   \INDPlAbsP{}   \theBPlAbsP{}  \boireVtPstBPlP{} \\
    \DEMSgOblG{}   \tableDSgOblG{}   \SOUSG{}   \INDDuErgG{}   \sourisBDuErgG{}   \INDPlAbsG{}   \theBPlAbsG{}  \boireVtPstBPlG{} \\
 Deux souris buvaient des thés sous cette table
\ex\glll
    \DEFSgObl{}   \maisonDSgObl{}   \DE{}   \DEFPlErg{}    \DEFSgObl{}   \villageCSgObl{}   \DE{}   \infirmiereAPlErg{}    \DEFPlObl{}   \chasseurCPlObl{}   \A{}   \INDPlAbs{}   \fruitAPlAbs{}  \donnerVdPrsAPl{} \\
    \DEFSgOblP{}   \maisonDSgOblP{}   \DEP{}   \DEFPlErgP{}    \DEFSgOblP{}   \villageCSgOblP{}   \DEP{}   \infirmiereAPlErgP{}    \DEFPlOblP{}   \chasseurCPlOblP{}   \AP{}   \INDPlAbsP{}   \fruitAPlAbsP{}  \donnerVdPrsAPlP{} \\
    \DEFSgOblG{}   \maisonDSgOblG{}   \DEG{}   \DEFPlErgG{}    \DEFSgOblG{}   \villageCSgOblG{}   \DEG{}   \infirmiereAPlErgG{}    \DEFPlOblG{}   \chasseurCPlOblG{}   \AG{}   \INDPlAbsG{}   \fruitAPlAbsG{}  \donnerVdPrsAPlG{} \\
 Les infirmières du village donnent des fruits aux chasseurs de la maison
\ex\glll
   \DEFPlAbs{}    \DEFSgObl{}   \villageCSgObl{}   \DE{}   \sourisBPlAbs{}    \DEFSgObl{}   \chatDSgObl{}   \DEVANT{}  \dormirViPrsBPl{} \\
   \DEFPlAbsP{}    \DEFSgOblP{}   \villageCSgOblP{}   \DEP{}   \sourisBPlAbsP{}    \DEFSgOblP{}   \chatDSgOblP{}   \DEVANTP{}  \dormirViPrsBPlP{} \\
   \DEFPlAbsG{}    \DEFSgOblG{}   \villageCSgOblG{}   \DEG{}   \sourisBPlAbsG{}    \DEFSgOblG{}   \chatDSgOblG{}   \DEVANTG{}  \dormirViPrsBPlG{} \\
 Les souris du village dorment devant le chat
\ex\glll
   \DEFPlAbs{}    \DEFPlObl{}   \grandCPl{}   \maigreCPl{}   \filleCPlObl{}   \DE{}   \petitDPl{}   \litDPlAbs{}    \DEFSgObl{}   \chambreBSgObl{}   \DANS{}  \arriverViPrsDPl{} \\
   \DEFPlAbsP{}    \DEFPlOblP{}   \grandCPlP{}   \maigreCPlP{}   \filleCPlOblP{}   \DEP{}   \petitDPlP{}   \litDPlAbsP{}    \DEFSgOblP{}   \chambreBSgOblP{}   \DANSP{}  \arriverViPrsDPlP{} \\
   \DEFPlAbsG{}    \DEFPlOblG{}   \grandCPlG{}   \maigreCPlG{}   \filleCPlOblG{}   \DEG{}   \petitDPlG{}   \litDPlAbsG{}    \DEFSgOblG{}   \chambreBSgOblG{}   \DANSG{}  \arriverViPrsDPlG{} \\
 Les petits lits des grandes filles maigres arrivent dans la chambre
\ex\glll
   \INDSgErg{}   \coussinBSgErg{}   \DEFSgAbs{}   \chatDSgAbs{}  \supporterVtPrsDSg{} \\
   \INDSgErgP{}   \coussinBSgErgP{}   \DEFSgAbsP{}   \chatDSgAbsP{}  \supporterVtPrsDSgP{} \\
   \INDSgErgG{}   \coussinBSgErgG{}   \DEFSgAbsG{}   \chatDSgAbsG{}  \supporterVtPrsDSgG{} \\
 Un coussin supporte le chat
\ex\glll
   \DEFDuAbs{}   \petitCDu{}   \oeufCDuAbs{}    \DEFSgObl{}   \grandBSg{}   \autrucheBSgObl{}   \SOUS{}  \arriverViPrsCDu{} \\
   \DEFDuAbsP{}   \petitCDuP{}   \oeufCDuAbsP{}    \DEFSgOblP{}   \grandBSgP{}   \autrucheBSgOblP{}   \SOUSP{}  \arriverViPrsCDuP{} \\
   \DEFDuAbsG{}   \petitCDuG{}   \oeufCDuAbsG{}    \DEFSgOblG{}   \grandBSgG{}   \autrucheBSgOblG{}   \SOUSG{}  \arriverViPrsCDuG{} \\
 Les deux petits oeufs arrivent sous la grande autruche
\ex\glll
    \DEFSgObl{}   \cuisineDSgObl{}   \DANS{}   \DEMPlErg{}   \troisDPl{}   \chatDPlErg{}   \DEFSgAbs{}    \INDSgObl{}   \KatishaASgObl{}   \DE{}   \theBSgAbs{}  \boireVtPstBSg{} \\
    \DEFSgOblP{}   \cuisineDSgOblP{}   \DANSP{}   \DEMPlErgP{}   \troisDPlP{}   \chatDPlErgP{}   \DEFSgAbsP{}    \INDSgOblP{}   \KatishaASgOblP{}   \DEP{}   \theBSgAbsP{}  \boireVtPstBSgP{} \\
    \DEFSgOblG{}   \cuisineDSgOblG{}   \DANSG{}   \DEMPlErgG{}   \troisDPlG{}   \chatDPlErgG{}   \DEFSgAbsG{}    \INDSgOblG{}   \KatishaASgOblG{}   \DEG{}   \theBSgAbsG{}  \boireVtPstBSgG{} \\
 Ces trois chats buvaient le thé de Katisha dans la cuisine
\ex\glll
    \INDPlObl{}   \coussinBPlObl{}   \SUR{}   \DEFPlErg{}   \infirmiereAPlErg{}   \INDPlAbs{}   \oeufCPlAbs{}  \mangerVtPrsCPl{} \\
    \INDPlOblP{}   \coussinBPlOblP{}   \SURP{}   \DEFPlErgP{}   \infirmiereAPlErgP{}   \INDPlAbsP{}   \oeufCPlAbsP{}  \mangerVtPrsCPlP{} \\
    \INDPlOblG{}   \coussinBPlOblG{}   \SURG{}   \DEFPlErgG{}   \infirmiereAPlErgG{}   \INDPlAbsG{}   \oeufCPlAbsG{}  \mangerVtPrsCPlG{} \\
 Les infirmières mangent des oeufs sur des coussins
\ex\glll
    \INDPlObl{}   \jauneAPl{}   \balaiAPlObl{}   \AVEC{}   \DEFPlErg{}   \troisCPl{}   \filleCPlErg{}   \DEMDuAbs{}    \DEFSgObl{}   \rougeDSg{}   \maisonDSgObl{}   \DEVANT{}   \coyoteCDuAbs{}  \chasserVtPrsCDu{} \\
    \INDPlOblP{}   \jauneAPlP{}   \balaiAPlOblP{}   \AVECP{}   \DEFPlErgP{}   \troisCPlP{}   \filleCPlErgP{}   \DEMDuAbsP{}    \DEFSgOblP{}   \rougeDSgP{}   \maisonDSgOblP{}   \DEVANTP{}   \coyoteCDuAbsP{}  \chasserVtPrsCDuP{} \\
    \INDPlOblG{}   \jauneAPlG{}   \balaiAPlOblG{}   \AVECG{}   \DEFPlErgG{}   \troisCPlG{}   \filleCPlErgG{}   \DEMDuAbsG{}    \DEFSgOblG{}   \rougeDSgG{}   \maisonDSgOblG{}   \DEVANTG{}   \coyoteCDuAbsG{}  \chasserVtPrsCDuG{} \\
 Les trois filles chassent ces deux coyotes devant la maison rouge avec des balais jaunes
\ex\glll
   \DEFPlErg{}   \chambreBPlErg{}   \INDPlAbs{}   \litDPlAbs{}  \supporterVtPrsDPl{} \\
   \DEFPlErgP{}   \chambreBPlErgP{}   \INDPlAbsP{}   \litDPlAbsP{}  \supporterVtPrsDPlP{} \\
   \DEFPlErgG{}   \chambreBPlErgG{}   \INDPlAbsG{}   \litDPlAbsG{}  \supporterVtPrsDPlG{} \\
 Les chambres supportent des lits
\ex\glll
   \DEMDuErg{}   \infirmiereADuErg{}   \INDDuAbs{}   \oeufCDuAbs{}  \mangerVtPstCDu{} \\
   \DEMDuErgP{}   \infirmiereADuErgP{}   \INDDuAbsP{}   \oeufCDuAbsP{}  \mangerVtPstCDuP{} \\
   \DEMDuErgG{}   \infirmiereADuErgG{}   \INDDuAbsG{}   \oeufCDuAbsG{}  \mangerVtPstCDuG{} \\
 Ces deux infirmières mangeaient deux oeufs
\ex\glll
   \INDSgErg{}   \NabilDSgErg{}    \DEFSgDat{}   \grandASg{}   \infirmiereASgDat{}   \INDSgAbs{}   \petitDSg{}   \litDSgAbs{}  \donnerVdPstDSg{} \\
   \INDSgErgP{}   \NabilDSgErgP{}    \DEFSgDatP{}   \grandASgP{}   \infirmiereASgDatP{}   \INDSgAbsP{}   \petitDSgP{}   \litDSgAbsP{}  \donnerVdPstDSgP{} \\
   \INDSgErgG{}   \NabilDSgErgG{}    \DEFSgDatG{}   \grandASgG{}   \infirmiereASgDatG{}   \INDSgAbsG{}   \petitDSgG{}   \litDSgAbsG{}  \donnerVdPstDSgG{} \\
 Nabil donnait un petit lit à la grande infirmière
\ex\glll
   \INDDuAbs{}   \cafeCDuAbs{}    \DEFPlObl{}    \DEFSgObl{}   \chambreBSgObl{}   \DE{}   \coussinBPlObl{}   \SUR{}  \tomberViPstCDu{} \\
   \INDDuAbsP{}   \cafeCDuAbsP{}    \DEFPlOblP{}    \DEFSgOblP{}   \chambreBSgOblP{}   \DEP{}   \coussinBPlOblP{}   \SURP{}  \tomberViPstCDuP{} \\
   \INDDuAbsG{}   \cafeCDuAbsG{}    \DEFPlOblG{}    \DEFSgOblG{}   \chambreBSgOblG{}   \DEG{}   \coussinBPlOblG{}   \SURG{}  \tomberViPstCDuG{} \\
 Deux cafés tombaient sur les coussins de la chambre
\ex\glll
   \INDSgAbs{}   \NicoleBSgAbs{}    \DEFPlObl{}    \DEFSgObl{}   \litDSgObl{}   \DE{}   \coussinBPlObl{}   \SUR{}  \tomberViPrsBSg{} \\
   \INDSgAbsP{}   \NicoleBSgAbsP{}    \DEFPlOblP{}    \DEFSgOblP{}   \litDSgOblP{}   \DEP{}   \coussinBPlOblP{}   \SURP{}  \tomberViPrsBSgP{} \\
   \INDSgAbsG{}   \NicoleBSgAbsG{}    \DEFPlOblG{}    \DEFSgOblG{}   \litDSgOblG{}   \DEG{}   \coussinBPlOblG{}   \SURG{}  \tomberViPrsBSgG{} \\
 Nicole tombe sur les coussins du lit
\ex\glll
   \INDSgAbs{}   \NabilDSgAbs{}    \DEFSgObl{}    \DEMSgObl{}   \petitASg{}   \infirmiereASgObl{}   \DE{}   \litDSgObl{}   \DANS{}  \tomberViPrsDSg{} \\
   \INDSgAbsP{}   \NabilDSgAbsP{}    \DEFSgOblP{}    \DEMSgOblP{}   \petitASgP{}   \infirmiereASgOblP{}   \DEP{}   \litDSgOblP{}   \DANSP{}  \tomberViPrsDSgP{} \\
   \INDSgAbsG{}   \NabilDSgAbsG{}    \DEFSgOblG{}    \DEMSgOblG{}   \petitASgG{}   \infirmiereASgOblG{}   \DEG{}   \litDSgOblG{}   \DANSG{}  \tomberViPrsDSgG{} \\
 Nabil tombe dans le lit de cette petite infirmière
\ex\glll
   \DEFSgAbs{}    \INDSgObl{}   \NicoleBSgObl{}   \DE{}   \filleCSgAbs{}    \DEFSgObl{}    \INDSgObl{}   \NabilDSgObl{}   \DE{}   \chambreBSgObl{}   \DANS{}  \dormirViPrsCSg{} \\
   \DEFSgAbsP{}    \INDSgOblP{}   \NicoleBSgOblP{}   \DEP{}   \filleCSgAbsP{}    \DEFSgOblP{}    \INDSgOblP{}   \NabilDSgOblP{}   \DEP{}   \chambreBSgOblP{}   \DANSP{}  \dormirViPrsCSgP{} \\
   \DEFSgAbsG{}    \INDSgOblG{}   \NicoleBSgOblG{}   \DEG{}   \filleCSgAbsG{}    \DEFSgOblG{}    \INDSgOblG{}   \NabilDSgOblG{}   \DEG{}   \chambreBSgOblG{}   \DANSG{}  \dormirViPrsCSgG{} \\
 La fille de Nicole dort dans la chambre de Nabil
\ex\glll
   \DEFPlErg{}   \infirmiereAPlErg{}    \DEFDuObl{}    \INDPlObl{}   \sourisBPlObl{}   \DE{}   \chasseurCDuObl{}   \A{}   \INDSgAbs{}   \cafeCSgAbs{}  \offrirVdPstCSg{} \\
   \DEFPlErgP{}   \infirmiereAPlErgP{}    \DEFDuOblP{}    \INDPlOblP{}   \sourisBPlOblP{}   \DEP{}   \chasseurCDuOblP{}   \AP{}   \INDSgAbsP{}   \cafeCSgAbsP{}  \offrirVdPstCSgP{} \\
   \DEFPlErgG{}   \infirmiereAPlErgG{}    \DEFDuOblG{}    \INDPlOblG{}   \sourisBPlOblG{}   \DEG{}   \chasseurCDuOblG{}   \AG{}   \INDSgAbsG{}   \cafeCSgAbsG{}  \offrirVdPstCSgG{} \\
 Les infirmières offraient un café aux deux chasseurs de souris
\ex\glll
   \INDSgAbs{}   \garconDSgAbs{}    \DEFSgObl{}    \DEFDuObl{}   \filleCDuObl{}   \DE{}   \chambreBSgObl{}   \DANS{}  \arriverViPrsDSg{} \\
   \INDSgAbsP{}   \garconDSgAbsP{}    \DEFSgOblP{}    \DEFDuOblP{}   \filleCDuOblP{}   \DEP{}   \chambreBSgOblP{}   \DANSP{}  \arriverViPrsDSgP{} \\
   \INDSgAbsG{}   \garconDSgAbsG{}    \DEFSgOblG{}    \DEFDuOblG{}   \filleCDuOblG{}   \DEG{}   \chambreBSgOblG{}   \DANSG{}  \arriverViPrsDSgG{} \\
 Un garçon arrive dans la chambre des deux filles
\ex\glll
    \INDDuObl{}   \grosDDu{}   \tableDDuObl{}   \SUR{}   \DEFPlAbs{}   \blancDPl{}   \chatDPlAbs{}  \dormirViPrsDPl{} \\
    \INDDuOblP{}   \grosDDuP{}   \tableDDuOblP{}   \SURP{}   \DEFPlAbsP{}   \blancDPlP{}   \chatDPlAbsP{}  \dormirViPrsDPlP{} \\
    \INDDuOblG{}   \grosDDuG{}   \tableDDuOblG{}   \SURG{}   \DEFPlAbsG{}   \blancDPlG{}   \chatDPlAbsG{}  \dormirViPrsDPlG{} \\
 Les chats blancs dorment sur deux grosses tables
\ex\glll
    \INDDuObl{}   \garconDDuObl{}   \AVEC{}   \INDDuAbs{}   \filleCDuAbs{}  \arriverViPrsCDu{} \\
    \INDDuOblP{}   \garconDDuOblP{}   \AVECP{}   \INDDuAbsP{}   \filleCDuAbsP{}  \arriverViPrsCDuP{} \\
    \INDDuOblG{}   \garconDDuOblG{}   \AVECG{}   \INDDuAbsG{}   \filleCDuAbsG{}  \arriverViPrsCDuG{} \\
 Deux filles arrivent avec deux garçons
\ex\glll
   \DEFPlAbs{}    \INDSgObl{}   \KatishaASgObl{}   \DE{}   \filleCPlAbs{}    \INDSgObl{}   \blancBSg{}   \sourisBSgObl{}   \AVEC{}  \dormirViPrsCPl{} \\
   \DEFPlAbsP{}    \INDSgOblP{}   \KatishaASgOblP{}   \DEP{}   \filleCPlAbsP{}    \INDSgOblP{}   \blancBSgP{}   \sourisBSgOblP{}   \AVECP{}  \dormirViPrsCPlP{} \\
   \DEFPlAbsG{}    \INDSgOblG{}   \KatishaASgOblG{}   \DEG{}   \filleCPlAbsG{}    \INDSgOblG{}   \blancBSgG{}   \sourisBSgOblG{}   \AVECG{}  \dormirViPrsCPlG{} \\
 Les filles de Katisha dorment avec une souris blanche
\ex\glll
   \INDDuAbs{}   \jauneCDu{}   \coyoteCDuAbs{}    \DEFSgObl{}   \grandASg{}   \plaineASgObl{}   \DANS{}  \arriverViPstCDu{} \\
   \INDDuAbsP{}   \jauneCDuP{}   \coyoteCDuAbsP{}    \DEFSgOblP{}   \grandASgP{}   \plaineASgOblP{}   \DANSP{}  \arriverViPstCDuP{} \\
   \INDDuAbsG{}   \jauneCDuG{}   \coyoteCDuAbsG{}    \DEFSgOblG{}   \grandASgG{}   \plaineASgOblG{}   \DANSG{}  \arriverViPstCDuG{} \\
 Deux coyotes jaunes arrivaient dans la grande plaine
\ex\glll
   \INDSgErg{}   \garconDSgErg{}    \DEFPlObl{}    \DEFSgObl{}   \plaineASgObl{}   \DE{}   \coyoteCPlObl{}   \A{}   \INDSgAbs{}   \jauneASg{}   \balaiASgAbs{}  \lancerVdPstASg{} \\
   \INDSgErgP{}   \garconDSgErgP{}    \DEFPlOblP{}    \DEFSgOblP{}   \plaineASgOblP{}   \DEP{}   \coyoteCPlOblP{}   \AP{}   \INDSgAbsP{}   \jauneASgP{}   \balaiASgAbsP{}  \lancerVdPstASgP{} \\
   \INDSgErgG{}   \garconDSgErgG{}    \DEFPlOblG{}    \DEFSgOblG{}   \plaineASgOblG{}   \DEG{}   \coyoteCPlOblG{}   \AG{}   \INDSgAbsG{}   \jauneASgG{}   \balaiASgAbsG{}  \lancerVdPstASgG{} \\
 Un garçon lançait un balai jaune aux coyotes de la plaine
\ex\glll
    \INDDuObl{}   \petitCDu{}   \oeufCDuObl{}   \AVEC{}   \DEMSgAbs{}   \autrucheBSgAbs{}    \INDSgObl{}   \coussinBSgObl{}   \SUR{}  \tomberViPrsBSg{} \\
    \INDDuOblP{}   \petitCDuP{}   \oeufCDuOblP{}   \AVECP{}   \DEMSgAbsP{}   \autrucheBSgAbsP{}    \INDSgOblP{}   \coussinBSgOblP{}   \SURP{}  \tomberViPrsBSgP{} \\
    \INDDuOblG{}   \petitCDuG{}   \oeufCDuOblG{}   \AVECG{}   \DEMSgAbsG{}   \autrucheBSgAbsG{}    \INDSgOblG{}   \coussinBSgOblG{}   \SURG{}  \tomberViPrsBSgG{} \\
 Cette autruche tombe sur un coussin avec deux petits oeufs
\ex\glll
   \DEFPlAbs{}   \petitBPl{}   \sourisBPlAbs{}    \INDSgObl{}   \noirDSg{}   \chatDSgObl{}   \SUR{}  \tomberViPrsBPl{} \\
   \DEFPlAbsP{}   \petitBPlP{}   \sourisBPlAbsP{}    \INDSgOblP{}   \noirDSgP{}   \chatDSgOblP{}   \SURP{}  \tomberViPrsBPlP{} \\
   \DEFPlAbsG{}   \petitBPlG{}   \sourisBPlAbsG{}    \INDSgOblG{}   \noirDSgG{}   \chatDSgOblG{}   \SURG{}  \tomberViPrsBPlG{} \\
 Les petites souris tombent sur un chat noir
\ex\glll
   \DEMDuErg{}   \coyoteCDuErg{}   \INDPlAbs{}   \autrucheBPlAbs{}  \chasserVtPstBPl{} \\
   \DEMDuErgP{}   \coyoteCDuErgP{}   \INDPlAbsP{}   \autrucheBPlAbsP{}  \chasserVtPstBPlP{} \\
   \DEMDuErgG{}   \coyoteCDuErgG{}   \INDPlAbsG{}   \autrucheBPlAbsG{}  \chasserVtPstBPlG{} \\
 Ces deux coyotes chassaient des autruches
\ex\glll
   \DEFPlErg{}   \coyoteCPlErg{}    \DEMPlDat{}   \chatDPlDat{}   \INDSgAbs{}   \sourisBSgAbs{}  \lancerVdPstBSg{} \\
   \DEFPlErgP{}   \coyoteCPlErgP{}    \DEMPlDatP{}   \chatDPlDatP{}   \INDSgAbsP{}   \sourisBSgAbsP{}  \lancerVdPstBSgP{} \\
   \DEFPlErgG{}   \coyoteCPlErgG{}    \DEMPlDatG{}   \chatDPlDatG{}   \INDSgAbsG{}   \sourisBSgAbsG{}  \lancerVdPstBSgG{} \\
 Les coyotes lançaient une souris à ces chats
\ex\glll
   \DEFSgErg{}   \petitDSg{}   \blancDSg{}   \litDSgErg{}   \INDPlAbs{}   \grosBPl{}   \noirBPl{}   \coussinBPlAbs{}  \supporterVtPrsBPl{} \\
   \DEFSgErgP{}   \petitDSgP{}   \blancDSgP{}   \litDSgErgP{}   \INDPlAbsP{}   \grosBPlP{}   \noirBPlP{}   \coussinBPlAbsP{}  \supporterVtPrsBPlP{} \\
   \DEFSgErgG{}   \petitDSgG{}   \blancDSgG{}   \litDSgErgG{}   \INDPlAbsG{}   \grosBPlG{}   \noirBPlG{}   \coussinBPlAbsG{}  \supporterVtPrsBPlG{} \\
 Le petit lit blanc supporte des gros coussins noirs
\ex\glll
   \DEMSgErg{}   \petitDSg{}   \garconDSgErg{}   \DEFDuAbs{}    \DEMSgObl{}   \autrucheBSgObl{}   \DE{}   \oeufCDuAbs{}  \mangerVtPrsCDu{} \\
   \DEMSgErgP{}   \petitDSgP{}   \garconDSgErgP{}   \DEFDuAbsP{}    \DEMSgOblP{}   \autrucheBSgOblP{}   \DEP{}   \oeufCDuAbsP{}  \mangerVtPrsCDuP{} \\
   \DEMSgErgG{}   \petitDSgG{}   \garconDSgErgG{}   \DEFDuAbsG{}    \DEMSgOblG{}   \autrucheBSgOblG{}   \DEG{}   \oeufCDuAbsG{}  \mangerVtPrsCDuG{} \\
 Ce petit garçon mange les deux oeufs de cette autruche
\ex\glll
    \DEFSgObl{}   \chambreBSgObl{}   \DANS{}   \INDPlErg{}   \petitCPl{}   \chasseurCPlErg{}   \INDSgAbs{}   \noirCSg{}   \oeufCSgAbs{}  \mangerVtPrsCSg{} \\
    \DEFSgOblP{}   \chambreBSgOblP{}   \DANSP{}   \INDPlErgP{}   \petitCPlP{}   \chasseurCPlErgP{}   \INDSgAbsP{}   \noirCSgP{}   \oeufCSgAbsP{}  \mangerVtPrsCSgP{} \\
    \DEFSgOblG{}   \chambreBSgOblG{}   \DANSG{}   \INDPlErgG{}   \petitCPlG{}   \chasseurCPlErgG{}   \INDSgAbsG{}   \noirCSgG{}   \oeufCSgAbsG{}  \mangerVtPrsCSgG{} \\
 Des petits chasseurs mangent un oeuf noir dans la chambre
\ex\glll
    \DEMDuObl{}   \maisonDDuObl{}   \DEVANT{}   \INDDuErg{}   \autrucheBDuErg{}   \INDPlAbs{}   \fruitAPlAbs{}  \mangerVtPstAPl{} \\
    \DEMDuOblP{}   \maisonDDuOblP{}   \DEVANTP{}   \INDDuErgP{}   \autrucheBDuErgP{}   \INDPlAbsP{}   \fruitAPlAbsP{}  \mangerVtPstAPlP{} \\
    \DEMDuOblG{}   \maisonDDuOblG{}   \DEVANTG{}   \INDDuErgG{}   \autrucheBDuErgG{}   \INDPlAbsG{}   \fruitAPlAbsG{}  \mangerVtPstAPlG{} \\
 Deux autruches mangeaient des fruits devant ces deux maisons
\ex\glll
   \DEFPlErg{}   \blancBPl{}   \sourisBPlErg{}   \INDSgAbs{}   \infirmiereASgAbs{}  \supporterVtPrsASg{} \\
   \DEFPlErgP{}   \blancBPlP{}   \sourisBPlErgP{}   \INDSgAbsP{}   \infirmiereASgAbsP{}  \supporterVtPrsASgP{} \\
   \DEFPlErgG{}   \blancBPlG{}   \sourisBPlErgG{}   \INDSgAbsG{}   \infirmiereASgAbsG{}  \supporterVtPrsASgG{} \\
 Les souris blanches supportent une infirmière
\ex\glll
   \DEFDuErg{}   \petitCDu{}   \filleCDuErg{}   \INDPlAbs{}   \grandCPl{}   \cafeCPlAbs{}  \acheterVtPrsCPl{} \\
   \DEFDuErgP{}   \petitCDuP{}   \filleCDuErgP{}   \INDPlAbsP{}   \grandCPlP{}   \cafeCPlAbsP{}  \acheterVtPrsCPlP{} \\
   \DEFDuErgG{}   \petitCDuG{}   \filleCDuErgG{}   \INDPlAbsG{}   \grandCPlG{}   \cafeCPlAbsG{}  \acheterVtPrsCPlG{} \\
 Les deux petites filles achètent des grands cafés
\ex\glll
    \DEFSgObl{}   \tableDSgObl{}   \SUR{}   \DEFPlErg{}   \infirmiereAPlErg{}    \INDSgDat{}   \NabilDSgDat{}   \DEFSgAbs{}   \balaiASgAbs{}  \donnerVdPrsASg{} \\
    \DEFSgOblP{}   \tableDSgOblP{}   \SURP{}   \DEFPlErgP{}   \infirmiereAPlErgP{}    \INDSgDatP{}   \NabilDSgDatP{}   \DEFSgAbsP{}   \balaiASgAbsP{}  \donnerVdPrsASgP{} \\
    \DEFSgOblG{}   \tableDSgOblG{}   \SURG{}   \DEFPlErgG{}   \infirmiereAPlErgG{}    \INDSgDatG{}   \NabilDSgDatG{}   \DEFSgAbsG{}   \balaiASgAbsG{}  \donnerVdPrsASgG{} \\
 Les infirmières donnent le balai à Nabil sur la table
\ex\glll
   \INDDuErg{}    \INDPlObl{}   \coyoteCPlObl{}   \DE{}   \chasseurCDuErg{}    \DEMSgDat{}   \chatDSgDat{}   \DEFSgAbs{}   \viandeASgAbs{}  \montrerVdPrsASg{} \\
   \INDDuErgP{}    \INDPlOblP{}   \coyoteCPlOblP{}   \DEP{}   \chasseurCDuErgP{}    \DEMSgDatP{}   \chatDSgDatP{}   \DEFSgAbsP{}   \viandeASgAbsP{}  \montrerVdPrsASgP{} \\
   \INDDuErgG{}    \INDPlOblG{}   \coyoteCPlOblG{}   \DEG{}   \chasseurCDuErgG{}    \DEMSgDatG{}   \chatDSgDatG{}   \DEFSgAbsG{}   \viandeASgAbsG{}  \montrerVdPrsASgG{} \\
 Deux chasseurs de coyotes montrent la viande à ce chat
\ex\glll
   \INDDuErg{}   \chatDDuErg{}   \INDPlAbs{}   \sourisBPlAbs{}  \mangerVtPrsBPl{} \\
   \INDDuErgP{}   \chatDDuErgP{}   \INDPlAbsP{}   \sourisBPlAbsP{}  \mangerVtPrsBPlP{} \\
   \INDDuErgG{}   \chatDDuErgG{}   \INDPlAbsG{}   \sourisBPlAbsG{}  \mangerVtPrsBPlG{} \\
 Deux chats mangent des souris
\ex\glll
   \INDDuErg{}   \garconDDuErg{}    \INDPlDat{}   \chatDPlDat{}   \INDPlAbs{}   \noirBPl{}   \sourisBPlAbs{}  \donnerVdPstBPl{} \\
   \INDDuErgP{}   \garconDDuErgP{}    \INDPlDatP{}   \chatDPlDatP{}   \INDPlAbsP{}   \noirBPlP{}   \sourisBPlAbsP{}  \donnerVdPstBPlP{} \\
   \INDDuErgG{}   \garconDDuErgG{}    \INDPlDatG{}   \chatDPlDatG{}   \INDPlAbsG{}   \noirBPlG{}   \sourisBPlAbsG{}  \donnerVdPstBPlG{} \\
 Deux garçons donnaient des souris noires à des chats
\ex\glll
   \INDDuErg{}   \grandDDu{}   \garconDDuErg{}   \INDPlAbs{}   \petitCPl{}   \cafeCPlAbs{}  \boireVtPrsCPl{} \\
   \INDDuErgP{}   \grandDDuP{}   \garconDDuErgP{}   \INDPlAbsP{}   \petitCPlP{}   \cafeCPlAbsP{}  \boireVtPrsCPlP{} \\
   \INDDuErgG{}   \grandDDuG{}   \garconDDuErgG{}   \INDPlAbsG{}   \petitCPlG{}   \cafeCPlAbsG{}  \boireVtPrsCPlG{} \\
 Deux grands garçons boivent des petits cafés
\ex\glll
   \DEFSgErg{}   \noirBSg{}   \theBSgErg{}   \DEFPlAbs{}   \coyoteCPlAbs{}  \chasserVtPrsCPl{} \\
   \DEFSgErgP{}   \noirBSgP{}   \theBSgErgP{}   \DEFPlAbsP{}   \coyoteCPlAbsP{}  \chasserVtPrsCPlP{} \\
   \DEFSgErgG{}   \noirBSgG{}   \theBSgErgG{}   \DEFPlAbsG{}   \coyoteCPlAbsG{}  \chasserVtPrsCPlG{} \\
 Le thé noir chasse les coyotes
\ex\glll
   \DEMPlErg{}   \troisCPl{}   \chasseurCPlErg{}    \DEFSgObl{}   \petitDSg{}   \blancDSg{}   \garconDSgObl{}   \A{}   \INDDuAbs{}    \DEFPlObl{}   \plaineAPlObl{}   \DE{}   \autrucheBDuAbs{}  \offrirVdPstBDu{} \\
   \DEMPlErgP{}   \troisCPlP{}   \chasseurCPlErgP{}    \DEFSgOblP{}   \petitDSgP{}   \blancDSgP{}   \garconDSgOblP{}   \AP{}   \INDDuAbsP{}    \DEFPlOblP{}   \plaineAPlOblP{}   \DEP{}   \autrucheBDuAbsP{}  \offrirVdPstBDuP{} \\
   \DEMPlErgG{}   \troisCPlG{}   \chasseurCPlErgG{}    \DEFSgOblG{}   \petitDSgG{}   \blancDSgG{}   \garconDSgOblG{}   \AG{}   \INDDuAbsG{}    \DEFPlOblG{}   \plaineAPlOblG{}   \DEG{}   \autrucheBDuAbsG{}  \offrirVdPstBDuG{} \\
 Ces trois chasseurs offraient deux autruches des plaines au petit garçon blanc
\ex\glll
   \DEFPlErg{}   \quatreAPl{}   \balaiAPlErg{}   \DEFSgAbs{}    \DEFSgObl{}   \cuisineDSgObl{}   \DE{}   \tableDSgAbs{}  \supporterVtPstDSg{} \\
   \DEFPlErgP{}   \quatreAPlP{}   \balaiAPlErgP{}   \DEFSgAbsP{}    \DEFSgOblP{}   \cuisineDSgOblP{}   \DEP{}   \tableDSgAbsP{}  \supporterVtPstDSgP{} \\
   \DEFPlErgG{}   \quatreAPlG{}   \balaiAPlErgG{}   \DEFSgAbsG{}    \DEFSgOblG{}   \cuisineDSgOblG{}   \DEG{}   \tableDSgAbsG{}  \supporterVtPstDSgG{} \\
 Les quatre balais supportaient la table de la cuisine
\ex\glll
   \INDDuErg{}   \filleCDuErg{}    \DEFSgObl{}   \garconDSgObl{}   \A{}   \INDPlAbs{}   \rougeBPl{}   \coussinBPlAbs{}  \offrirVdPstBPl{} \\
   \INDDuErgP{}   \filleCDuErgP{}    \DEFSgOblP{}   \garconDSgOblP{}   \AP{}   \INDPlAbsP{}   \rougeBPlP{}   \coussinBPlAbsP{}  \offrirVdPstBPlP{} \\
   \INDDuErgG{}   \filleCDuErgG{}    \DEFSgOblG{}   \garconDSgOblG{}   \AG{}   \INDPlAbsG{}   \rougeBPlG{}   \coussinBPlAbsG{}  \offrirVdPstBPlG{} \\
 Deux filles offraient des coussins rouges au garçon
\ex\glll
   \DEFDuErg{}   \chasseurCDuErg{}   \DEFPlAbs{}    \INDSgObl{}   \NabilDSgObl{}   \DE{}   \coyoteCPlAbs{}  \supporterVtPstCPl{} \\
   \DEFDuErgP{}   \chasseurCDuErgP{}   \DEFPlAbsP{}    \INDSgOblP{}   \NabilDSgOblP{}   \DEP{}   \coyoteCPlAbsP{}  \supporterVtPstCPlP{} \\
   \DEFDuErgG{}   \chasseurCDuErgG{}   \DEFPlAbsG{}    \INDSgOblG{}   \NabilDSgOblG{}   \DEG{}   \coyoteCPlAbsG{}  \supporterVtPstCPlG{} \\
 Les deux chasseurs supportaient les coyotes de Nabil
\ex\glll
   \DEFSgErg{}   \chasseurCSgErg{}    \DEFSgObl{}    \DEFPlObl{}   \garconDPlObl{}   \DE{}   \chatDSgObl{}   \A{}   \DEMSgAbs{}   \viandeASgAbs{}  \donnerVdPrsASg{} \\
   \DEFSgErgP{}   \chasseurCSgErgP{}    \DEFSgOblP{}    \DEFPlOblP{}   \garconDPlOblP{}   \DEP{}   \chatDSgOblP{}   \AP{}   \DEMSgAbsP{}   \viandeASgAbsP{}  \donnerVdPrsASgP{} \\
   \DEFSgErgG{}   \chasseurCSgErgG{}    \DEFSgOblG{}    \DEFPlOblG{}   \garconDPlOblG{}   \DEG{}   \chatDSgOblG{}   \AG{}   \DEMSgAbsG{}   \viandeASgAbsG{}  \donnerVdPrsASgG{} \\
 Le chasseur donne cette viande au chat des garçons
\ex\glll
   \INDSgAbs{}   \sourisBSgAbs{}    \DEFSgObl{}    \INDSgObl{}   \KatishaASgObl{}   \DE{}   \litDSgObl{}   \DANS{}  \arriverViPrsBSg{} \\
   \INDSgAbsP{}   \sourisBSgAbsP{}    \DEFSgOblP{}    \INDSgOblP{}   \KatishaASgOblP{}   \DEP{}   \litDSgOblP{}   \DANSP{}  \arriverViPrsBSgP{} \\
   \INDSgAbsG{}   \sourisBSgAbsG{}    \DEFSgOblG{}    \INDSgOblG{}   \KatishaASgOblG{}   \DEG{}   \litDSgOblG{}   \DANSG{}  \arriverViPrsBSgG{} \\
 Une souris arrive dans le lit de Katisha
\ex\glll
   \INDSgErg{}   \NicoleBSgErg{}   \DEFDuAbs{}   \grosBDu{}   \autrucheBDuAbs{}  \supporterVtPrsBDu{} \\
   \INDSgErgP{}   \NicoleBSgErgP{}   \DEFDuAbsP{}   \grosBDuP{}   \autrucheBDuAbsP{}  \supporterVtPrsBDuP{} \\
   \INDSgErgG{}   \NicoleBSgErgG{}   \DEFDuAbsG{}   \grosBDuG{}   \autrucheBDuAbsG{}  \supporterVtPrsBDuG{} \\
 Nicole supporte les deux grosses autruches
\ex\glll
    \DEFSgObl{}   \cuisineDSgObl{}   \DANS{}   \INDDuErg{}   \maigreCDu{}   \chasseurCDuErg{}    \INDSgDat{}   \KatishaASgDat{}   \INDPlAbs{}   \jauneCPl{}   \oeufCPlAbs{}  \lancerVdPstCPl{} \\
    \DEFSgOblP{}   \cuisineDSgOblP{}   \DANSP{}   \INDDuErgP{}   \maigreCDuP{}   \chasseurCDuErgP{}    \INDSgDatP{}   \KatishaASgDatP{}   \INDPlAbsP{}   \jauneCPlP{}   \oeufCPlAbsP{}  \lancerVdPstCPlP{} \\
    \DEFSgOblG{}   \cuisineDSgOblG{}   \DANSG{}   \INDDuErgG{}   \maigreCDuG{}   \chasseurCDuErgG{}    \INDSgDatG{}   \KatishaASgDatG{}   \INDPlAbsG{}   \jauneCPlG{}   \oeufCPlAbsG{}  \lancerVdPstCPlG{} \\
 Deux chasseurs maigres lançaient des oeufs jaunes à Katisha dans la cuisine
\ex\glll
   \DEMDuAbs{}   \jauneADu{}   \fruitADuAbs{}    \DEFSgObl{}    \DEFSgObl{}   \chatDSgObl{}   \DE{}   \maisonDSgObl{}   \DANS{}  \entrerViPrsADu{} \\
   \DEMDuAbsP{}   \jauneADuP{}   \fruitADuAbsP{}    \DEFSgOblP{}    \DEFSgOblP{}   \chatDSgOblP{}   \DEP{}   \maisonDSgOblP{}   \DANSP{}  \entrerViPrsADuP{} \\
   \DEMDuAbsG{}   \jauneADuG{}   \fruitADuAbsG{}    \DEFSgOblG{}    \DEFSgOblG{}   \chatDSgOblG{}   \DEG{}   \maisonDSgOblG{}   \DANSG{}  \entrerViPrsADuG{} \\
 Ces deux fruits jaunes entrent dans la maison du chat
\ex\glll
   \DEFDuAbs{}   \infirmiereADuAbs{}    \DEFSgObl{}    \DEFSgObl{}   \chasseurCSgObl{}   \DE{}   \chambreBSgObl{}   \DANS{}  \arriverViPrsADu{} \\
   \DEFDuAbsP{}   \infirmiereADuAbsP{}    \DEFSgOblP{}    \DEFSgOblP{}   \chasseurCSgOblP{}   \DEP{}   \chambreBSgOblP{}   \DANSP{}  \arriverViPrsADuP{} \\
   \DEFDuAbsG{}   \infirmiereADuAbsG{}    \DEFSgOblG{}    \DEFSgOblG{}   \chasseurCSgOblG{}   \DEG{}   \chambreBSgOblG{}   \DANSG{}  \arriverViPrsADuG{} \\
 Les deux infirmières arrivent dans la chambre du chasseur
\ex\glll
   \DEMPlErg{}   \maigreDPl{}   \garconDPlErg{}   \DEFDuAbs{}    \DEFPlObl{}   \chasseurCPlObl{}   \DE{}   \blancBDu{}   \sourisBDuAbs{}  \supporterVtPstBDu{} \\
   \DEMPlErgP{}   \maigreDPlP{}   \garconDPlErgP{}   \DEFDuAbsP{}    \DEFPlOblP{}   \chasseurCPlOblP{}   \DEP{}   \blancBDuP{}   \sourisBDuAbsP{}  \supporterVtPstBDuP{} \\
   \DEMPlErgG{}   \maigreDPlG{}   \garconDPlErgG{}   \DEFDuAbsG{}    \DEFPlOblG{}   \chasseurCPlOblG{}   \DEG{}   \blancBDuG{}   \sourisBDuAbsG{}  \supporterVtPstBDuG{} \\
 Ces garçons maigres supportaient les deux souris blanches des chasseurs
\ex\glll
   \INDDuErg{}   \infirmiereADuErg{}    \INDDuDat{}   \chatDDuDat{}   \INDSgAbs{}   \litDSgAbs{}  \montrerVdPstDSg{} \\
   \INDDuErgP{}   \infirmiereADuErgP{}    \INDDuDatP{}   \chatDDuDatP{}   \INDSgAbsP{}   \litDSgAbsP{}  \montrerVdPstDSgP{} \\
   \INDDuErgG{}   \infirmiereADuErgG{}    \INDDuDatG{}   \chatDDuDatG{}   \INDSgAbsG{}   \litDSgAbsG{}  \montrerVdPstDSgG{} \\
 Deux infirmières montraient un lit à deux chattes
\ex\glll
    \DEFSgObl{}   \jauneDSg{}   \maisonDSgObl{}   \DEVANT{}   \DEMPlErg{}   \filleCPlErg{}   \INDPlAbs{}   \balaiAPlAbs{}  \acheterVtPstAPl{} \\
    \DEFSgOblP{}   \jauneDSgP{}   \maisonDSgOblP{}   \DEVANTP{}   \DEMPlErgP{}   \filleCPlErgP{}   \INDPlAbsP{}   \balaiAPlAbsP{}  \acheterVtPstAPlP{} \\
    \DEFSgOblG{}   \jauneDSgG{}   \maisonDSgOblG{}   \DEVANTG{}   \DEMPlErgG{}   \filleCPlErgG{}   \INDPlAbsG{}   \balaiAPlAbsG{}  \acheterVtPstAPlG{} \\
 Ces filles achetaient des balais devant la maison jaune
\ex\glll
   \INDSgErg{}   \NicoleBSgErg{}    \DEFPlObl{}    \DEFPlObl{}   \infirmiereAPlObl{}   \DE{}   \filleCPlObl{}   \A{}   \DEFSgAbs{}    \DEFSgObl{}   \maisonDSgObl{}   \DE{}   \cuisineDSgAbs{}  \montrerVdPrsDSg{} \\
   \INDSgErgP{}   \NicoleBSgErgP{}    \DEFPlOblP{}    \DEFPlOblP{}   \infirmiereAPlOblP{}   \DEP{}   \filleCPlOblP{}   \AP{}   \DEFSgAbsP{}    \DEFSgOblP{}   \maisonDSgOblP{}   \DEP{}   \cuisineDSgAbsP{}  \montrerVdPrsDSgP{} \\
   \INDSgErgG{}   \NicoleBSgErgG{}    \DEFPlOblG{}    \DEFPlOblG{}   \infirmiereAPlOblG{}   \DEG{}   \filleCPlOblG{}   \AG{}   \DEFSgAbsG{}    \DEFSgOblG{}   \maisonDSgOblG{}   \DEG{}   \cuisineDSgAbsG{}  \montrerVdPrsDSgG{} \\
 Nicole montre la cuisine de la maison aux filles des infirmières
\ex\glll
   \DEFSgAbs{}    \DEFPlObl{}   \balaiAPlObl{}   \AVEC{}   \tableDSgAbs{}    \DEFPlObl{}    \DEFSgObl{}   \cuisineDSgObl{}   \DANS{}   \chasseurCPlObl{}   \DEVANT{}  \tomberViPstDSg{} \\
   \DEFSgAbsP{}    \DEFPlOblP{}   \balaiAPlOblP{}   \AVECP{}   \tableDSgAbsP{}    \DEFPlOblP{}    \DEFSgOblP{}   \cuisineDSgOblP{}   \DANSP{}   \chasseurCPlOblP{}   \DEVANTP{}  \tomberViPstDSgP{} \\
   \DEFSgAbsG{}    \DEFPlOblG{}   \balaiAPlOblG{}   \AVECG{}   \tableDSgAbsG{}    \DEFPlOblG{}    \DEFSgOblG{}   \cuisineDSgOblG{}   \DANSG{}   \chasseurCPlOblG{}   \DEVANTG{}  \tomberViPstDSgG{} \\
 La table avec les balais tombait devant les chasseurs dans la cuisine
\ex\glll
    \DEFSgObl{}   \noirBSg{}   \chambreBSgObl{}   \DANS{}   \INDSgAbs{}   \tableDSgAbs{}  \tomberViPrsDSg{} \\
    \DEFSgOblP{}   \noirBSgP{}   \chambreBSgOblP{}   \DANSP{}   \INDSgAbsP{}   \tableDSgAbsP{}  \tomberViPrsDSgP{} \\
    \DEFSgOblG{}   \noirBSgG{}   \chambreBSgOblG{}   \DANSG{}   \INDSgAbsG{}   \tableDSgAbsG{}  \tomberViPrsDSgG{} \\
 Une table tombe dans la chambre noire
\ex\glll
   \INDSgErg{}   \petitDSg{}   \chatDSgErg{}   \INDPlAbs{}   \troisBPl{}   \blancBPl{}   \sourisBPlAbs{}  \chasserVtPstBPl{} \\
   \INDSgErgP{}   \petitDSgP{}   \chatDSgErgP{}   \INDPlAbsP{}   \troisBPlP{}   \blancBPlP{}   \sourisBPlAbsP{}  \chasserVtPstBPlP{} \\
   \INDSgErgG{}   \petitDSgG{}   \chatDSgErgG{}   \INDPlAbsG{}   \troisBPlG{}   \blancBPlG{}   \sourisBPlAbsG{}  \chasserVtPstBPlG{} \\
 Un petit chat chassait trois souris blanches
\ex\glll
   \DEFPlAbs{}   \rougeAPl{}   \fruitAPlAbs{}    \DEFSgObl{}    \DEFSgObl{}    \INDSgObl{}   \NabilDSgObl{}   \DE{}   \filleCSgObl{}   \DE{}   \villageCSgObl{}   \DANS{}  \arriverViPrsAPl{} \\
   \DEFPlAbsP{}   \rougeAPlP{}   \fruitAPlAbsP{}    \DEFSgOblP{}    \DEFSgOblP{}    \INDSgOblP{}   \NabilDSgOblP{}   \DEP{}   \filleCSgOblP{}   \DEP{}   \villageCSgOblP{}   \DANSP{}  \arriverViPrsAPlP{} \\
   \DEFPlAbsG{}   \rougeAPlG{}   \fruitAPlAbsG{}    \DEFSgOblG{}    \DEFSgOblG{}    \INDSgOblG{}   \NabilDSgOblG{}   \DEG{}   \filleCSgOblG{}   \DEG{}   \villageCSgOblG{}   \DANSG{}  \arriverViPrsAPlG{} \\
 Les fruits rouges arrivent dans le village de la fille de Nabil
\ex\glll
    \DEFDuObl{}   \petitCDu{}   \chasseurCDuObl{}   \AVEC{}   \DEMSgErg{}   \filleCSgErg{}   \INDSgAbs{}   \noirBSg{}   \theBSgAbs{}  \boireVtPrsBSg{} \\
    \DEFDuOblP{}   \petitCDuP{}   \chasseurCDuOblP{}   \AVECP{}   \DEMSgErgP{}   \filleCSgErgP{}   \INDSgAbsP{}   \noirBSgP{}   \theBSgAbsP{}  \boireVtPrsBSgP{} \\
    \DEFDuOblG{}   \petitCDuG{}   \chasseurCDuOblG{}   \AVECG{}   \DEMSgErgG{}   \filleCSgErgG{}   \INDSgAbsG{}   \noirBSgG{}   \theBSgAbsG{}  \boireVtPrsBSgG{} \\
 Cette fille boit un thé noir avec les deux petits chasseurs
\ex\glll
   \DEFDuAbs{}   \blancDDu{}   \chatDDuAbs{}    \DEFSgObl{}   \basDSg{}   \tableDSgObl{}   \SUR{}  \dormirViPrsDDu{} \\
   \DEFDuAbsP{}   \blancDDuP{}   \chatDDuAbsP{}    \DEFSgOblP{}   \basDSgP{}   \tableDSgOblP{}   \SURP{}  \dormirViPrsDDuP{} \\
   \DEFDuAbsG{}   \blancDDuG{}   \chatDDuAbsG{}    \DEFSgOblG{}   \basDSgG{}   \tableDSgOblG{}   \SURG{}  \dormirViPrsDDuG{} \\
 Les deux chats blancs dorment sur la table basse
\ex\glll
   \DEFSgAbs{}    \INDPlObl{}   \blancBPl{}   \sourisBPlObl{}   \DE{}   \chasseurCSgAbs{}    \DEFSgObl{}    \INDSgObl{}   \KatishaASgObl{}   \DE{}   \grandDSg{}   \cuisineDSgObl{}   \DANS{}  \entrerViPrsCSg{} \\
   \DEFSgAbsP{}    \INDPlOblP{}   \blancBPlP{}   \sourisBPlOblP{}   \DEP{}   \chasseurCSgAbsP{}    \DEFSgOblP{}    \INDSgOblP{}   \KatishaASgOblP{}   \DEP{}   \grandDSgP{}   \cuisineDSgOblP{}   \DANSP{}  \entrerViPrsCSgP{} \\
   \DEFSgAbsG{}    \INDPlOblG{}   \blancBPlG{}   \sourisBPlOblG{}   \DEG{}   \chasseurCSgAbsG{}    \DEFSgOblG{}    \INDSgOblG{}   \KatishaASgOblG{}   \DEG{}   \grandDSgG{}   \cuisineDSgOblG{}   \DANSG{}  \entrerViPrsCSgG{} \\
 Le chasseur de souris blanches entre dans la grande cuisine de Katisha
\ex\glll
   \DEFSgErg{}   \tableDSgErg{}   \INDPlAbs{}   \rougeAPl{}   \fruitAPlAbs{}  \supporterVtPrsAPl{} \\
   \DEFSgErgP{}   \tableDSgErgP{}   \INDPlAbsP{}   \rougeAPlP{}   \fruitAPlAbsP{}  \supporterVtPrsAPlP{} \\
   \DEFSgErgG{}   \tableDSgErgG{}   \INDPlAbsG{}   \rougeAPlG{}   \fruitAPlAbsG{}  \supporterVtPrsAPlG{} \\
 La table supporte des fruits rouges
\ex\glll
    \DEFSgObl{}   \cuisineDSgObl{}   \POUR{}   \INDSgErg{}   \NabilDSgErg{}    \INDSgDat{}   \NicoleBSgDat{}   \DEMPlAbs{}   \viandeAPlAbs{}  \offrirVdPrsAPl{} \\
    \DEFSgOblP{}   \cuisineDSgOblP{}   \POURP{}   \INDSgErgP{}   \NabilDSgErgP{}    \INDSgDatP{}   \NicoleBSgDatP{}   \DEMPlAbsP{}   \viandeAPlAbsP{}  \offrirVdPrsAPlP{} \\
    \DEFSgOblG{}   \cuisineDSgOblG{}   \POURG{}   \INDSgErgG{}   \NabilDSgErgG{}    \INDSgDatG{}   \NicoleBSgDatG{}   \DEMPlAbsG{}   \viandeAPlAbsG{}  \offrirVdPrsAPlG{} \\
 Nabil offre ces viandes à Nicole pour la cuisine
\ex\glll
   \DEFPlErg{}   \petitDPl{}   \garconDPlErg{}   \INDDuAbs{}   \blancBDu{}   \coussinBDuAbs{}  \acheterVtPrsBDu{} \\
   \DEFPlErgP{}   \petitDPlP{}   \garconDPlErgP{}   \INDDuAbsP{}   \blancBDuP{}   \coussinBDuAbsP{}  \acheterVtPrsBDuP{} \\
   \DEFPlErgG{}   \petitDPlG{}   \garconDPlErgG{}   \INDDuAbsG{}   \blancBDuG{}   \coussinBDuAbsG{}  \acheterVtPrsBDuG{} \\
 Les petits garçons achètent deux coussins blancs
\ex\glll
   \DEFPlErg{}   \infirmiereAPlErg{}    \INDPlObl{}   \balaiAPlObl{}   \AVEC{}   \INDSgAbs{}   \sourisBSgAbs{}  \chasserVtPstBSg{} \\
   \DEFPlErgP{}   \infirmiereAPlErgP{}    \INDPlOblP{}   \balaiAPlOblP{}   \AVECP{}   \INDSgAbsP{}   \sourisBSgAbsP{}  \chasserVtPstBSgP{} \\
   \DEFPlErgG{}   \infirmiereAPlErgG{}    \INDPlOblG{}   \balaiAPlOblG{}   \AVECG{}   \INDSgAbsG{}   \sourisBSgAbsG{}  \chasserVtPstBSgG{} \\
 Les infirmières chassaient une souris avec des balais
\ex\glll
   \DEMPlAbs{}   \autrucheBPlAbs{}    \DEFSgObl{}    \INDSgObl{}   \NabilDSgObl{}   \DE{}   \villageCSgObl{}   \DANS{}  \arriverViPrsBPl{} \\
   \DEMPlAbsP{}   \autrucheBPlAbsP{}    \DEFSgOblP{}    \INDSgOblP{}   \NabilDSgOblP{}   \DEP{}   \villageCSgOblP{}   \DANSP{}  \arriverViPrsBPlP{} \\
   \DEMPlAbsG{}   \autrucheBPlAbsG{}    \DEFSgOblG{}    \INDSgOblG{}   \NabilDSgOblG{}   \DEG{}   \villageCSgOblG{}   \DANSG{}  \arriverViPrsBPlG{} \\
 Ces autruches arrivent dans le village de Nabil
\ex\glll
    \DEFSgObl{}   \maisonDSgObl{}   \DEVANT{}   \DEFPlErg{}   \chasseurCPlErg{}    \DEFDuObl{}   \petitCDu{}   \filleCDuObl{}   \A{}   \INDSgAbs{}   \autrucheBSgAbs{}  \montrerVdPstBSg{} \\
    \DEFSgOblP{}   \maisonDSgOblP{}   \DEVANTP{}   \DEFPlErgP{}   \chasseurCPlErgP{}    \DEFDuOblP{}   \petitCDuP{}   \filleCDuOblP{}   \AP{}   \INDSgAbsP{}   \autrucheBSgAbsP{}  \montrerVdPstBSgP{} \\
    \DEFSgOblG{}   \maisonDSgOblG{}   \DEVANTG{}   \DEFPlErgG{}   \chasseurCPlErgG{}    \DEFDuOblG{}   \petitCDuG{}   \filleCDuOblG{}   \AG{}   \INDSgAbsG{}   \autrucheBSgAbsG{}  \montrerVdPstBSgG{} \\
 Les chasseurs montraient une autruche aux deux petites filles devant la maison
\ex\glll
   \INDPlAbs{}   \garconDPlAbs{}    \DEFDuObl{}   \petitBDu{}   \blancBDu{}   \chambreBDuObl{}   \DANS{}  \arriverViPrsDPl{} \\
   \INDPlAbsP{}   \garconDPlAbsP{}    \DEFDuOblP{}   \petitBDuP{}   \blancBDuP{}   \chambreBDuOblP{}   \DANSP{}  \arriverViPrsDPlP{} \\
   \INDPlAbsG{}   \garconDPlAbsG{}    \DEFDuOblG{}   \petitBDuG{}   \blancBDuG{}   \chambreBDuOblG{}   \DANSG{}  \arriverViPrsDPlG{} \\
 Des garçons arrivent dans les deux petites chambres blanches
\ex\glll
    \DEFSgObl{}    \DEFSgObl{}    \DEFSgObl{}   \petitDSg{}   \blancDSg{}   \maisonDSgObl{}   \DE{}   \cuisineDSgObl{}   \DE{}   \tableDSgObl{}   \SUR{}   \DEFSgErg{}   \grosCSg{}   \chasseurCSgErg{}   \INDSgAbs{}   \cafeCSgAbs{}  \boireVtPstCSg{} \\
    \DEFSgOblP{}    \DEFSgOblP{}    \DEFSgOblP{}   \petitDSgP{}   \blancDSgP{}   \maisonDSgOblP{}   \DEP{}   \cuisineDSgOblP{}   \DEP{}   \tableDSgOblP{}   \SURP{}   \DEFSgErgP{}   \grosCSgP{}   \chasseurCSgErgP{}   \INDSgAbsP{}   \cafeCSgAbsP{}  \boireVtPstCSgP{} \\
    \DEFSgOblG{}    \DEFSgOblG{}    \DEFSgOblG{}   \petitDSgG{}   \blancDSgG{}   \maisonDSgOblG{}   \DEG{}   \cuisineDSgOblG{}   \DEG{}   \tableDSgOblG{}   \SURG{}   \DEFSgErgG{}   \grosCSgG{}   \chasseurCSgErgG{}   \INDSgAbsG{}   \cafeCSgAbsG{}  \boireVtPstCSgG{} \\
 Le gros chasseur buvait un café sur la table de la cuisine de la petite maison blanche
\ex\glll
   \INDSgAbs{}   \petitBSg{}   \sourisBSgAbs{}    \DEFSgObl{}    \DEFSgObl{}   \grosDSg{}   \chatDSgObl{}   \DE{}   \maisonDSgObl{}   \DANS{}  \entrerViPrsBSg{} \\
   \INDSgAbsP{}   \petitBSgP{}   \sourisBSgAbsP{}    \DEFSgOblP{}    \DEFSgOblP{}   \grosDSgP{}   \chatDSgOblP{}   \DEP{}   \maisonDSgOblP{}   \DANSP{}  \entrerViPrsBSgP{} \\
   \INDSgAbsG{}   \petitBSgG{}   \sourisBSgAbsG{}    \DEFSgOblG{}    \DEFSgOblG{}   \grosDSgG{}   \chatDSgOblG{}   \DEG{}   \maisonDSgOblG{}   \DANSG{}  \entrerViPrsBSgG{} \\
 Une petite souris entre dans la maison du gros chat
\ex\glll
    \DEMSgObl{}   \grandDSg{}   \litDSgObl{}   \SOUS{}   \DEFSgErg{}   \petitBSg{}   \sourisBSgErg{}   \INDSgAbs{}   \petitBSg{}   \theBSgAbs{}  \boireVtPrsBSg{} \\
    \DEMSgOblP{}   \grandDSgP{}   \litDSgOblP{}   \SOUSP{}   \DEFSgErgP{}   \petitBSgP{}   \sourisBSgErgP{}   \INDSgAbsP{}   \petitBSgP{}   \theBSgAbsP{}  \boireVtPrsBSgP{} \\
    \DEMSgOblG{}   \grandDSgG{}   \litDSgOblG{}   \SOUSG{}   \DEFSgErgG{}   \petitBSgG{}   \sourisBSgErgG{}   \INDSgAbsG{}   \petitBSgG{}   \theBSgAbsG{}  \boireVtPrsBSgG{} \\
 La petite souris boit un petit thé sous ce grand lit
\ex\glll
    \INDPlObl{}   \coussinBPlObl{}   \SUR{}   \DEFPlErg{}   \grosCPl{}   \chasseurCPlErg{}   \INDSgAbs{}   \petitBSg{}   \noirBSg{}   \autrucheBSgAbs{}  \mangerVtPstBSg{} \\
    \INDPlOblP{}   \coussinBPlOblP{}   \SURP{}   \DEFPlErgP{}   \grosCPlP{}   \chasseurCPlErgP{}   \INDSgAbsP{}   \petitBSgP{}   \noirBSgP{}   \autrucheBSgAbsP{}  \mangerVtPstBSgP{} \\
    \INDPlOblG{}   \coussinBPlOblG{}   \SURG{}   \DEFPlErgG{}   \grosCPlG{}   \chasseurCPlErgG{}   \INDSgAbsG{}   \petitBSgG{}   \noirBSgG{}   \autrucheBSgAbsG{}  \mangerVtPstBSgG{} \\
 Les gros chasseurs mangeaient une petite autruche noire sur des coussins
\ex\glll
    \DEFPlObl{}   \infirmiereAPlObl{}   \AVEC{}   \DEFPlAbs{}   \petitBPl{}   \coussinBPlAbs{}    \DEFSgObl{}   \grandDSg{}   \maisonDSgObl{}   \DANS{}  \arriverViPstBPl{} \\
    \DEFPlOblP{}   \infirmiereAPlOblP{}   \AVECP{}   \DEFPlAbsP{}   \petitBPlP{}   \coussinBPlAbsP{}    \DEFSgOblP{}   \grandDSgP{}   \maisonDSgOblP{}   \DANSP{}  \arriverViPstBPlP{} \\
    \DEFPlOblG{}   \infirmiereAPlOblG{}   \AVECG{}   \DEFPlAbsG{}   \petitBPlG{}   \coussinBPlAbsG{}    \DEFSgOblG{}   \grandDSgG{}   \maisonDSgOblG{}   \DANSG{}  \arriverViPstBPlG{} \\
 Les petits coussins arrivaient dans la grande maison avec les infirmières
\ex\glll
   \DEFPlAbs{}   \grosCPl{}   \chasseurCPlAbs{}    \DEFPlObl{}   \grandDPl{}   \noirDPl{}   \tableDPlObl{}   \SOUS{}  \dormirViPrsCPl{} \\
   \DEFPlAbsP{}   \grosCPlP{}   \chasseurCPlAbsP{}    \DEFPlOblP{}   \grandDPlP{}   \noirDPlP{}   \tableDPlOblP{}   \SOUSP{}  \dormirViPrsCPlP{} \\
   \DEFPlAbsG{}   \grosCPlG{}   \chasseurCPlAbsG{}    \DEFPlOblG{}   \grandDPlG{}   \noirDPlG{}   \tableDPlOblG{}   \SOUSG{}  \dormirViPrsCPlG{} \\
 Les gros chasseurs dorment sous les grandes tables noires
\ex\glll
    \DEFSgObl{}   \cuisineDSgObl{}   \DEVANT{}   \INDDuAbs{}   \chasseurCDuAbs{}    \INDPlObl{}   \coussinBPlObl{}   \SUR{}  \dormirViPstCDu{} \\
    \DEFSgOblP{}   \cuisineDSgOblP{}   \DEVANTP{}   \INDDuAbsP{}   \chasseurCDuAbsP{}    \INDPlOblP{}   \coussinBPlOblP{}   \SURP{}  \dormirViPstCDuP{} \\
    \DEFSgOblG{}   \cuisineDSgOblG{}   \DEVANTG{}   \INDDuAbsG{}   \chasseurCDuAbsG{}    \INDPlOblG{}   \coussinBPlOblG{}   \SURG{}  \dormirViPstCDuG{} \\
 Deux chasseurs dormaient sur des coussins devant la cuisine
\ex\glll
   \DEFSgAbs{}   \coyoteCSgAbs{}    \DEFSgObl{}    \DEFSgObl{}   \villageCSgObl{}   \DE{}   \jauneDSg{}   \maisonDSgObl{}   \DANS{}  \entrerViPrsCSg{} \\
   \DEFSgAbsP{}   \coyoteCSgAbsP{}    \DEFSgOblP{}    \DEFSgOblP{}   \villageCSgOblP{}   \DEP{}   \jauneDSgP{}   \maisonDSgOblP{}   \DANSP{}  \entrerViPrsCSgP{} \\
   \DEFSgAbsG{}   \coyoteCSgAbsG{}    \DEFSgOblG{}    \DEFSgOblG{}   \villageCSgOblG{}   \DEG{}   \jauneDSgG{}   \maisonDSgOblG{}   \DANSG{}  \entrerViPrsCSgG{} \\
 Le coyote entre dans la maison jaune du village
\ex\glll
   \DEFPlErg{}   \chasseurCPlErg{}    \DEFPlObl{}   \grandBPl{}   \autrucheBPlObl{}   \A{}   \INDPlAbs{}   \jauneAPl{}   \fruitAPlAbs{}  \donnerVdPstAPl{} \\
   \DEFPlErgP{}   \chasseurCPlErgP{}    \DEFPlOblP{}   \grandBPlP{}   \autrucheBPlOblP{}   \AP{}   \INDPlAbsP{}   \jauneAPlP{}   \fruitAPlAbsP{}  \donnerVdPstAPlP{} \\
   \DEFPlErgG{}   \chasseurCPlErgG{}    \DEFPlOblG{}   \grandBPlG{}   \autrucheBPlOblG{}   \AG{}   \INDPlAbsG{}   \jauneAPlG{}   \fruitAPlAbsG{}  \donnerVdPstAPlG{} \\
 Les chasseurs donnaient des fruits jaunes aux grandes autruches
\ex\glll
   \DEFDuErg{}   \autrucheBDuErg{}   \DEMDuAbs{}   \grandCDu{}   \cafeCDuAbs{}  \boireVtPrsCDu{} \\
   \DEFDuErgP{}   \autrucheBDuErgP{}   \DEMDuAbsP{}   \grandCDuP{}   \cafeCDuAbsP{}  \boireVtPrsCDuP{} \\
   \DEFDuErgG{}   \autrucheBDuErgG{}   \DEMDuAbsG{}   \grandCDuG{}   \cafeCDuAbsG{}  \boireVtPrsCDuG{} \\
 Les deux autruches boivent ces deux grands cafés
\ex\glll
   \INDDuErg{}   \autrucheBDuErg{}    \DEMPlDat{}   \maigreCPl{}   \coyoteCPlDat{}   \INDSgAbs{}   \oeufCSgAbs{}  \donnerVdPrsCSg{} \\
   \INDDuErgP{}   \autrucheBDuErgP{}    \DEMPlDatP{}   \maigreCPlP{}   \coyoteCPlDatP{}   \INDSgAbsP{}   \oeufCSgAbsP{}  \donnerVdPrsCSgP{} \\
   \INDDuErgG{}   \autrucheBDuErgG{}    \DEMPlDatG{}   \maigreCPlG{}   \coyoteCPlDatG{}   \INDSgAbsG{}   \oeufCSgAbsG{}  \donnerVdPrsCSgG{} \\
 Deux autruches donnent un oeuf à ces coyotes maigres
\ex\glll
   \INDSgAbs{}   \oeufCSgAbs{}    \DEFSgObl{}   \litDSgObl{}   \DANS{}  \tomberViPrsCSg{} \\
   \INDSgAbsP{}   \oeufCSgAbsP{}    \DEFSgOblP{}   \litDSgOblP{}   \DANSP{}  \tomberViPrsCSgP{} \\
   \INDSgAbsG{}   \oeufCSgAbsG{}    \DEFSgOblG{}   \litDSgOblG{}   \DANSG{}  \tomberViPrsCSgG{} \\
 Un oeuf tombe dans le lit
\ex\glll
   \DEMDuErg{}   \grosBDu{}   \sourisBDuErg{}   \DEFPlAbs{}    \DEFPlObl{}    \INDSgObl{}   \NicoleBSgObl{}   \DE{}   \filleCPlObl{}   \DE{}   \petitDPl{}   \chatDPlAbs{}  \chasserVtPrsDPl{} \\
   \DEMDuErgP{}   \grosBDuP{}   \sourisBDuErgP{}   \DEFPlAbsP{}    \DEFPlOblP{}    \INDSgOblP{}   \NicoleBSgOblP{}   \DEP{}   \filleCPlOblP{}   \DEP{}   \petitDPlP{}   \chatDPlAbsP{}  \chasserVtPrsDPlP{} \\
   \DEMDuErgG{}   \grosBDuG{}   \sourisBDuErgG{}   \DEFPlAbsG{}    \DEFPlOblG{}    \INDSgOblG{}   \NicoleBSgOblG{}   \DEG{}   \filleCPlOblG{}   \DEG{}   \petitDPlG{}   \chatDPlAbsG{}  \chasserVtPrsDPlG{} \\
 Ces deux grosses souris chassent les petits chats des filles de Nicole
\ex\glll
    \DEFSgObl{}   \plaineASgObl{}   \DANS{}   \DEFDuErg{}   \coyoteCDuErg{}   \INDPlAbs{}   \autrucheBPlAbs{}  \chasserVtPrsBPl{} \\
    \DEFSgOblP{}   \plaineASgOblP{}   \DANSP{}   \DEFDuErgP{}   \coyoteCDuErgP{}   \INDPlAbsP{}   \autrucheBPlAbsP{}  \chasserVtPrsBPlP{} \\
    \DEFSgOblG{}   \plaineASgOblG{}   \DANSG{}   \DEFDuErgG{}   \coyoteCDuErgG{}   \INDPlAbsG{}   \autrucheBPlAbsG{}  \chasserVtPrsBPlG{} \\
 Les deux coyotes chassent des autruches dans la plaine
\ex\glll
   \DEFPlAbs{}   \noirBPl{}   \theBPlAbs{}    \DEFPlObl{}   \coussinBPlObl{}   \SOUS{}  \tomberViPrsBPl{} \\
   \DEFPlAbsP{}   \noirBPlP{}   \theBPlAbsP{}    \DEFPlOblP{}   \coussinBPlOblP{}   \SOUSP{}  \tomberViPrsBPlP{} \\
   \DEFPlAbsG{}   \noirBPlG{}   \theBPlAbsG{}    \DEFPlOblG{}   \coussinBPlOblG{}   \SOUSG{}  \tomberViPrsBPlG{} \\
 Les thés noirs tombent sous les coussins
\ex\glll
   \DEFDuErg{}   \jauneDDu{}   \chatDDuErg{}    \DEFSgObl{}   \petitDSg{}   \cuisineDSgObl{}   \DANS{}   \DEFPlAbs{}    \DEFSgObl{}   \maisonDSgObl{}   \DE{}   \sourisBPlAbs{}  \chasserVtPrsBPl{} \\
   \DEFDuErgP{}   \jauneDDuP{}   \chatDDuErgP{}    \DEFSgOblP{}   \petitDSgP{}   \cuisineDSgOblP{}   \DANSP{}   \DEFPlAbsP{}    \DEFSgOblP{}   \maisonDSgOblP{}   \DEP{}   \sourisBPlAbsP{}  \chasserVtPrsBPlP{} \\
   \DEFDuErgG{}   \jauneDDuG{}   \chatDDuErgG{}    \DEFSgOblG{}   \petitDSgG{}   \cuisineDSgOblG{}   \DANSG{}   \DEFPlAbsG{}    \DEFSgOblG{}   \maisonDSgOblG{}   \DEG{}   \sourisBPlAbsG{}  \chasserVtPrsBPlG{} \\
 Les deux chats jaunes chassent les souris de la maison dans la petite cuisine
\ex\glll
   \DEFPlErg{}   \infirmiereAPlErg{}    \INDSgDat{}   \KatishaASgDat{}   \INDSgAbs{}   \litDSgAbs{}  \offrirVdPrsDSg{} \\
   \DEFPlErgP{}   \infirmiereAPlErgP{}    \INDSgDatP{}   \KatishaASgDatP{}   \INDSgAbsP{}   \litDSgAbsP{}  \offrirVdPrsDSgP{} \\
   \DEFPlErgG{}   \infirmiereAPlErgG{}    \INDSgDatG{}   \KatishaASgDatG{}   \INDSgAbsG{}   \litDSgAbsG{}  \offrirVdPrsDSgG{} \\
 Les infirmières offrent un lit à Katisha
\ex\glll
   \DEMPlAbs{}   \infirmiereAPlAbs{}    \DEFPlObl{}   \grandBPl{}   \chambreBPlObl{}   \DANS{}  \dormirViPrsAPl{} \\
   \DEMPlAbsP{}   \infirmiereAPlAbsP{}    \DEFPlOblP{}   \grandBPlP{}   \chambreBPlOblP{}   \DANSP{}  \dormirViPrsAPlP{} \\
   \DEMPlAbsG{}   \infirmiereAPlAbsG{}    \DEFPlOblG{}   \grandBPlG{}   \chambreBPlOblG{}   \DANSG{}  \dormirViPrsAPlG{} \\
 Ces infirmières dorment dans les grandes chambres
\ex\glll
    \DEFSgObl{}    \DEFDuObl{}   \infirmiereADuObl{}   \DE{}   \chambreBSgObl{}   \SOUS{}   \INDPlAbs{}   \coyoteCPlAbs{}  \dormirViPstCPl{} \\
    \DEFSgOblP{}    \DEFDuOblP{}   \infirmiereADuOblP{}   \DEP{}   \chambreBSgOblP{}   \SOUSP{}   \INDPlAbsP{}   \coyoteCPlAbsP{}  \dormirViPstCPlP{} \\
    \DEFSgOblG{}    \DEFDuOblG{}   \infirmiereADuOblG{}   \DEG{}   \chambreBSgOblG{}   \SOUSG{}   \INDPlAbsG{}   \coyoteCPlAbsG{}  \dormirViPstCPlG{} \\
 Des coyotes dormaient sous la chambre des deux infirmières
\ex\glll
   \INDSgErg{}   \KatishaASgErg{}    \DEMDuDat{}   \filleCDuDat{}   \DEMPlAbs{}   \troisDPl{}   \chatDPlAbs{}  \donnerVdPrsDPl{} \\
   \INDSgErgP{}   \KatishaASgErgP{}    \DEMDuDatP{}   \filleCDuDatP{}   \DEMPlAbsP{}   \troisDPlP{}   \chatDPlAbsP{}  \donnerVdPrsDPlP{} \\
   \INDSgErgG{}   \KatishaASgErgG{}    \DEMDuDatG{}   \filleCDuDatG{}   \DEMPlAbsG{}   \troisDPlG{}   \chatDPlAbsG{}  \donnerVdPrsDPlG{} \\
 Katisha donne ces trois chats à ces deux filles
\ex\glll
   \DEMDuErg{}   \tableDDuErg{}   \DEFPlAbs{}   \filleCPlAbs{}  \supporterVtPrsCPl{} \\
   \DEMDuErgP{}   \tableDDuErgP{}   \DEFPlAbsP{}   \filleCPlAbsP{}  \supporterVtPrsCPlP{} \\
   \DEMDuErgG{}   \tableDDuErgG{}   \DEFPlAbsG{}   \filleCPlAbsG{}  \supporterVtPrsCPlG{} \\
 Ces deux tables supportent les filles
\ex\glll
    \INDDuObl{}   \grosDDu{}   \chatDDuObl{}   \AVEC{}   \INDSgAbs{}   \NicoleBSgAbs{}    \DEFSgObl{}   \chambreBSgObl{}   \DANS{}  \arriverViPstBSg{} \\
    \INDDuOblP{}   \grosDDuP{}   \chatDDuOblP{}   \AVECP{}   \INDSgAbsP{}   \NicoleBSgAbsP{}    \DEFSgOblP{}   \chambreBSgOblP{}   \DANSP{}  \arriverViPstBSgP{} \\
    \INDDuOblG{}   \grosDDuG{}   \chatDDuOblG{}   \AVECG{}   \INDSgAbsG{}   \NicoleBSgAbsG{}    \DEFSgOblG{}   \chambreBSgOblG{}   \DANSG{}  \arriverViPstBSgG{} \\
 Nicole arrivait dans la chambre avec deux gros chats
\ex\glll
   \INDPlErg{}   \filleCPlErg{}   \INDSgAbs{}   \maigreDSg{}   \noirDSg{}   \chatDSgAbs{}  \acheterVtPrsDSg{} \\
   \INDPlErgP{}   \filleCPlErgP{}   \INDSgAbsP{}   \maigreDSgP{}   \noirDSgP{}   \chatDSgAbsP{}  \acheterVtPrsDSgP{} \\
   \INDPlErgG{}   \filleCPlErgG{}   \INDSgAbsG{}   \maigreDSgG{}   \noirDSgG{}   \chatDSgAbsG{}  \acheterVtPrsDSgG{} \\
 Des filles achètent un maigre chat noir
\ex\glll
    \DEFSgObl{}    \DEFSgObl{}   \cuisineDSgObl{}   \DEVANT{}   \plaineASgObl{}   \DANS{}   \INDSgErg{}   \grandCSg{}   \chasseurCSgErg{}    \INDSgDat{}   \NabilDSgDat{}   \INDSgAbs{}   \petitCSg{}   \coyoteCSgAbs{}  \montrerVdPstCSg{} \\
    \DEFSgOblP{}    \DEFSgOblP{}   \cuisineDSgOblP{}   \DEVANTP{}   \plaineASgOblP{}   \DANSP{}   \INDSgErgP{}   \grandCSgP{}   \chasseurCSgErgP{}    \INDSgDatP{}   \NabilDSgDatP{}   \INDSgAbsP{}   \petitCSgP{}   \coyoteCSgAbsP{}  \montrerVdPstCSgP{} \\
    \DEFSgOblG{}    \DEFSgOblG{}   \cuisineDSgOblG{}   \DEVANTG{}   \plaineASgOblG{}   \DANSG{}   \INDSgErgG{}   \grandCSgG{}   \chasseurCSgErgG{}    \INDSgDatG{}   \NabilDSgDatG{}   \INDSgAbsG{}   \petitCSgG{}   \coyoteCSgAbsG{}  \montrerVdPstCSgG{} \\
 Un grand chasseur montrait un petit coyote à Nabil dans la plaine devant la cuisine
\ex\glll
   \INDDuAbs{}   \grosDDu{}   \chatDDuAbs{}    \DEFSgObl{}    \DEFSgObl{}   \grandBSg{}   \chambreBSgObl{}   \DE{}   \petitDSg{}   \tableDSgObl{}   \SOUS{}  \dormirViPrsDDu{} \\
   \INDDuAbsP{}   \grosDDuP{}   \chatDDuAbsP{}    \DEFSgOblP{}    \DEFSgOblP{}   \grandBSgP{}   \chambreBSgOblP{}   \DEP{}   \petitDSgP{}   \tableDSgOblP{}   \SOUSP{}  \dormirViPrsDDuP{} \\
   \INDDuAbsG{}   \grosDDuG{}   \chatDDuAbsG{}    \DEFSgOblG{}    \DEFSgOblG{}   \grandBSgG{}   \chambreBSgOblG{}   \DEG{}   \petitDSgG{}   \tableDSgOblG{}   \SOUSG{}  \dormirViPrsDDuG{} \\
 Deux gros chats dorment sous la petite table de la grande chambre
\ex\glll
   \DEMDuAbs{}   \fruitADuAbs{}    \DEFSgObl{}   \plaineASgObl{}   \DE{}  \arriverViPrsADu{} \\
   \DEMDuAbsP{}   \fruitADuAbsP{}    \DEFSgOblP{}   \plaineASgOblP{}   \DEP{}  \arriverViPrsADuP{} \\
   \DEMDuAbsG{}   \fruitADuAbsG{}    \DEFSgOblG{}   \plaineASgOblG{}   \DEG{}  \arriverViPrsADuG{} \\
 Ces deux fruits arrivent de la plaine
\ex\glll
   \INDDuErg{}   \oeufCDuErg{}   \DEMSgAbs{}   \autrucheBSgAbs{}  \supporterVtPrsBSg{} \\
   \INDDuErgP{}   \oeufCDuErgP{}   \DEMSgAbsP{}   \autrucheBSgAbsP{}  \supporterVtPrsBSgP{} \\
   \INDDuErgG{}   \oeufCDuErgG{}   \DEMSgAbsG{}   \autrucheBSgAbsG{}  \supporterVtPrsBSgG{} \\
 Deux oeufs supportent cette autruche
\ex\glll
   \DEFSgAbs{}   \coussinBSgAbs{}    \DEFSgObl{}   \litDSgObl{}   \SUR{}  \arriverViPrsBSg{} \\
   \DEFSgAbsP{}   \coussinBSgAbsP{}    \DEFSgOblP{}   \litDSgOblP{}   \SURP{}  \arriverViPrsBSgP{} \\
   \DEFSgAbsG{}   \coussinBSgAbsG{}    \DEFSgOblG{}   \litDSgOblG{}   \SURG{}  \arriverViPrsBSgG{} \\
 Le coussin arrive sur le lit
\ex\glll
   \INDSgErg{}   \petitBSg{}   \blancBSg{}   \autrucheBSgErg{}    \DEFSgObl{}   \maisonDSgObl{}   \DEVANT{}   \INDSgAbs{}   \noirBSg{}   \theBSgAbs{}  \boireVtPrsBSg{} \\
   \INDSgErgP{}   \petitBSgP{}   \blancBSgP{}   \autrucheBSgErgP{}    \DEFSgOblP{}   \maisonDSgOblP{}   \DEVANTP{}   \INDSgAbsP{}   \noirBSgP{}   \theBSgAbsP{}  \boireVtPrsBSgP{} \\
   \INDSgErgG{}   \petitBSgG{}   \blancBSgG{}   \autrucheBSgErgG{}    \DEFSgOblG{}   \maisonDSgOblG{}   \DEVANTG{}   \INDSgAbsG{}   \noirBSgG{}   \theBSgAbsG{}  \boireVtPrsBSgG{} \\
 Une petite autruche blanche boit un thé noir devant la maison
\ex\glll
   \DEFPlErg{}   \litDPlErg{}   \DEFDuAbs{}   \grosCDu{}   \chasseurCDuAbs{}  \supporterVtPrsCDu{} \\
   \DEFPlErgP{}   \litDPlErgP{}   \DEFDuAbsP{}   \grosCDuP{}   \chasseurCDuAbsP{}  \supporterVtPrsCDuP{} \\
   \DEFPlErgG{}   \litDPlErgG{}   \DEFDuAbsG{}   \grosCDuG{}   \chasseurCDuAbsG{}  \supporterVtPrsCDuG{} \\
 Les lits supportent les deux gros chasseurs
\ex\glll
   \INDSgErg{}   \grosBSg{}   \autrucheBSgErg{}   \INDSgAbs{}   \petitDSg{}   \garconDSgAbs{}  \supporterVtPrsDSg{} \\
   \INDSgErgP{}   \grosBSgP{}   \autrucheBSgErgP{}   \INDSgAbsP{}   \petitDSgP{}   \garconDSgAbsP{}  \supporterVtPrsDSgP{} \\
   \INDSgErgG{}   \grosBSgG{}   \autrucheBSgErgG{}   \INDSgAbsG{}   \petitDSgG{}   \garconDSgAbsG{}  \supporterVtPrsDSgG{} \\
 Une grosse autruche supporte un petit garçon
\ex\glll
   \DEMDuErg{}   \garconDDuErg{}    \DEFPlObl{}   \chasseurCPlObl{}   \A{}   \INDSgAbs{}   \tableDSgAbs{}  \offrirVdPrsDSg{} \\
   \DEMDuErgP{}   \garconDDuErgP{}    \DEFPlOblP{}   \chasseurCPlOblP{}   \AP{}   \INDSgAbsP{}   \tableDSgAbsP{}  \offrirVdPrsDSgP{} \\
   \DEMDuErgG{}   \garconDDuErgG{}    \DEFPlOblG{}   \chasseurCPlOblG{}   \AG{}   \INDSgAbsG{}   \tableDSgAbsG{}  \offrirVdPrsDSgG{} \\
 Ces deux garçons offrent une table aux chasseurs
\ex\glll
    \DEFSgObl{}    \DEFSgObl{}   \cuisineDSgObl{}   \DE{}   \tableDSgObl{}   \SUR{}   \DEFPlErg{}   \infirmiereAPlErg{}    \INDSgDat{}   \KatishaASgDat{}   \DEFSgAbs{}    \DEMPlObl{}   \coyoteCPlObl{}   \DE{}   \viandeASgAbs{}  \lancerVdPrsASg{} \\
    \DEFSgOblP{}    \DEFSgOblP{}   \cuisineDSgOblP{}   \DEP{}   \tableDSgOblP{}   \SURP{}   \DEFPlErgP{}   \infirmiereAPlErgP{}    \INDSgDatP{}   \KatishaASgDatP{}   \DEFSgAbsP{}    \DEMPlOblP{}   \coyoteCPlOblP{}   \DEP{}   \viandeASgAbsP{}  \lancerVdPrsASgP{} \\
    \DEFSgOblG{}    \DEFSgOblG{}   \cuisineDSgOblG{}   \DEG{}   \tableDSgOblG{}   \SURG{}   \DEFPlErgG{}   \infirmiereAPlErgG{}    \INDSgDatG{}   \KatishaASgDatG{}   \DEFSgAbsG{}    \DEMPlOblG{}   \coyoteCPlOblG{}   \DEG{}   \viandeASgAbsG{}  \lancerVdPrsASgG{} \\
 Les infirmières lancent la viande de ces coyotes à Katisha sur la table de la cuisine
\ex\glll
    \DEFPlObl{}   \grosBPl{}   \noirBPl{}   \coussinBPlObl{}   \SUR{}   \DEFPlErg{}   \troisBPl{}   \maigreBPl{}   \autrucheBPlErg{}   \INDPlAbs{}   \blancCPl{}   \oeufCPlAbs{}  \donnerVdPstCPl{} \\
    \DEFPlOblP{}   \grosBPlP{}   \noirBPlP{}   \coussinBPlOblP{}   \SURP{}   \DEFPlErgP{}   \troisBPlP{}   \maigreBPlP{}   \autrucheBPlErgP{}   \INDPlAbsP{}   \blancCPlP{}   \oeufCPlAbsP{}  \donnerVdPstCPlP{} \\
    \DEFPlOblG{}   \grosBPlG{}   \noirBPlG{}   \coussinBPlOblG{}   \SURG{}   \DEFPlErgG{}   \troisBPlG{}   \maigreBPlG{}   \autrucheBPlErgG{}   \INDPlAbsG{}   \blancCPlG{}   \oeufCPlAbsG{}  \donnerVdPstCPlG{} \\
 Les trois maigres autruches donnaient des oeufs blancs sur les gros coussins noirs
\ex\glll
    \DEFPlObl{}   \chasseurCPlObl{}   \AVEC{}   \DEMDuErg{}   \infirmiereADuErg{}   \INDSgAbs{}   \cafeCSgAbs{}  \boireVtPrsCSg{} \\
    \DEFPlOblP{}   \chasseurCPlOblP{}   \AVECP{}   \DEMDuErgP{}   \infirmiereADuErgP{}   \INDSgAbsP{}   \cafeCSgAbsP{}  \boireVtPrsCSgP{} \\
    \DEFPlOblG{}   \chasseurCPlOblG{}   \AVECG{}   \DEMDuErgG{}   \infirmiereADuErgG{}   \INDSgAbsG{}   \cafeCSgAbsG{}  \boireVtPrsCSgG{} \\
 Ces deux infirmières boivent un café avec les chasseurs
\ex\glll
   \INDSgErg{}   \petitCSg{}   \filleCSgErg{}    \DEFSgObl{}   \grosCSg{}   \coyoteCSgObl{}   \A{}   \INDDuAbs{}   \viandeADuAbs{}  \lancerVdPrsADu{} \\
   \INDSgErgP{}   \petitCSgP{}   \filleCSgErgP{}    \DEFSgOblP{}   \grosCSgP{}   \coyoteCSgOblP{}   \AP{}   \INDDuAbsP{}   \viandeADuAbsP{}  \lancerVdPrsADuP{} \\
   \INDSgErgG{}   \petitCSgG{}   \filleCSgErgG{}    \DEFSgOblG{}   \grosCSgG{}   \coyoteCSgOblG{}   \AG{}   \INDDuAbsG{}   \viandeADuAbsG{}  \lancerVdPrsADuG{} \\
 Une petite fille lance deux viandes au gros coyote
\end{exe}


\begin{description}
\item [Votre mission :] Votre collègue a identifié qu'il y a des variations de forme pour la plupart des catégories mais il n'a pas  noté les paramètres qui régissent ces variations. Complétez son travail~:
\end{description}
\begin{enumerate}
\item Rétablissez l'écriture des formes  et les transcriptions manquantes.
\item Décrivez les phénomènes d'accord rencontrés dans le corpus.
\item Donnez des gloses pour chaque mot en tenant compte notamment de ces phénomènes
d'accord. 
%\begin{reponse}
%Plusieurs phénomènes d'accord sont visibles dans le corpus :
%\end{reponse}
\item Décrivez l'organisation générale de la phrase dans ce kalaba. \\
Pour chaque constituant, donnez l'ordre d'apparition des catégories par comparaison à celui du français indiqué ci-dessous sans tenir compte de l'optionnalité :
\vspace{-2ex}\setcounter{exx}{0}\begin{multicols}{2}
\ea Français
\ea Phrase \fldr{} GN GV GP
	\ex GV \fldr{} V GN GP
	\ex GN \fldr{} Dét N Adj GP
	\ex GP \fldr{} Prép GN
	\z
\ex Kalaba
	\ea Phrase \fldr{} ... ... ...
	\ex GV \fldr{} ... ... ...
	\ex GN \fldr{} ... ... ... ...
	\ex GP \fldr{} ... ...
	\z
%\ex Kalaba
%	\ea Phrase \fldr{} GV GN GP
%	\ex GV \fldr{} V GN GP
%	\ex GN \fldr{} GP Dét N Adj
%	\ex GP \fldr{} GN Prép
%	\z
\z
\end{multicols}\vspace{-2ex}
\end{enumerate}
\pagebreak	

\section{\textbf{Écriture de règle} (20\%)}
\noindent
Rédigez en quelques lignes la règle sur l'orthographe des numéraux cardinaux.
\smallskip	

\noindent
Il serait judicieux que votre règle rende compte de l'orthographe en toutes lettres des exemples suivants :
\begin{itemize}
\item[--] 21 pages
\item[--] 200 arbres
\item[--] 80 ans.
\end{itemize}

\noindent
Chaque point de la règle doit être illustré par un exemple \emph{pertinent}.
\smallskip	

\noindent
\textbf{N.B.}
\begin{itemize}
\item [--] Votre règle \emph{doit} dire quelque chose de cohérent à propos des numéraux cardinaux qui se terminent par 1, 20 et 100.
\item [--] 21, 253144 et leurs cousins constituent chacun un numéral cardinal. Ils sont constitués de composants plus petits dont certains sont des nombres...
\end{itemize}
\end{document}
