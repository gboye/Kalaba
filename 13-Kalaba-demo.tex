\documentclass[a4paper,12pt]{article}
\usepackage[utf8]{inputenc}
\usepackage{linearb}
%\usepackage{coffee4}
\usepackage{graphicx}
\usepackage{fourier}
\usepackage{tipa,booktabs,multicol,geometry,pifont,gb4e,natbib,relsize}

%\usepackage[T1]{fontenc}
%\usepackage{baskervald}


\geometry{scale={0.91,0.86}} % scale={largeur,hauteur}

\newcommand{\blanc}[1]{\textsc{#1}}
\newcommand{\cacherGloses}[1]{\tiny{}#1\normalsize{}}
\renewcommand{\blanc}[1]{\tiny{}#1\normalsize{}}
%\renewcommand{\blanc}[1]{}
\newcommand{\commentaire}[1]{}
\newcommand{\refp}[1]{(\ref{#1})}
\newcommand{\key}[1]{}
\newcommand{\strutgb}[1]{\rule[-#1]{0pt}{#1}}
\newcommand{\strutgh}[1]{\rule{0pt}{#1}}
\newcommand{\grapho}[1]{\footnotesize\textlinb{#1}}
%\renewcommand{\grapho}[1]{\tiny}

\newcommand{\blot}[2]{\includegraphics[scale=#2]{#1}}
\newcommand{\cachea}[1]{\raisebox{-8pt}[0pt][0pt]{\hspace{-2pt}\makebox[0pt][l]{\blot{blot1n.png}{.1}}}\hspace{2pt}#1}
\newcommand{\cacheb}[1]{\raisebox{-8pt}[0pt][0pt]{\hspace{-2pt}\makebox[0pt][l]{\blot{blot2.png}{.09}}}\hspace{2pt}#1}
\newcommand{\cachec}[1]{\raisebox{-12pt}[0pt][0pt]{\hspace{-10pt}\makebox[0pt][l]{\blot{blot3.png}{.17}}}\hspace{10pt}#1}
\newcommand{\cached}[1]{\raisebox{-10pt}[0pt][0pt]{\hspace{-9pt}\makebox[0pt][l]{\blot{blot4.png}{.3}}}\hspace{9pt}#1}
\newcommand{\cachee}[1]{\raisebox{-8pt}[0pt][0pt]{\hspace{-8pt}\makebox[0pt][l]{\blot{blot5.png}{.15}}}\hspace{8pt}#1}
\newcommand{\cachef}[1]{\raisebox{-14pt}[0pt][0pt]{\hspace{-8pt}\makebox[0pt][l]{\blot{blot6.png}{.15}}}\hspace{8pt}#1}
\newcommand{\cacheg}[1]{\raisebox{-8pt}[0pt][0pt]{\hspace{-0pt}\makebox[0pt][l]{\blot{blot7.png}{.12}}}\hspace{0pt}#1}
\newcommand{\cacheh}[1]{#1}
%\newcommand{\cachei}[1]{\raisebox{-8pt}[0pt][0pt]{\hspace{-2pt}\makebox[0pt][l]{\blot{blot1g.png}{.1}}}\hspace{2pt}#1}
%\newcommand{\cachej}[1]{\raisebox{-8pt}[0pt][0pt]{\hspace{-2pt}\makebox[0pt][l]{\blot{blot1d.png}{.1}}}\hspace{2pt}#1}

\let\gbtextipa=\textipa
\renewcommand{\textipa}[1]{\textsf{\gbtextipa{#1}}}

\newcommand{\textecursif}[1]{\fontfamily{pzc}\large\emph{#1}\fontfamily{ptm}}

\newenvironment{reponse}{\begin{quote}\begin{sffamily}}{\end{sffamily}\end{quote}}
\renewenvironment{reponse}[1]{}{}

\newcommand{\observation}[1]{\linebreak\parbox{.9\textwidth}{\noindent%
\textecursif{#1}%
\medskip}%parbox
}

\title{\vspace{-3\baselineskip}Description grammaticale}
% \author{}
\date{\vspace{-2\baselineskip}\Large Devoir à rendre \textbf{avant} le 11 décembre, 12h.\\
\normalsize une copie par groupe de 4 ou 3 personnes}
\begin{document}
\maketitle
%\pagestyle{empty}
\thispagestyle{empty}

\newcommand{\indice}[1]{\relsize{-2}{\ensuremath{_{\textnormal{#1}}}}}
\newcommand{\fldr}{\ensuremath{\longrightarrow}}


\section{\textbf{Analyse de corpus kalaba} (80\%)}\setcounter{exx}{0}%
%\vspace{-1\baselineskip}%
\noindent
Imaginez que vous faites partie d'une équipe chargée de fabriquer une grammaire du 
kalaba. Un de vos collègues, un linguiste débutant, s'est déjà rendu sur le terrain, 
il a constitué un échantillon et a presque terminé de transcrire les données.
\bigskip
%\key{%
%\pagebreak
%}


%: Lexique
\textbf{Lexique}\label{sect-lex}



\input{./14-Kalaba-Declarations}
\input{./14-Kalaba-Exemples}

\begin{description}
\item [Votre mission :] Votre collègue a identifié qu'il y a des variations de forme pour la plupart des catégories mais il n'a pas  noté les paramètres qui régissent ces variations. Complétez son travail~:
\end{description}
\begin{itemize}
\item NOMS\\[-3ex]
\begin{multicols}{2}
\begin{tabular}[t]{|l|l|l|}
\addlinespace[-1.0em]\hline
Mot & Roman & Glose  \\
\hline\strutgh{14pt}%
\infirmiereASgAbs & \infirmiereASgAbsP & \\
\infirmiereASgObl & \infirmiereASgOblP & \\
\infirmiereASgDat & \infirmiereASgDatP & \\
\infirmiereADuErg & \infirmiereADuErgP & \\
\infirmiereADuAbs & \infirmiereADuAbsP & \\
\infirmiereADuObl & \infirmiereADuOblP & \\
\infirmiereAPlErg & \infirmiereAPlErgP & \\
\infirmiereAPlAbs & \infirmiereAPlAbsP & \\
\infirmiereAPlObl & \infirmiereAPlOblP & \\
\infirmiereAPlDat & \infirmiereAPlDatP & \\
\KatishaASgErg & \KatishaASgErgP & \\
\KatishaASgObl & \KatishaASgOblP & \\
\KatishaASgDat & \KatishaASgDatP & \\
\plaineASgObl & \plaineASgOblP & \\
\plaineAPlObl & \plaineAPlOblP & \\
\viandeASgAbs & \viandeASgAbsP & \\
\viandeADuAbs & \viandeADuAbsP & \\
\viandeAPlAbs & \viandeAPlAbsP & \\
\balaiASgAbs & \balaiASgAbsP & \\
\balaiAPlErg & \balaiAPlErgP & \\
\balaiAPlAbs & \balaiAPlAbsP & \\
\balaiAPlObl & \balaiAPlOblP & \\
\fruitADuAbs & \fruitADuAbsP & \\
\fruitAPlAbs & \fruitAPlAbsP & \\
\filleCSgErg & \filleCSgErgP & \\
\filleCSgAbs & \filleCSgAbsP & \\
\filleCSgObl & \filleCSgOblP & \\
\filleCDuErg & \filleCDuErgP & \\
\filleCDuAbs & \filleCDuAbsP & \\
\filleCDuObl & \filleCDuOblP & \\
\filleCDuDat & \filleCDuDatP & \\
\filleCPlErg & \filleCPlErgP & \\
\filleCPlAbs & \filleCPlAbsP & \\
\filleCPlObl & \filleCPlOblP & \\
\filleCPlDat & \filleCPlDatP & \\
\hline\end{tabular}\\
\begin{tabular}[t]{|l|l|l|}
\addlinespace[-1.0em]\hline
Mot & Roman & Glose  \\
\hline\strutgh{14pt}%
\coyoteCSgAbs & \coyoteCSgAbsP & \\
\coyoteCSgDat & \coyoteCSgDatP & \\
\coyoteCDuErg & \coyoteCDuErgP & \\
\coyoteCDuAbs & \coyoteCDuAbsP & \\
\coyoteCPlErg & \coyoteCPlErgP & \\
\coyoteCPlAbs & \coyoteCPlAbsP & \\
\coyoteCPlObl & \coyoteCPlOblP & \\
\coyoteCPlDat & \coyoteCPlDatP & \\
\oeufCSgAbs & \oeufCSgAbsP & \\
\oeufCDuErg & \oeufCDuErgP & \\
\oeufCDuAbs & \oeufCDuAbsP & \\
\oeufCDuObl & \oeufCDuOblP & \\
\oeufCPlAbs & \oeufCPlAbsP & \\
\villageCSgObl & \villageCSgOblP & \\
\cafeCSgAbs & \cafeCSgAbsP & \\
\cafeCDuAbs & \cafeCDuAbsP & \\
\cafeCPlAbs & \cafeCPlAbsP & \\
\chasseurCSgErg & \chasseurCSgErgP & \\
\chasseurCSgAbs & \chasseurCSgAbsP & \\
\chasseurCSgObl & \chasseurCSgOblP & \\
\chasseurCDuErg & \chasseurCDuErgP & \\
\chasseurCDuAbs & \chasseurCDuAbsP & \\
\chasseurCDuObl & \chasseurCDuOblP & \\
\chasseurCDuDat & \chasseurCDuDatP & \\
\chasseurCPlErg & \chasseurCPlErgP & \\
\chasseurCPlAbs & \chasseurCPlAbsP & \\
\chasseurCPlObl & \chasseurCPlOblP & \\
\chasseurCPlDat & \chasseurCPlDatP & \\
\NicoleBSgErg & \NicoleBSgErgP & \\
\NicoleBSgAbs & \NicoleBSgAbsP & \\
\NicoleBSgObl & \NicoleBSgOblP & \\
\NicoleBSgDat & \NicoleBSgDatP & \\
\theBSgErg & \theBSgErgP & \\
\theBSgAbs & \theBSgAbsP & \\
\theBPlAbs & \theBPlAbsP & \\
\hline\end{tabular}\\
\begin{tabular}[t]{|l|l|l|}
\addlinespace[-1.0em]\hline
Mot & Roman & Glose  \\
\hline\strutgh{14pt}%
\sourisBSgErg & \sourisBSgErgP & \\
\sourisBSgAbs & \sourisBSgAbsP & \\
\sourisBSgObl & \sourisBSgOblP & \\
\sourisBDuErg & \sourisBDuErgP & \\
\sourisBDuAbs & \sourisBDuAbsP & \\
\sourisBPlErg & \sourisBPlErgP & \\
\sourisBPlAbs & \sourisBPlAbsP & \\
\sourisBPlObl & \sourisBPlOblP & \\
\chambreBSgObl & \chambreBSgOblP & \\
\chambreBDuObl & \chambreBDuOblP & \\
\chambreBPlErg & \chambreBPlErgP & \\
\chambreBPlObl & \chambreBPlOblP & \\
\autrucheBSgErg & \autrucheBSgErgP & \\
\autrucheBSgAbs & \autrucheBSgAbsP & \\
\autrucheBSgObl & \autrucheBSgOblP & \\
\autrucheBSgDat & \autrucheBSgDatP & \\
\autrucheBDuErg & \autrucheBDuErgP & \\
\autrucheBDuAbs & \autrucheBDuAbsP & \\
\autrucheBPlErg & \autrucheBPlErgP & \\
\autrucheBPlAbs & \autrucheBPlAbsP & \\
\autrucheBPlObl & \autrucheBPlOblP & \\
\autrucheBPlDat & \autrucheBPlDatP & \\
\coussinBSgErg & \coussinBSgErgP & \\
\coussinBSgAbs & \coussinBSgAbsP & \\
\coussinBSgObl & \coussinBSgOblP & \\
\coussinBDuAbs & \coussinBDuAbsP & \\
\coussinBPlAbs & \coussinBPlAbsP & \\
\coussinBPlObl & \coussinBPlOblP & \\
\maisonDSgObl & \maisonDSgOblP & \\
\maisonDDuObl & \maisonDDuOblP & \\
\tableDSgErg & \tableDSgErgP & \\
\tableDSgAbs & \tableDSgAbsP & \\
\tableDSgObl & \tableDSgOblP & \\
\tableDDuErg & \tableDDuErgP & \\
\hline\end{tabular}\\
\begin{tabular}[t]{|l|l|l|}
\addlinespace[-1.0em]\hline
Mot & Roman & Glose  \\
\hline\strutgh{14pt}%
\tableDDuObl & \tableDDuOblP & \\
\tableDPlObl & \tableDPlOblP & \\
\garconDSgErg & \garconDSgErgP & \\
\garconDSgAbs & \garconDSgAbsP & \\
\garconDSgDat & \garconDSgDatP & \\
\garconDDuErg & \garconDDuErgP & \\
\garconDDuAbs & \garconDDuAbsP & \\
\garconDDuObl & \garconDDuOblP & \\
\garconDPlErg & \garconDPlErgP & \\
\garconDPlAbs & \garconDPlAbsP & \\
\garconDPlObl & \garconDPlOblP & \\
\litDSgErg & \litDSgErgP & \\
\litDSgAbs & \litDSgAbsP & \\
\litDSgObl & \litDSgOblP & \\
\litDPlErg & \litDPlErgP & \\
\litDPlAbs & \litDPlAbsP & \\
\cuisineDSgAbs & \cuisineDSgAbsP & \\
\cuisineDSgObl & \cuisineDSgOblP & \\
\NabilDSgErg & \NabilDSgErgP & \\
\NabilDSgAbs & \NabilDSgAbsP & \\
\NabilDSgObl & \NabilDSgOblP & \\
\NabilDSgDat & \NabilDSgDatP & \\
\chatDSgErg & \chatDSgErgP & \\
\chatDSgAbs & \chatDSgAbsP & \\
\chatDSgObl & \chatDSgOblP & \\
\chatDSgDat & \chatDSgDatP & \\
\chatDDuErg & \chatDDuErgP & \\
\chatDDuAbs & \chatDDuAbsP & \\
\chatDDuObl & \chatDDuOblP & \\
\chatDDuDat & \chatDDuDatP & \\
\chatDPlErg & \chatDPlErgP & \\
\chatDPlAbs & \chatDPlAbsP & \\
\chatDPlDat & \chatDPlDatP & \\
\hline\end{tabular}\\
\end{multicols}
\item ADJECTIFS\\[-3ex]
\begin{multicols}{3}
\begin{tabular}[t]{|l|l|l|}
\addlinespace[-1.0em]\hline
Mot & Roman & Glose  \\
\hline\strutgh{14pt}%
\noirBSg & \noirBSgP & \\
\noirBPl & \noirBPlP & \\
\noirCSg & \noirCSgP & \\
\noirCDu & \noirCDuP & \\
\noirCPl & \noirCPlP & \\
\noirDSg & \noirDSgP & \\
\noirDPl & \noirDPlP & \\
\grandASg & \grandASgP & \\
\grandBSg & \grandBSgP & \\
\grandBPl & \grandBPlP & \\
\grandCSg & \grandCSgP & \\
\hline\end{tabular}\\
\begin{tabular}[t]{|l|l|l|}
\addlinespace[-1.0em]\hline
Mot & Roman & Glose  \\
\hline\strutgh{14pt}%
\grandCDu & \grandCDuP & \\
\grandCPl & \grandCPlP & \\
\grandDSg & \grandDSgP & \\
\grandDDu & \grandDDuP & \\
\grandDPl & \grandDPlP & \\
\petitASg & \petitASgP & \\
\petitBSg & \petitBSgP & \\
\petitBDu & \petitBDuP & \\
\petitBPl & \petitBPlP & \\
\petitCSg & \petitCSgP & \\
\petitCDu & \petitCDuP & \\
\hline\end{tabular}\\
\begin{tabular}[t]{|l|l|l|}
\addlinespace[-1.0em]\hline
Mot & Roman & Glose  \\
\hline\strutgh{14pt}%
\petitCPl & \petitCPlP & \\
\petitDSg & \petitDSgP & \\
\petitDPl & \petitDPlP & \\
\blancBSg & \blancBSgP & \\
\blancBDu & \blancBDuP & \\
\blancBPl & \blancBPlP & \\
\blancCSg & \blancCSgP & \\
\blancCPl & \blancCPlP & \\
\blancDSg & \blancDSgP & \\
\blancDDu & \blancDDuP & \\
\blancDPl & \blancDPlP & \\
\hline\end{tabular}\\
\begin{tabular}[t]{|l|l|l|}
\addlinespace[-1.0em]\hline
Mot & Roman & Glose  \\
\hline\strutgh{14pt}%
\basDSg & \basDSgP & \\
\quatreAPl & \quatreAPlP & \\
\quatreCPl & \quatreCPlP & \\
\troisBPl & \troisBPlP & \\
\troisCPl & \troisCPlP & \\
\troisDPl & \troisDPlP & \\
\jauneASg & \jauneASgP & \\
\jauneADu & \jauneADuP & \\
\jauneAPl & \jauneAPlP & \\
\jauneCDu & \jauneCDuP & \\
\hline\end{tabular}\\
\begin{tabular}[t]{|l|l|l|}
\addlinespace[-1.0em]\hline
Mot & Roman & Glose  \\
\hline\strutgh{14pt}%
\jauneCPl & \jauneCPlP & \\
\jauneDSg & \jauneDSgP & \\
\jauneDDu & \jauneDDuP & \\
\rougeAPl & \rougeAPlP & \\
\rougeBPl & \rougeBPlP & \\
\rougeDSg & \rougeDSgP & \\
\grosBSg & \grosBSgP & \\
\grosBDu & \grosBDuP & \\
\grosBPl & \grosBPlP & \\
\grosCSg & \grosCSgP & \\
\hline\end{tabular}\\
\begin{tabular}[t]{|l|l|l|}
\addlinespace[-1.0em]\hline
Mot & Roman & Glose  \\
\hline\strutgh{14pt}%
\grosCDu & \grosCDuP & \\
\grosCPl & \grosCPlP & \\
\grosDSg & \grosDSgP & \\
\grosDDu & \grosDDuP & \\
\maigreBPl & \maigreBPlP & \\
\maigreCDu & \maigreCDuP & \\
\maigreCPl & \maigreCPlP & \\
\maigreDSg & \maigreDSgP & \\
\maigreDPl & \maigreDPlP & \\
\hline\end{tabular}\\
\end{multicols}
\item VERBES\\[-3ex]
\begin{multicols}{2}
\begin{tabular}[t]{|l|l|l|}
\addlinespace[-1.0em]\hline
Mot & Roman & Glose  \\
\hline\strutgh{14pt}%
\tomberViPrsBSg & \tomberViPrsBSgP & \\
\tomberViPrsBPl & \tomberViPrsBPlP & \\
\tomberViPrsCSg & \tomberViPrsCSgP & \\
\tomberViPrsCDu & \tomberViPrsCDuP & \\
\tomberViPrsDSg & \tomberViPrsDSgP & \\
\tomberViPstCDu & \tomberViPstCDuP & \\
\tomberViPstCPl & \tomberViPstCPlP & \\
\tomberViPstDSg & \tomberViPstDSgP & \\
\entrerViPrsADu & \entrerViPrsADuP & \\
\entrerViPrsBSg & \entrerViPrsBSgP & \\
\entrerViPrsCSg & \entrerViPrsCSgP & \\
\arriverViPrsADu & \arriverViPrsADuP & \\
\arriverViPrsAPl & \arriverViPrsAPlP & \\
\arriverViPrsBSg & \arriverViPrsBSgP & \\
\arriverViPrsBPl & \arriverViPrsBPlP & \\
\arriverViPrsCDu & \arriverViPrsCDuP & \\
\arriverViPrsDSg & \arriverViPrsDSgP & \\
\arriverViPrsDPl & \arriverViPrsDPlP & \\
\arriverViPstBSg & \arriverViPstBSgP & \\
\arriverViPstBPl & \arriverViPstBPlP & \\
\arriverViPstCDu & \arriverViPstCDuP & \\
\dormirViPrsAPl & \dormirViPrsAPlP & \\
\dormirViPrsBDu & \dormirViPrsBDuP & \\
\dormirViPrsBPl & \dormirViPrsBPlP & \\
\dormirViPrsCSg & \dormirViPrsCSgP & \\
\dormirViPrsCPl & \dormirViPrsCPlP & \\
\dormirViPrsDDu & \dormirViPrsDDuP & \\
\dormirViPrsDPl & \dormirViPrsDPlP & \\
\dormirViPstCDu & \dormirViPstCDuP & \\
\dormirViPstCPl & \dormirViPstCPlP & \\
\dormirViPstDSg & \dormirViPstDSgP & \\
\boireVtPrsBSg & \boireVtPrsBSgP & \\
\boireVtPrsCSg & \boireVtPrsCSgP & \\
\boireVtPrsCDu & \boireVtPrsCDuP & \\
\boireVtPrsCPl & \boireVtPrsCPlP & \\
\hline\end{tabular}\\
\begin{tabular}[t]{|l|l|l|}
\addlinespace[-1.0em]\hline
Mot & Roman & Glose  \\
\hline\strutgh{14pt}%
\boireVtPstBSg & \boireVtPstBSgP & \\
\boireVtPstBPl & \boireVtPstBPlP & \\
\boireVtPstCSg & \boireVtPstCSgP & \\
\supporterVtPrsASg & \supporterVtPrsASgP & \\
\supporterVtPrsAPl & \supporterVtPrsAPlP & \\
\supporterVtPrsBSg & \supporterVtPrsBSgP & \\
\supporterVtPrsBDu & \supporterVtPrsBDuP & \\
\supporterVtPrsBPl & \supporterVtPrsBPlP & \\
\supporterVtPrsCDu & \supporterVtPrsCDuP & \\
\supporterVtPrsCPl & \supporterVtPrsCPlP & \\
\supporterVtPrsDSg & \supporterVtPrsDSgP & \\
\supporterVtPrsDPl & \supporterVtPrsDPlP & \\
\supporterVtPstBDu & \supporterVtPstBDuP & \\
\supporterVtPstCPl & \supporterVtPstCPlP & \\
\supporterVtPstDSg & \supporterVtPstDSgP & \\
\supporterVtPstDDu & \supporterVtPstDDuP & \\
\acheterVtPrsBDu & \acheterVtPrsBDuP & \\
\acheterVtPrsCPl & \acheterVtPrsCPlP & \\
\acheterVtPrsDSg & \acheterVtPrsDSgP & \\
\acheterVtPstAPl & \acheterVtPstAPlP & \\
\acheterVtPstCPl & \acheterVtPstCPlP & \\
\acheterVtPstDSg & \acheterVtPstDSgP & \\
\mangerVtPrsBDu & \mangerVtPrsBDuP & \\
\mangerVtPrsBPl & \mangerVtPrsBPlP & \\
\mangerVtPrsCSg & \mangerVtPrsCSgP & \\
\mangerVtPrsCDu & \mangerVtPrsCDuP & \\
\mangerVtPrsCPl & \mangerVtPrsCPlP & \\
\mangerVtPstAPl & \mangerVtPstAPlP & \\
\mangerVtPstBSg & \mangerVtPstBSgP & \\
\mangerVtPstCDu & \mangerVtPstCDuP & \\
\chasserVtPrsBPl & \chasserVtPrsBPlP & \\
\chasserVtPrsCDu & \chasserVtPrsCDuP & \\
\chasserVtPrsCPl & \chasserVtPrsCPlP & \\
\chasserVtPrsDPl & \chasserVtPrsDPlP & \\
\chasserVtPstBSg & \chasserVtPstBSgP & \\
\hline\end{tabular}\\
\begin{tabular}[t]{|l|l|l|}
\addlinespace[-1.0em]\hline
Mot & Roman & Glose  \\
\hline\strutgh{14pt}%
\chasserVtPstBPl & \chasserVtPstBPlP & \\
\donnerVdPrsASg & \donnerVdPrsASgP & \\
\donnerVdPrsAPl & \donnerVdPrsAPlP & \\
\donnerVdPrsBDu & \donnerVdPrsBDuP & \\
\donnerVdPrsCSg & \donnerVdPrsCSgP & \\
\donnerVdPrsDPl & \donnerVdPrsDPlP & \\
\donnerVdPstAPl & \donnerVdPstAPlP & \\
\donnerVdPstBPl & \donnerVdPstBPlP & \\
\donnerVdPstCPl & \donnerVdPstCPlP & \\
\donnerVdPstDSg & \donnerVdPstDSgP & \\
\lancerVdPrsASg & \lancerVdPrsASgP & \\
\lancerVdPrsADu & \lancerVdPrsADuP & \\
\lancerVdPrsCSg & \lancerVdPrsCSgP & \\
\lancerVdPstASg & \lancerVdPstASgP & \\
\lancerVdPstBSg & \lancerVdPstBSgP & \\
\hline\end{tabular}\\
\begin{tabular}[t]{|l|l|l|}
\addlinespace[-1.0em]\hline
Mot & Roman & Glose  \\
\hline\strutgh{14pt}%
\lancerVdPstCPl & \lancerVdPstCPlP & \\
\offrirVdPrsAPl & \offrirVdPrsAPlP & \\
\offrirVdPrsDSg & \offrirVdPrsDSgP & \\
\offrirVdPstBDu & \offrirVdPstBDuP & \\
\offrirVdPstBPl & \offrirVdPstBPlP & \\
\offrirVdPstCSg & \offrirVdPstCSgP & \\
\montrerVdPrsASg & \montrerVdPrsASgP & \\
\montrerVdPrsCPl & \montrerVdPrsCPlP & \\
\montrerVdPrsDSg & \montrerVdPrsDSgP & \\
\montrerVdPstBSg & \montrerVdPstBSgP & \\
\montrerVdPstBPl & \montrerVdPstBPlP & \\
\montrerVdPstCSg & \montrerVdPstCSgP & \\
\montrerVdPstDSg & \montrerVdPstDSgP & \\
\hline\end{tabular}\\
\end{multicols}
\item DÉTERMINANTS\\[-3ex]
\begin{multicols}{3}
\begin{tabular}[t]{|l|l|l|}
\addlinespace[-1.0em]\hline
Mot & Roman & Glose  \\
\hline\strutgh{14pt}%
\INDSgErg & \INDSgErgP & \\
\INDSgAbs & \INDSgAbsP & \\
\INDSgObl & \INDSgOblP & \\
\INDSgDat & \INDSgDatP & \\
\INDDuErg & \INDDuErgP & \\
\INDDuAbs & \INDDuAbsP & \\
\INDDuObl & \INDDuOblP & \\
\INDDuDat & \INDDuDatP & \\
\INDPlErg & \INDPlErgP & \\
\INDPlAbs & \INDPlAbsP & \\
\INDPlObl & \INDPlOblP & \\
\INDPlDat & \INDPlDatP & \\
\hline\end{tabular}\\
\begin{tabular}[t]{|l|l|l|}
\addlinespace[-1.0em]\hline
Mot & Roman & Glose  \\
\hline\strutgh{14pt}%
\DEFSgErg & \DEFSgErgP & \\
\DEFSgAbs & \DEFSgAbsP & \\
\DEFSgObl & \DEFSgOblP & \\
\DEFSgDat & \DEFSgDatP & \\
\DEFDuErg & \DEFDuErgP & \\
\DEFDuAbs & \DEFDuAbsP & \\
\DEFDuObl & \DEFDuOblP & \\
\DEFDuDat & \DEFDuDatP & \\
\DEFPlErg & \DEFPlErgP & \\
\DEFPlAbs & \DEFPlAbsP & \\
\DEFPlObl & \DEFPlOblP & \\
\DEFPlDat & \DEFPlDatP & \\
\hline\end{tabular}\\
\begin{tabular}[t]{|l|l|l|}
\addlinespace[-1.0em]\hline
Mot & Roman & Glose  \\
\hline\strutgh{14pt}%
\DEMSgErg & \DEMSgErgP & \\
\DEMSgAbs & \DEMSgAbsP & \\
\DEMSgObl & \DEMSgOblP & \\
\DEMSgDat & \DEMSgDatP & \\
\DEMDuErg & \DEMDuErgP & \\
\DEMDuAbs & \DEMDuAbsP & \\
\DEMDuObl & \DEMDuOblP & \\
\DEMDuDat & \DEMDuDatP & \\
\DEMPlErg & \DEMPlErgP & \\
\DEMPlAbs & \DEMPlAbsP & \\
\DEMPlObl & \DEMPlOblP & \\
\DEMPlDat & \DEMPlDatP & \\
\hline\end{tabular}\\
\end{multicols}
\item PRÉPOSITIONS\\[-3ex]
\begin{multicols}{3}
\begin{tabular}[t]{|l|l|l|}
\addlinespace[-1.0em]\hline
Mot & Roman & Glose  \\
\hline\strutgh{14pt}%
\POUR & \POURP & \\
\DEVANT & \DEVANTP & \\
\SUR & \SURP & \\
\hline\end{tabular}\\
\begin{tabular}[t]{|l|l|l|}
\addlinespace[-1.0em]\hline
Mot & Roman & Glose  \\
\hline\strutgh{14pt}%
\DANS & \DANSP & \\
\SOUS & \SOUSP & \\
\AVEC & \AVECP & \\
\hline\end{tabular}\\
\begin{tabular}[t]{|l|l|l|}
\addlinespace[-1.0em]\hline
Mot & Roman & Glose  \\
\hline\strutgh{14pt}%
\DE & \DEP & \\
\hline\end{tabular}\\
\end{multicols}
\end{itemize}

\begin{enumerate}
\item Rétablissez l'écriture des formes  et les transcriptions manquantes.
\item Décrivez les phénomènes d'accord rencontrés dans le corpus.
\item Donnez des gloses pour chaque mot en tenant compte notamment de ces phénomènes
d'accord. 
%\begin{reponse}
%Plusieurs phénomènes d'accord sont visibles dans le corpus :
%\end{reponse}
\item Décrivez l'organisation générale de la phrase dans ce kalaba. \\
Pour chaque constituant, donnez l'ordre d'apparition des catégories par comparaison à celui du français indiqué ci-dessous sans tenir compte de l'optionnalité :
\vspace{-2ex}\setcounter{exx}{0}\begin{multicols}{2}
\ea Français
\ea Phrase \fldr{} GN GV GP
	\ex GV \fldr{} V GN GP
	\ex GN \fldr{} Dét N Adj GP
	\ex GP \fldr{} Prép GN
	\z
\ex Kalaba
	\ea Phrase \fldr{} ... ... ...
	\ex GV \fldr{} ... ... ...
	\ex GN \fldr{} ... ... ... ...
	\ex GP \fldr{} ... ...
	\z
%\ex Kalaba
%	\ea Phrase \fldr{} GV GN GP
%	\ex GV \fldr{} V GN GP
%	\ex GN \fldr{} GP Dét N Adj
%	\ex GP \fldr{} GN Prép
%	\z
\z
\end{multicols}\vspace{-2ex}
\end{enumerate}
\pagebreak	

\section{\textbf{Écriture de règle} (20\%)}
\noindent
Rédigez en quelques lignes la règle sur l'orthographe des numéraux cardinaux.
\smallskip	

\noindent
Il serait judicieux que votre règle rende compte de l'orthographe en toutes lettres des exemples suivants :
\begin{itemize}
\item[--] 21 pages
\item[--] 200 arbres
\item[--] 80 ans.
\end{itemize}

\noindent
Chaque point de la règle doit être illustré par un exemple \emph{pertinent}.
\smallskip	

\noindent
\textbf{N.B.}
\begin{itemize}
\item [--] Votre règle \emph{doit} dire quelque chose de cohérent à propos des numéraux cardinaux qui se terminent par 1, 20 et 100.
\item [--] 21, 253144 et leurs cousins constituent chacun un numéral cardinal. Ils sont constitués de composants plus petits dont certains sont des nombres...
\end{itemize}
\end{document}
